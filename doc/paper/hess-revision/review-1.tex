\documentclass[12pt]{article}
% \usepackage{amsmath}
% \usepackage{amssymb}
% \usepackage{bm}
% \usepackage{graphicx}
% \usepackage{epstopdf}
% \usepackage{xcolor}
% \DeclareGraphicsRule{.tif}{png}{.png}{`convert #1 `basename #1 .tif`.png}
% \usepackage{color}
% \pagestyle{plain}

% \usepackage[
% 	backend=biber,
% 	url=false,
% 	isbn=false,
% 	doi=false,
%         eprint=false,
%         bibstyle=authoryear,
%         style=authoryear,
%         % style=numeric-comp,
% 	% uniquename=false,
% 	% uniquelist=false,
% 	% firstinits=true,
%  	natbib=true,
% 	sorting=none,
%         ]{biblatex}

% \addbibresource{./references.bib}

\RequirePackage{natbib}
\bibliographystyle{agufull08}

% Bib aliases
\makeatletter
\def\@citex[#1]#2{\leavevmode
  \let\@citea\@empty
  \@cite{\@for\@citeb:=#2\do
    {\@citea\def\@citea{,\penalty\@m\ }%
\edef\magic##1{\let##1\expandafter\noexpand\csname bibalias@\@citeb\endcsname}%
\magic\tmp \ifx\tmp\relax\else \let\@citeb\tmp\fi
     \edef\@citeb{\expandafter\@firstofone\@citeb\@empty}%
     \if@filesw\immediate\write\@auxout{\string\citation{\@citeb}}\fi
     \@ifundefined{b@\@citeb}{\hbox{\reset@font\bfseries ?}%
       \G@refundefinedtrue
       \@latex@warning
         {Citation `\@citeb' on page \thepage \space undefined}}%
       {\@cite@ofmt{\csname b@\@citeb\endcsname}}}}{#1}}
\def\bibalias#1#2{\expandafter\def\csname bibalias@#1\endcsname{#2}}
\makeatother

\title{Response to RC1 of ``When does vapor pressure deficit drive or
  reduce evapotranspiration?''}

\author{Adam Massmann, Pierre Gentine, and Changjie Lin}

\date{December 17th, 2018}

\begin{document}

Thank you very much for an excellent and very thoughtful
review. Addressing your comments will greatly improve the
manuscript. We agree with your comments, and think the manuscript
could use substantial re-framing and rewording to clarify how we are
answering our research question, and how the answer may vary with
climate and environmental factors.

Just a quick general comment before addressing your specific
comments. By including a PFT-focused analysis we did not fully
communicate the major goal and scope of our project: we are trying to
characterize the response of ET to VPD, with all other environment
variables held fixed. To accomplish this, we need to formulate an
explicit function of ET in terms of environmental variables and
parameters, where any parameters can be approximated as constant with
regards to some [arbitrary] VPD perturbation scenario. We will
elaborate more below on why we used PFT-focused scenarios for much of
our analysis. However, the goal of our manuscript was much more simple
and fundamental: at a given place or time, if you introduce a
perturbation to VPD, what is the immediate ET response (e.g. positive
or negative)? Answering this question does not require the much
stronger assertion that any parameters must be invariant within a
given PFT.

\section{Compounding uncertainties}

We agree that the uncertainties are large both within a PFT and across
PFTs. Looking back, we believe that some of our language
communicating \cite{Lin_2018}, \cite{Medlyn_2017}, and
\cite{Zhou_2015}'s results was misleading, and this was exacerbated by
our focus on PFT-analysis. We will remove that language from the
manuscript, and add language to better communicate our results as
reflections of the considerable within-PFT uncertainty.

We think comparing the rankings of given PFT values for g$_1$ and uWUE
can be misleading, given the magnitude of the uncertainties
involved. Most of our calculated values for uWUE are within one
standard deviation of \cite{Zhou_2015}'s results, but these deviations
can result in some changes in the ordering of PFTs from high to low
uWUE. The bigger problem in our eyes is that we made a mistake with
some language suggesting within intra-PFT variability is less than
inter-PFT variability, but clearly this is not the case. We will
remove this language. This misinterpretation should not have been in
the manuscript and actually contradicts other manuscript content; for
example, we included \cite{Zhou_2015}'s results on uWUE in Table 2
explicitly to be transparent about the within-PFT uncertainties. g$_1$
values also exhibit considerable uncertainty, and again, while the
relative rankings of these values may change, they are all within
observed ranges in \cite{Medlyn_2017}. We did not include these in
Table 2 because \cite{Medlyn_2017} does not provide the numerical
values; however, we think it would be useful to provide estimates of
these ranges from \cite{Medlyn_2017}'s figures in our Table 2.

Regarding how to interpret g$_1$ and uWUE: we think the best way
is with established physical relations from \cite{MEDLYN_2011}
(Equation 3 in the manuscript) and \cite{Zhou_2014} (Pg. 7, line 10 in
the manuscript), with the knowledge that there is uncertainty
involved. The quantities in these relationships all have some
intrinsic physical meaning, but can vary substantially within a
PFT. Again, we need to alter our language to better reflect this
intra-PFT variability.

However, \cite{Zhou_2014} did establish a constant uWUE approximation
as a good approximation for capturing the relationship between GPP, ET
and VPD at a given place and time, so it is still a very useful and
robust approximation for answering our research question (see
discussion in the introduction of this comment). Additionally, a constant g$_1$
approximation is used in many earth system models, so while it
introduces uncertainty, it also makes our framework useful for
interpreting modeled vegetation response in ESMs and GCMs.

We also agree with the comment that the best way to think of this is
probabilistically. This is the most robust approach to dealing with
approximations - we introduce randomness to our variables and
parameters to account for all of the physics that are not explicitly
accounted for, as well as observational uncertainty. However, as far
as we know we still have not developed a general, robust, and efficient
arithmetic for random variables. We could try and adapt a Bayesian
model to this problem, but given the large amount of arithmetic
involved, fully incorporating a Bayesian representation to every
variable in the analysis would be a very hard problem, and a
significant research project in its own right. We think a good
compromise is to add language directing the reader to interpret our
results more probabilistically, which we have already presented
probabilistically in the figures. For example, in Figure 5 the range of
values in each plot represents a range of possible responses in the
sign term. Within this figure, it's worth noting that the intra-PFT
variability is greater than the inter-PFT variability, which is
consistent with some of the previous results you highlight. Again, we
need to add language highlighting this, and its consistency with
previous results. We included this variability and uncertainty
explicitly to be transparent. A more difficult question is how much of
this variability is due to observational and model error, and how much
of it is due to climate and plant physiological variability (see
Section 3.5).

\section{Attribution to physiological responses}

We do not want our analysis to be interpreted as an assertion that
g$_1$ and uWUE are attributes of PFT only. We expect them to vary both
within a PFT and across a PFT. Our primary goal in using g$_1$ and
uWUE was to develop an explicit expression of ET as a function of
environmental variables, and use this to assess the response to a
change in VPD. We focused our analysis using PFTs because this is how
it was framed in previous studies in the field, and specifically the
studies used in our derivation \citep{Zhou_2014, Zhou_2015,
  Medlyn_2017}. We were originally thinking the PFT-focused analysis
could be useful, especially given that climate models generally hold
plant parameters fixed with respect to PFT, so long as we were
transparent about the large uncertainties and problems with this
approach (see Figure 5).

We agree with you that stating that any quantity is fixed within a PFT
is hard to believe. Phenotypic variation and adaptation within a given
\textit{species} can be considerable (i.e. effecting $\lambda$, g$_1$,
and uWUE), so it would be hard to say that anything would be constant
within a PFT made up of many different diverse species. Both
phenotypic variation as well as the species distribution and dominant
PFT at a given location will all be strongly optimized in response to
climate. In this sense, the distribution and evolution of local
climate is a strong control on the local structure and physiology of a
given ecosystem.

What we need to communicate better is that a given ecosystem's state
at a place or time controls its response to a VPD perturbation. We are
making an approximation that we can parameterize the effect of the
ecosystem's state on VPD response with g$_1$ and uWUE. These quantities
can also vary due to soil moisture condition; the approximation we
need to answer our research question is that they are fixed
with respect to a VPD perturbation. In this sense, we do not view the
two statements: ``plants that are evolved to bred to prioritize
primary production over water conservation (e.g., crops) exhibit a
higher likelihood of atmospheric demand-driven response'' and
``ecosystem types are responding in this way because they have, on the
whole, been subjected to less soil water limitation (due to the
non-negligible effects of irrigation?)'' as mutually exclusive. In
fact, we view them as consistent with a view that crops and their
physiology (parameterized by g$_1$ and uWUE) exist at a given time and
place because they have not been subjected to soil water limitation,
and they have a given response to VPD because of their physiology
(which is a direct effect of the environment). In this way, climate
and land surface state are causes of the VPD response both directly
and through their effect on plant physiology (parameterized by g$_1$
and uWUE).

When writing this manuscript, there was definitely some internal
tension and debate about how to best frame the analysis and
results. We could either focus on PFT-oriented results as previous
literature has done (e.g. holding plant physiology fixed within a
given PFT), or allow plant physiology terms to vary through time and
space and look at the distribution of ET response to VPD (see Section
3.5). After re-examining the manuscript both after some time away from
the problem and in light of your comments we think a strong argument
could be made that we made the wrong choice with respect to this
focus. It may have made more sense to focus our analysis on the ET
response more generally across space and time as ecosystem-scale plant
physiology varies in response to climate and soil moisture.

\section{Next Steps}

In order to improve the manuscript and our communication of the answer to
the question ``When does VPD drive or reduce ET?'', we see a few
potential paths that we will consider between now and the final
response after the discussion period. If you (or anyone else) has any
opinions or comments in the meantime, we would appreciate the insight
and feedback.

\begin{itemize}
\item \textbf{Option 1:} We include the discussion presented in this
  review and response, and include language explicitly stating that
  the purpose of the PFT analysis is to provide connections to other
  PFT-constant analyses and models (e.g. ESMs and GCMs). We will add
  extensive language on the uncertainty and weaknesses of this
  approach, and reframe our conclusions to reflect his uncertainty.
\item \textbf{Option 2:} We instead alter our analysis to look at how
  VPD response varies with climate and general plant physiological
  variability, instead of focusing only on PFT analysis that poorly
  captures all of the observed variability in ET response. Sections
  3.1 - 3.2 would be replaced by analysis directly relating ET
  response to general climate and plant physiological terms, rather
  than PFT-mean analysis. For example, the ``scaling term'' analysis
  would be presented in terms of generic changes in plant height and
  temperature, and the ``sign term'' analysis would be framed by
  generic changes to g$_1$ and uWUE, as informed by previous
  literature.
\item \textbf{Option 3:} This is the most extreme option in terms of
  modifying the manuscript. We significantly alter the manuscript and
  instead focus on just the general shape of the ET-VPD curve (with
  environmental variables held fixed). This was not discussed in the
  review, but one of our most noteworthy results is on the general
  shape of the ET-VPD curve being concave up, given an assumption of a
  square-root VPD dependence of ET. This result is independent of any
  assumptions of uWUE and g$_1$, and to our knowledge it is first
  derived curve of ecosystem response to VPD (see Section 3.7). It
  also highlights the importance of discerning the exact exponent of
  VPD dependence, as this alters the fundamental nature and shape of
  the curve. Essentially, the manuscript would become just our
  motivation and derivation, followed by Section 3.7 exploring the
  consequences of the derivation for the shape of the VPD curve. This
  alteration of the manuscript would likely result in more of a
  technical note-type paper, and we are hesitant to do this because we
  think there is still a lot of useful information obtained by tying
  our results to real-world scenarios (as in Options 1 and 2 above).
\end{itemize}

Here we present some final minor comments and concerns on the
technical details of our proposed changes to the manuscript, which are
relevant to representing uncertainty and spatiotemporal variability in
uWUE \textit{and} g$_1$. In the original manuscript we used a single
$\sigma$ term to represent this variability. We did this because
changes in uWUE and g$_1$ induce a very similar change in the ET
solution, which made solving for independent $\sigma_{uWUE}$ and
$\sigma_{g1}$ terms intractable at a given time and place, and
representing variability in both g$_1$ and uWUE difficult. We decided
to hold g$_1$ fixed within a PFT for a two reasons: 1) ESMs and GCMs
generally hold g$_1$ fixed, and 2) letting uWUE vary seemed more
appropriate to represent specifically soil water variations on
stomatal conductance, as uWUE modifies stomatal conductance
analogously to how soil moisture factors modify a maximum stomatal
conductance in land surface models (e.g. it is a multiplicative factor
on the entire stomatal conductance term). Because uWUE and g$_1$
induce similar changes in ET, at least qualitatively we think that
having a single $\sigma$ can represent some of the variability in both
$uWUE$ and g$_1$. However, formally there is no explicit variability
in the g$_1$ term. The discussion in this review rightly points out
that based on previous results we do expect some variability in g$_1$
as well. So, an unresolved question is how important is it to the
analysis and its interpretation to include an explicit g$_1$
variability term in addition to the existing $\sigma$ term. We are
including these comments to hopefully stimulate some discussion on the
importance of explicitly representing g$_1$ variability, given the
difficulties of doing so within our framework. However, we could
imagine some timescale based approaches where we might be able to
account for both the g$_1$ and uWUE variability, for example by making
assumptions over what time scale each quantity is fixed, and fitting
based on that (e.g. g$_1$ is fixed for a given season, and uWUE is
fixed for a given day). This approach might allow a tractable
solution, and also could help us filter model error and observational
noise from ``true'' plant physiological and climatic variability in
uWUE and g$_1$. The cost of all of this is a significant increase in
the analysis and content of the paper, as well as some increased
opacity to the methods. Comments are welcome.

Thank you again for the thoughtful review. We hope it stimulates
further discussion.


% \printbibliography
\bibliography{references}

\end{document}