\documentclass[12pt]{article}
\usepackage{amsmath}
\usepackage{amssymb}
\usepackage{bm}
\usepackage{graphicx}
\usepackage{epstopdf}
\usepackage{xcolor}
\usepackage{hyperref}
\DeclareGraphicsRule{.tif}{png}{.png}{`convert #1 `basename #1 .tif`.png}
\usepackage{color}
\pagestyle{plain}

\graphicspath{{../../../etc/plots/climate_et/test_plots/}}

\RequirePackage{natbib}
\bibliographystyle{agufull08}

% Bib aliases
\makeatletter
\def\@citex[#1]#2{\leavevmode
  \let\@citea\@empty
  \@cite{\@for\@citeb:=#2\do
    {\@citea\def\@citea{,\penalty\@m\ }%
\edef\magic##1{\let##1\expandafter\noexpand\csname bibalias@\@citeb\endcsname}%
\magic\tmp \ifx\tmp\relax\else \let\@citeb\tmp\fi
     \edef\@citeb{\expandafter\@firstofone\@citeb\@empty}%
     \if@filesw\immediate\write\@auxout{\string\citation{\@citeb}}\fi
     \@ifundefined{b@\@citeb}{\hbox{\reset@font\bfseries ?}%
       \G@refundefinedtrue
       \@latex@warning
         {Citation `\@citeb' on page \thepage \space undefined}}%
       {\@cite@ofmt{\csname b@\@citeb\endcsname}}}}{#1}}
\def\bibalias#1#2{\expandafter\def\csname bibalias@#1\endcsname{#2}}
\makeatother

\begin{document}

Thank you for a very thoughtful review; as in our response to RC1, we
feel very lucky to have received such quality reviews.

We read through your major and minor comments, and do not anticipate
issues reconciling your minor comments for the final response after
the discussion period. We will focus our response on your major
comments, with the goal of opening the door to further discussion.

Also please note our response to RC1, as some of the discussion there
is relevant to this response; particularly discussion on PFT-scale
analysis and variability of g$_1$ and uWUE
(\href{https://editor.copernicus.org/index.php/hess-2018-553-AC1.pdf?_mdl=msover_md&_jrl=13&_lcm=oc108lcm109w&_acm=get_comm_file&_ms=72556&c=153745&salt=17166479111051651323}{RC1
Response}).

\section{Major comment (i)}
\label{i}
One issue with analyses attempting to diagnose the ET response to VPD
with pure data (e.g. using binned boxplots, as you suggest), is that
ET varies strongly with both aerodynamic conductance and
radiation. Simply binning by VPD or by site includes too many
confounding factors to really diagnose what the attributable response
is to VPD. This issue is really what motivated our analysis and makes
it unique; by building an analytical framework based on confirmed
results in the literature (uWUE, Medlyn model, Penman-Monteith [PM]),
we are able to formally deduce what the actual ET response is, with
other environmental factors held fixed. This ``other environmental
factors held fixed'' component is, in our mind, nearly impossible to
deduce from pure data, because the environment is always changing and
direct analogues are either impossible to find or result in too small
of a sample size to form a meaningful conclusion. However, while we
cannot assess our formulation of the derivative with respect to VPD,
we can assess the actual model of ET, which includes all
approximations we introduced (specifically uWUE). As a part of our
analysis test suite we generated a figure of this comparison and
reproduce it here (Figure \ref{rmse}). Apologies this is not a
publication quality figure, but we wanted to get this response out
sooner rather than later to stimulate discussion, especially as we are
already behind because of the AGU Fall Meeting. This comparison is
fair in the sense that by using uWUE we introduce an extra free
parameter, so in the original PM model we introduce an extra free
parameter modifying g$_s$ and fit it also for each given PFT. The
introduction of the uWUE approximation does degrade the quality of the
ET model relative to the original PM model. However, we expect this,
as using uWUE is a simplification over using GPP directly, and may
break down in extreme, limiting cases like when VPD, ET or GPP go to
0. Additionally, we expect errors introduced by assuming that uWUE is
fixed within a given PFT. These errors may be reduced if we account
for spatiotemporal variation in uWUE, as discussed further in our
response to RC1.  As a purpose-built model for assessing the ET
response, we think the uWUE approach is justified. Essentially we are
saying that to the degree we trust arithmetic, uWUE, Penman-Monteith,
and the Medlyn stomatal conductance model, our evaluation of the
derivative is robust.

\begin{figure}
  \centering \includegraphics[width=0.75\textwidth]{rmse_bias.png}
  \caption{Mean bias (absolute value) and root mean squared error
    (RMSE) for three different ET estimates, relative to FLUXNET-2015
    observations: original (original Penman-Monteith formulation,
    e.g. Equation 1 in the manuscript, with Equation 4 providing
    g$_s$); uWUE (Equation 8 in the manuscript); iWUE (as in Equation
    8, but if we had used iWUE to remove GPP dependence rather than
    uWUE).}
  \label{rmse}
\end{figure}

However, the discussion so far assumes that the variables in our model
(Equation 8) do not change directly with VPD and we are free to
evaluate the derivative. You bring up a very good example that we did
not consider: a change in VPD will induce a change in net radiation
through surface temperature that PM does not consider. We will
definitely add language addressing this oversight. Ideally, we would
incorporate that effect directly in Equation 8. We briefly tried to do
this in time for this response, but difficulties deriving a diagnostic
equation for temperature sabotaged the effort, and we do not want to
delay the response further. We will continue to work on this for the
final response phase. In the meantime, we think we can make some
logic-based arguments in the interest of discussion as to what the
effect would be. In the case where the aerodynamic and physiological
ET response (which the manuscript does consider) increases with VPD,
we would expect a decrease in surface temperature. This would induce
an increase in net radiation, as OLR and ground heat flux
decrease. This increase in net radiation would induce a further
increase in ET response. Granted, the net radiation perturbation would
damp the negative temperature response to the
aerodynamic/physiological terms, but all of the signs of the change
would stay the same. The converse is true for a decrease in ET with
respect to VPD: surface temperature increases, OLR and ground heat
flux increase, net radiation decreases, ET decreases. In this way, we
believe the effects of VPD on radiation do not change the nature of
the sign of the response, but could amplify the magnitude. Comments on
this are encouraged, and we will work to include a more quantitative
and rigorous analysis in the final response. Sometimes it is easy to
trick oneself with these types of qualitative arguments, particularly
with complicated systems, so hopefully we have not presented faulty
logic in our rush to make this response public.

\section{Major comment (ii)}
\label{ii}

Thank you for having an open mind about the concave up ET-VPD curve
result and providing clear explanations of concerns about this result
based on literature. Given the apparent controversy, we will
definitely add more language connecting previous results, particularly
at the leaf scale, with the presented results. Here we will informally
elaborate in the interest of discussion.

One of the primary differences between ecosystem scale assessments of
ET (specifically using Penman Monteith) and experimental and modeling
results of leaf scale response to VPD is that by taking an energy
balance approach Penman Monteith consistently accounts for the effects
of the energy cost of evaporating water from a surface (e.g., changes
in e$_{sat}(T)$). To demonstrate this, compare Figure \ref{leaf} and
Figure \ref{PM} (``Original Penman Monteith'' Curve). In Figure
\ref{leaf} we apply a conceptual leaf scale approach to calculate
ecosystem ET without the energy balance consideration introduced by
PM. In this figure, the curve shows a concave downward shape, in
direct analogy to leaf scale T. Once the effects of evaporation's
thermodynamics are included in the energy balance, the curve's shape
is no longer concave downward (Figure \ref{PM}, ``Original Penman
Monteith'' curve). How these conceptual results translate to leaf
scale experiments depends on the experimental setup
\citep[e.g.,][]{Rawson1977, Turner1984, Mott2013}; however generally
chamber-based leaf scale experiments do not preserve the energy
balance relationships we expect for a surface in a natural environment
(sometimes intentionally by design). Additionally, we would like to
note that a concave up result is not necessarily inconsistent with the
statement ``transpiration at the leaf scale is mostly positive'' with
VPD. Especially for plants in well watered environments like crops,
the manuscript's ecosystem scale shape (Figure \ref{PM}) would not be
out of place on the figures in \cite{Rawson1977} and \cite{Mott2007}
for the range of VPD considered (Figure \ref{PM} compared to Fig. 2 in
\cite{Rawson1977}, also Figure 7 in \cite{Mott2007}). However, the
change in relative magnitudes are not similar, and this effect can be
traced partially to the effect of decreasing surface $e_{sat}(T)$ in
response to increasing evapotranspiration. While not directly
considered here (or in the manuscript) we also think it is important
to consider how plant physiologic response in the natural environment
could vary from plants grown in well-watered conditions in
experiments. It is not inconceivable that phenotypic variability and
adaptation in response to water limitation could result in plants with
different responses to those grown in a lab. We presented crops in our
figures because they are the most likely to well watered and analogous
with laboratory results; however for other PFTs the concavity of the
VPD-ET curves is more pronounced at higher VPD as compared to Figure
\ref{PM}.

\begin{figure}
  \centering \includegraphics[width=1.0\textwidth]{CRO_et_vpd_curve_leaf.png}
  \caption{The conceptual relationship between ET with VPD using a
    model analogous to leaf scale models and experiments, evaluated
    with the study's median ecosystem scale resistances for the crop
    PFT. This estimation of ET is by definition not physically
    representative, but the figure is intended as a conceptual
    description of how leaf-scale theory's ET-VPD relationship deviates
    from energy balance ecosystem theory (Fig. \ref{PM}). This figure is
    demonstrative, and not intended to be of publication quality.}
  \label{leaf}
\end{figure}

\begin{figure}
  \centering \includegraphics[width=1.0\textwidth]{CRO_et_vpd_curve_with_pm.png}
  \caption{The conceptual relationship between ET and VPD using the
    uWUE derived version of Penman Monteith (Equation 8 in the manuscript)
    and the original Penman Monteith formulation (Equation 1 in the
    manuscript), evaluated at the study's median conditions for the
    crop PFT. uWUE Penman Monteith includes the effects of VPD on
    photosynthesis. This figure is demonstrative, and not intended to be of
    publication quality.}
  \label{PM}
\end{figure}

The previous paragraph discussed the change in frame of reference
induced by using PM (and energy balance) to assess ecosystem scale ET
rather than directly applying leaf-scale logic from controlled
experiments. However, our uWUE-based PM framework additionally
accounts for changes in photosynthesis induced by VPD (Figure
\ref{PM}). Discussion in \cite{Damour2010} (``The issue of
co-regulation of g$_s$ and A$_{net}$'') provides some interesting
background both on the general weaknesses of current stomatal theory,
and specifically on the possible importance of resolving the effects
of water stress on $A_{net}$ in stomatal conductance. Given that the
current state of the art in physically-derived theory has not
converged on the proper way to account for changes in $A_{net}$ and
$g_s$ \citep{Damour2010}, we believe using the validated uWUE
empirical results \citep[e.g. Figure 3,][]{Zhou_2014} is a good
approach to develop an estimate of ET response to VPD accounting for
changes in photosynthesis. This method of using an empirical result in
the absence of a complete physical theory has been used often in
science (e.g. turbulence) to increase our knowledge about a given
system's behavior, and we think it is well motivated here. However, it
is important to add language on how introducing this empiricism could
alter interpretation of our results. In particular, while uWUE was
developed using leaf scale photosynthesis theory, it was validated
with observations of a fully coupled land-atmosphere system. It
captures a relatively consistent observed relationship between GPP,
ET, and VPD as they vary. There is a chance that some of this
relationship could be due to feedbacks between the land and the
atmosphere, even though the relationship was derived by leaf scale
theory without feedbacks (e.g., if the assumptions behind the uWUE
theory introduced ``real'' errors that were compensated for by
feedbacks or ecosystem-scale processes in the observations).  Were
this the case, then we introduced a conceptual impurity to the
statement ``we are examining the one way response of ET to VPD.''
But, given the theoretical motivations for uWUE we think the portion
of the relationship related to land-atmosphere feedbacks are likely
small, and regardless we argue that this approach is still an
improvement over ignoring the documented effect of water stress on net
photosynthesis.

Given the continued gaps in our ability to represent stomatal behavior
\citep{Damour2010}, particularly for ecosystems in natural
environments, we think it is hard to rule out our results. In the end,
it comes down to the degree to which we trust our three tools: the
Medlyn model, Penman-Monteith, and uWUE. To us these offered the most
compete analytical approach to the problem at present, and are
supported in the literature. However, it's important to emphasize that
we are not saying this is certainly correct, and to do so would be
irresponsible given the continued development of stomatal theory and
how it translates to the ecosystem scale. What is important is to
recognize that something as fundamental as the shape of the ET-VPD
curve is still very much an area of open research, and sensitive to
the representation of plant physiology (of which there are many
possibilities) in a given model or framework. We thank the reviewer
for having an open mind and giving us a chance to elaborate; others in
the community have rejected this result off-hand, which reinforces for
us the importance of demonstrating current uncertainty in VPD-ET
relationships.

\vspace{0.2in}
Just a quick final note:

We think that we miscommunicated with our (mis)use of ``plant physiology''
throughout the manuscript, and will alter our language in the final
response phase to be more precise. Many times when we refer to ``plant
physiology'' we mean any effect (direct or indirect) of an ecosystem
response, e.g. any behavior that would not be observed over a wet or
bare soil surface.

\section{Major comment (iii)}
\label{iii}
Yes, through our manipulation of PM we have replaced stomatal
conductance with two parameters (g$_1$ and uWUE). However, given that
any physically reasonable representation of stomatal conductance will
include more parameters (usually at least some sort of VPD-related
parameter [g$_1$], and a model for photosynthesis with more
parameters), we think we are actually reducing the number of unknowns
in our model for ET. We agree with you that in reality g$_1$ and
uWUE are spatially variable, and we elaborate in detail in our
response to Reviewer 1
(\href{https://editor.copernicus.org/index.php/hess-2018-553-AC1.pdf?_mdl=msover_md&_jrl=13&_lcm=oc108lcm109w&_acm=get_comm_file&_ms=72556&c=153745&salt=1660920846993335212}{RC1
  Response})
on why and how it might be better to frame our results more in terms
of this variability. Also, see our comments in Section \ref{i} on the
effect of VPD on R$_n$ and the suitability of a binning approach to
VPD.

We also agree with you that uWUE includes a representation of
ecosystem response. However, we would argue that this response
dependence of stomatal conductance is not introduced by uWUE, but is
actually an issue that exists inherently in any stomatal conductance
theory that includes a photosynthesis term: photosynthesis depends on
the physiological response and environmental conditions. Using uWUE is
our (approximate) solution to this problem, and it works because we
are able to algebraically manipulate PM to isolate ET after
introducing uWUE. We do recognize that using uWUE introduces some
uncertainty, but as previously discussed in this response we believe
it is a useful tool for representing the photosynthesis dependence of
stomatal conductance in an analytical framework, with the specific
goal of examining VPD sensitivity.

\section{Major comment (iv)}

We agree on both counts: use of ``leading order'' is misguided and
confusing in this context. Also, better separation of theory and
empirical results would help communication/understanding.
\vspace{0.3in}

Thank you again for such a thoughtful review. Including these insights
will really improve the manuscript in the final response phase. In our
response to Reviewer 1
(\href{https://editor.copernicus.org/index.php/hess-2018-553-AC1.pdf?_mdl=msover_md&_jrl=13&_lcm=oc108lcm109w&_acm=get_comm_file&_ms=72556&c=153745&salt=1660920846993335212}{RC1
Response})
we also laid out some alternatives to our approach for the final
review phase not explicitly discussed here. If you (or anyone else)
has any opinions on that please weigh in.

\bibliography{references}

\end{document}