\documentclass[12pt]{article}
\usepackage{amsmath}
\usepackage{amssymb}
\usepackage{bm}
\usepackage{graphicx}
\usepackage{epstopdf}
\usepackage{xcolor}
\DeclareGraphicsRule{.tif}{png}{.png}{`convert #1 `basename #1 .tif`.png}
\usepackage{color}
\pagestyle{plain}

\RequirePackage{natbib}
\bibliographystyle{agufull08}

% Bib aliases
\makeatletter
\def\@citex[#1]#2{\leavevmode
  \let\@citea\@empty
  \@cite{\@for\@citeb:=#2\do
    {\@citea\def\@citea{,\penalty\@m\ }%
\edef\magic##1{\let##1\expandafter\noexpand\csname bibalias@\@citeb\endcsname}%
\magic\tmp \ifx\tmp\relax\else \let\@citeb\tmp\fi
     \edef\@citeb{\expandafter\@firstofone\@citeb\@empty}%
     \if@filesw\immediate\write\@auxout{\string\citation{\@citeb}}\fi
     \@ifundefined{b@\@citeb}{\hbox{\reset@font\bfseries ?}%
       \G@refundefinedtrue
       \@latex@warning
         {Citation `\@citeb' on page \thepage \space undefined}}%
       {\@cite@ofmt{\csname b@\@citeb\endcsname}}}}{#1}}
\def\bibalias#1#2{\expandafter\def\csname bibalias@#1\endcsname{#2}}
\makeatother


\begin{document}

Thank you for a very thoughtful review; as in our response to RC1, we
feel very lucky to have received such quality reviews.

We read through your major and minor comments, and do not anticipate
issues reconciling your minor comments for the final response after
the discussion period. We will focus out response on your major
comments, with the goal of opening the door to further discussion.

Also please note our response to RC1, as some of the discussion there
is relelvant to this response; particularly discussion on PFT-scale
analysis \url{https://editor.copernicus.org/index.php/hess-2018-553-AC1.pdf?_mdl=msover_md&_jrl=13&_lcm=oc108lcm109w&_acm=get_comm_file&_ms=72556&c=153745&salt=17166479111051651323}.

\section{Major comment (i)}

One issue with analyses attempting to diagnose the ET response to VPD
with pure data (i.e. using binned boxplots, as you suggest), is that
ET varies strongly with both aerodynamic resistance and
radiation. Simply binning by VPD or by site includes too many
confounding factors to really diagnose what the attirbutable response
is to VPD. This issue is really what motivated our analysis and makes
it unique; by building an analytical framework based on confirmed
results in the literatre (uWUE, Medlyn model, Penman-Monteith), we are
able to formally deduce what the actual ET response is, with other
environmental factors held fixed. This ``other environmental factors
held fixed'' component is what is, in our mind, nearly impossible to
deduce from pure data, because the environment is always changing and
direct analogues are either impossible to find or result in too small
of a sample size to form a meaningful conclusion. However, while we
cannot assess our formulation of the derivative with respect to VPD,
we can assess the actual model of ET, which includes all
approximations we introudced (specifically uWUE). As a part of our
analysis test suite we generated a figure of this comparison and
reproduce it here (Figure \ref{rmse}). Apologies this is not a
publication quality figure, but we wanted to get this response out
sooner rather than later to stimulate discussion, especially as we are
already behind because of the AGU Fall Meeting. This comparison is
fair in the sense that by using uWUE we introduce an extra free
parameter, so in the original PM model we introduce an extra free
parameter modifying g$_s$ and fit it also for each given PFT. The
introduction of the uWUE approximation does degrade the quality of the
ET model relative to the original PM model. However, we expect this,
as using uWUE is a simplification over using GPP directly, and may
break down in extreme, limiting cases like VPD $\to$ 0. As
a purpose-built model for assessing the ET response, we think the uWUE
approach is justified. Essentially we are saying that to the degree we
trust arithmetic, uWUE, Penman-Monteith, and the Medlyn stomatal
conductance model, our evaluation of the derivative is robust.

\begin{figure}[h]
  \centering \includegraphics{./rmse_bias.png}
  \caption{Mean bias and root mean squared error (RMSE) for three
    different ET estimates, relative to FLUXNET-2015 observations:
    original (original Penman-Monteith formulation, e.g. Equation 1 in
    the manuscript, with Equation 4 providing g$_s$; uWUE (Equation 8
    in the manuscript); iWUE (as in Equation 8, but if we had used
    iWUE to remove GPP dependence rather than uWUE).}
  \label{rmse}
\end{figure}

However, the discussion so far assumes that the variables in our model
(Equation 8) do not change directly with VPD and we are free to
evaluate the derivative. You bring up a very good example that we do
not explictly consider: a change in VPD will induce a change in
surface temperature, which will alter the net radiation term. We will
definitely add language addressing this oversight. Ideally, we would
incorporate that effect directly in Equation 8. We briefly tried to do
this in time for this response, but difficulties deriving a diagnostic
equation for temperature sabatoged the effort, and we do not want to
delay the response further. We will continue to work on this for the
final response phase. In the meantime, I think we can make some
hand-wavy arguments in the interest of discussion as to what the
effect would be. In the case where the aerodynamic and physiological
ET response (which the manuscript does consider) increases with VPD,
we would expect a decrease in surface temperature. This would induce
an increase in net radiation, as OLR and ground heat flux
decrease. This increase in net radation would induce a further
increase in ET response. Granted, the net radiation perturbation would
damp the decreased temperature response to the
aerodynamic/physiological terms, but all of the signs of the change
would stay the same. The converse is true for a decrease in ET with
respect to VPD: surface temperature increases, OLR and ground heat
flux increase, net radiation decreases, ET decreases. In this way, we
believe the effects of VPD on radiation do not change the nature of
the sign of the response, but could amplify the magnitude. Comments on
this are incouraged, and we will work to include a more quantitative or
rigorous analysis in the final response. Sometimes it is easy to trick
onself with these types of qualitative arguments, particularly with
complicated systems, so hopefully we have not presented faulty logic
in our rush to make this response public.




\end{document}