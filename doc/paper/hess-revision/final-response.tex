\documentclass[12pt]{article}
% \usepackage{amsmath}
% \usepackage{amssymb}
% \usepackage{bm}
% \usepackage{graphicx}
% \usepackage{epstopdf}
% \usepackage{xcolor}
% \DeclareGraphicsRule{.tif}{png}{.png}{`convert #1 `basename #1 .tif`.png}
% \usepackage{color}
% \pagestyle{plain}

% \usepackage[
% 	backend=biber,
% 	url=false,
% 	isbn=false,
% 	doi=false,
%         eprint=false,
%         bibstyle=authoryear,
%         style=authoryear,
%         % style=numeric-comp,
% 	% uniquename=false,
% 	% uniquelist=false,
% 	% firstinits=true,
%  	natbib=true,
% 	sorting=none,
%         ]{biblatex}

% \addbibresource{./references.bib}

\RequirePackage{natbib}
\bibliographystyle{agufull08}

% Bib aliases
\makeatletter
\def\@citex[#1]#2{\leavevmode
  \let\@citea\@empty
  \@cite{\@for\@citeb:=#2\do
    {\@citea\def\@citea{,\penalty\@m\ }%
\edef\magic##1{\let##1\expandafter\noexpand\csname bibalias@\@citeb\endcsname}%
\magic\tmp \ifx\tmp\relax\else \let\@citeb\tmp\fi
     \edef\@citeb{\expandafter\@firstofone\@citeb\@empty}%
     \if@filesw\immediate\write\@auxout{\string\citation{\@citeb}}\fi
     \@ifundefined{b@\@citeb}{\hbox{\reset@font\bfseries ?}%
       \G@refundefinedtrue
       \@latex@warning
         {Citation `\@citeb' on page \thepage \space undefined}}%
       {\@cite@ofmt{\csname b@\@citeb\endcsname}}}}{#1}}
\def\bibalias#1#2{\expandafter\def\csname bibalias@#1\endcsname{#2}}
\makeatother

\title{Final Response: When does vapor pressure deficit drive or
  reduce evapotranspiration?}

\author{Adam Massmann, Pierre Gentine, and Changjie Lin}

\date{December 17th, 2018}

\begin{document}

Thank you very much for an excellent and very thoughtful
review. Addressing your comments will greatly improve the
manuscript. We agree with your comments, and think the manuscript
could use substantial re-framing and rewording to clarify how we are
answering our research question, and how the answer may vary with
climate and environmental factors.

Just a quick general comment before addressing your specific
comments. By including a PFT-focused analysis we did not fully
communicate the major goal and scope of our project: we are trying to
characterize the response of ET to VPD, with all other environment
variables held fixed. To accomplish this, we need to formulate an
explicit function of ET in terms of environmental variables and
parameters, where any parameters can be approximated as constant with
regards to some [arbitrary] VPD perturbation scenario. We will
elaborate more below on why we used PFT-focused scenarios for much of
our analysis. However, the goal of our manuscript was much more simple
and fundamental: at a given place or time, if you introduce a
perturbation to VPD, what is the immediate ET response (e.g. positive
or negative)? Answering this question does not require the much
stronger assertion that any parameters must be invariant within a
given PFT.

\section{Reviewer 1}

Thank you for a very thoughtful review. You identified some critical
flaws in our methods and presentation, and addressing these flaws
have, in our opinion, greatly improved the strength and clariy of the
manuscript.

\subsection{Compounding uncertainties}

As stated in our response during the open discussion period, we agree
with you that there are problems prescribing any parameters as fixed
within a given PFT, as we would expect relatively large variability
within a PFT. While this type of analysis was motivated by examples in
previous literature, we think it is misleading in this case to present
results where these parameters are suggested to be fixed. These issues
were exacerbated by the manner in which we fit our parameters: it was
unclear given our methods how model and observational uncertainty,
which can be considerable, are partitioned into our PFT-wide estimates
of g1 and uWUE. Also, as you point out, there is already previous
literature on the values of g1 and uWUE, so in many ways by fitting
out own values we are repeating previous work, but with defects given
our handling of observational and model error. Also, much of our
observational work, in addition to its flaws, decreased the clarity of
our manuscript and distracted from the focus, which a flexible
derivation, based completely on previous literature, of ET explicitly
as a function of enviornmental variables and plant parameters.

These realizations led us to make significant alterations to our
manuscript. We no longer use observations or fit parameters
ourselves. Instead we frame our theoretical derivation in terms of
parameters established in the literature. Using this literature, we
can then make statements about the general liklihood of a positive or
negative ET response to increasing VPD in terms of climate, plant
type, and photosynthetic pathway. This seemed a more natural
delineation of scope: our paper presents a derivation tying plant
parameters to an ET response to VPD, while previous literature
establishes relationships between plant parameters and plant
type/climate.

\subsection{Attribution to physiological responses}

In our open discussion response we detailed our agreement with you that g1 and
uWUE should vary within a given PFT. Removing this PFT-focused
analysis should prevent any misinterpretation that we are making
claims that these paremeters should be fixed within a given PFT. For
convenience, we also reproduce a particlarly relevant passage from our
orignal response:

``What we need to communicate better is that a given ecosystem's state
at a place or time controls its response to a VPD perturbation. We are
making an approximation that we can parameterize the effect of the
ecosystem's state on VPD response with g$_1$ and uWUE. These quantities
can also vary due to soil moisture condition; the approximation we
need to answer our research question is that they are fixed
with respect to a VPD perturbation. In this sense, we do not view the
two statements: ``plants that are evolved to bred to prioritize
primary production over water conservation (e.g., crops) exhibit a
higher likelihood of atmospheric demand-driven response'' and
``ecosystem types are responding in this way because they have, on the
whole, been subjected to less soil water limitation (due to the
non-negligible effects of irrigation?)'' as mutually exclusive. In
fact, we view them as consistent with a view that crops and their
physiology (parameterized by g$_1$ and uWUE) exist at a given time and
place because they have not been subjected to soil water limitation,
and they have a given response to VPD because of their physiology
(which is a direct effect of the environment). In this way, climate
and land surface state are causes of the VPD response both directly
and through their effect on plant physiology (parameterized by g$_1$
and uWUE).''

In addition we added language to the manuscript that our results
should be used by a reader/scientist under the assumption that plant
parameters are fixed with respect to the VPD perturbation
considered. Previous literature \citep{Zhou_2015, Lin_2015} suggests
that these parameters might also be approximately constant across a
relatively broader range of plant types and environmental conditions,
but our manucript avoids claims about this costancy other than the
importance of conceptually considering these parameters to be constant
when applying our theory to a problem of interest.

The issue of soil moisture dependence is of particular importance,
however. If our plant parameters are more stationary with respect to
soil moisture, then our theory can be applied to a broader range of
VPD scenarios, including observed compound events between high VPD and
low soil moisture \citep{Zhou_2019}. To help the reader assess the
soil moisture dependence of uWUE (and partially by extension
$\lambda$), we provided supplementaty material showing the
distribution of uWUE with SWC for each of 66 FLUXNET sites. The
functional relationship between uWUE and SWC varies, with a mix of
sites showing strong and weak SWC-dependence. For all sites the ratio
of signal to noise is very low, an unfortunate consequence of taking a
ratio of two highly uncertain eddy-covariance derived fluxes. In
general we find this analysis inconclusive. Our paper does not rely on
assumptions about uWUE's functional relationship to soil moisture so
we do not include these analysis in the body of the text, but these
types of assumptions can make our theory more useful to the
practitioner so we provide them for their interpretation and
motivation for future research.


% \printbibliography
\bibliography{references}

\end{document}