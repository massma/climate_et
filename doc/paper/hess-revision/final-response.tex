\documentclass[12pt]{article}
\usepackage{amsmath}
\usepackage{amssymb}
\usepackage{bm}
\usepackage{graphicx}
\usepackage{epstopdf}
\usepackage{xcolor}
\usepackage{hyperref}
\DeclareGraphicsRule{.tif}{png}{.png}{`convert #1 `basename #1 .tif`.png}
\usepackage{color}
\pagestyle{plain}
\usepackage{longtable}
\usepackage{textcomp}
\graphicspath{{../../../etc/plots/climate_et/test_plots/}}

\RequirePackage{natbib}
\bibliographystyle{agufull08}

% Bib aliases
\makeatletter
\def\@citex[#1]#2{\leavevmode
  \let\@citea\@empty
  \@cite{\@for\@citeb:=#2\do
    {\@citea\def\@citea{,\penalty\@m\ }%
\edef\magic##1{\let##1\expandafter\noexpand\csname bibalias@\@citeb\endcsname}%
\magic\tmp \ifx\tmp\relax\else \let\@citeb\tmp\fi
     \edef\@citeb{\expandafter\@firstofone\@citeb\@empty}%
     \if@filesw\immediate\write\@auxout{\string\citation{\@citeb}}\fi
     \@ifundefined{b@\@citeb}{\hbox{\reset@font\bfseries ?}%
       \G@refundefinedtrue
       \@latex@warning
         {Citation `\@citeb' on page \thepage \space undefined}}%
       {\@cite@ofmt{\csname b@\@citeb\endcsname}}}}{#1}}
\def\bibalias#1#2{\expandafter\def\csname bibalias@#1\endcsname{#2}}
\makeatother

\title{Final Response: When does vapor pressure deficit drive or
  reduce evapotranspiration?}

\author{Adam Massmann, Pierre Gentine, and Changjie Lin}

\date{February 28th, 2019}

\begin{document}

\section{To the editor}

Thank you for facilitating a rewarding review process. In addressing
the reviewers' comments we made significant alterations to the
manuscript, bordering on a complete rewrite. We believe this was
warranted to improve the clarity and address some methodological flaws
identified by the reviewers.

Relevant to your comments to us, we added a lot of language clarifying
that with a partial derivative, environmental conditions and parameters
should be considered fixed with respect to a (conceptual) change in
VPD. Your comments suggest that such a conceptual scenario does not
have scientific utility unless it is explicitly observed in the
observations (e.g. we see observe only VPD changes and nothing
else). On this point I think we disagree, as we believe such idealized
and conceptual approaches still have tremendous utility disentangling
complicated observations in which many variables are co-varying
together, and represent a path towards deeper understanding for how
extremely complicated systems behave. Understanding how one quantity
induces a change in another, with all other quantities held fixed, is
for us a crucial first step towards quantifying environmental drivers,
causal relationships, and feedback pathways. A serious issue with the
original version of the manuscript was that we believe we did a poor
job communicating both the conceptual consequences and utility of a
partial derivative approach. We hope that the language we added in
this version of the manuscript more strongly communicates our approach
and how it is intended to be used.

We also hope that the manuscript is not misleading. Part (most?)  of
the problem may have been afore mentioned deficiencies in our
communication in the original version. We took a simpler, idealized
approach with very explicit assumptions in large part because we value
conceptual clarity and ease of interpretation. With improved language
and organization in the updated manuscript version hopefully the
reader feels the same way. The real benefit of this conceptual clarity
is that it enables the reader to make a decision about whether the
tool or result is useful or relevant to them. We think this has
tremendous value, and find that many published papers have limited
utility explicitly because they do not provide this value (it often
feels impossible to identify all the assumptions some results rely
on). Our reasons for mentioning ESM/GCMs was not to provide evidence
that it is scientifically justified to use a constant slope parameter
or $\lambda$, but to 1) draw attention of an application of our
framework for interpreting GCMs/ESMs, and more importantly 2) draw
attention to a difference in approaches: while we have made an
assumption very explicit (increasing the interpretability of our
results), many published results (based on ESMs) are inherently opaque
because of the same assumption made implicitly and buried deep in
documentation (and even more complicated by multiple such assumptions
interacting non-linearly). Our worry is that there exists a
possibility that the more explicit one makes assumptions, the lower
the likelihood that results will be accepted and disseminated. We hope
the relationship is the opposite. To us, a result we may not agree
with, but for which we understand the assumptions behind it, is far
more valuable than a result we agree with but for which we cannot
decipher the underlying assumptions. However, hopefully improved
communication and organization in the updated version of the
manuscript mitigates a lot of the the concerns about misleading
readers.

Thank you again for your comments and facilitating such a high quality
review process. We look forward to further comments and feedback, and
greatly appreciate the significant time everyone has spent improving
our manuscript.


\section{To all reviewers}

Thank you for all for very thoughtful reviews. You identified some
critical flaws in our methods and presentation, as well as key points
to elaborate upon. Addressing these flaws have, in our opinion,
greatly improved the strength and clarity of the manuscript.

\section{Reviewer 1}

\subsection{Compounding uncertainties}
\label{pft}
As stated in
\href{https://editor.copernicus.org/index.php/hess-2018-553-AC1.pdf}{our
  response} during the open discussion period, we agree
with you that there are problems prescribing any parameters as fixed
within a given PFT, as we would expect relatively large variability
within a PFT. While this type of analysis was motivated by examples in
previous literature, we think it is misleading in this case to present
results where these parameters are suggested to be fixed. These issues
were exacerbated by the manner in which we fit our parameters: it was
unclear given our methods how model and observational uncertainty,
which can be considerable, are partitioned into our PFT-wide estimates
of g1 and uWUE. Also, as you point out, there is already previous
literature on the values of g1 and uWUE, so in many ways by fitting
out own values we are repeating previous work, but with defects given
our handling of observational and model error. Also, much of our
observational work, in addition to its flaws, decreased the clarity of
our manuscript and distracted from the focus, which a flexible
derivation, based completely on previous literature, of ET explicitly
as a function of environmental variables and plant parameters.

These realizations led us to make significant alterations to our
manuscript. We no longer use observations or fit parameters
ourselves. Instead we frame our theoretical derivation in terms of
parameters established in the literature. Using this literature, we
can then make statements about the general likelihood of a positive or
negative ET response to increasing VPD in terms of climate, plant
type, and photosynthetic pathway. This seemed a more natural
delineation of scope: our paper presents a derivation tying plant
parameters to an ET response to VPD, while previous literature
establishes relationships between plant parameters and plant
type/climate.

\subsection{Attribution to physiological responses}

In our open discussion
\href{https://editor.copernicus.org/index.php/hess-2018-553-AC1.pdf}{response}
we detailed our agreement with you that g1 and uWUE should vary within
a given PFT. Removing this PFT-focused analysis should prevent any
misinterpretation that we are making claims that these parameters
should be fixed within a given PFT. For convenience, we also reproduce
a particularly relevant passage from our original response:

\medskip

``What we need to communicate better is that a given ecosystem's state
at a place or time controls its response to a VPD perturbation. We are
making an approximation that we can parameterize the effect of the
ecosystem's state on VPD response with g$_1$ and uWUE. These quantities
can also vary due to soil moisture condition; the approximation we
need to answer our research question is that they are fixed
with respect to a VPD perturbation. In this sense, we do not view the
two statements: ``plants that are evolved to bred to prioritize
primary production over water conservation (e.g., crops) exhibit a
higher likelihood of atmospheric demand-driven response'' and
``ecosystem types are responding in this way because they have, on the
whole, been subjected to less soil water limitation (due to the
non-negligible effects of irrigation?)'' as mutually exclusive. In
fact, we view them as consistent with a view that crops and their
physiology (parameterized by g$_1$ and uWUE) exist at a given time and
place because they have not been subjected to soil water limitation,
and they have a given response to VPD because of their physiology
(which is a direct effect of the environment). In this way, climate
and land surface state are causes of the VPD response both directly
and through their effect on plant physiology (parameterized by g$_1$
and uWUE).''

\medskip

In addition we added language to the manuscript stating that our
results should be used by a reader/scientist under the assumption that
plant parameters are fixed with respect to the VPD perturbation
considered.

The issue of soil moisture dependence is of particular importance,
however. If our plant parameters are more stationary with respect to
soil moisture, then our theory can be applied to a broader range of
VPD scenarios, including observed compound events between high VPD and
low soil moisture \citep{Zhou_2019}. To help the reader assess the
soil moisture dependence of uWUE (and partially by extension
$\lambda$), we provide figures in supplementary material showing the
distribution of uWUE with SWC for each of 66 FLUXNET sites from the
FLUXNET-2015 database. The functional relationship between uWUE and
SWC varies, with a mix of sites showing strong and weak
SWC-dependence. For all sites the ratio of signal to noise is very
low, an unfortunate consequence of taking a ratio of two highly
uncertain eddy-covariance derived fluxes. In general we find this
analysis inconclusive. The paper does not rely on assumptions about
uWUE's functional relationship with soil moisture so we do not include
the figures in the body of the manuscript. But given that constant
uWUE and g$_1$ assumptions can make our theory more useful to the
reader we provide the figures for their interpretation, and motivation
for future research.


\section{Reviewer 2}
\subsection{(i)  ET Response to VPD diagnosed from data}
\label{(i)}
Please see our
\href{https://www.hydrol-earth-syst-sci-discuss.net/hess-2018-553/hess-2018-553-AC2-supplement.pdf}{previous
  response} for a full discussion of the barriers to assessing the ET
response to VPD using pure data. For convenience, we inclined an excerpt
here:

\medskip

``One issue with analyses attempting to diagnose the ET response to
VPD with pure data (e.g. using binned boxplots, as you suggest), is
that ET varies strongly with both aerodynamic conductance and
radiation. Simply binning by VPD or by site includes too many
confounding factors to really diagnose what the attributable response
is to VPD. This issue is really what motivated our analysis and makes
it unique; by building an analytical framework based on confirmed
results in the literature (uWUE, Medlyn model, Penman-Monteith [PM]),
we are able to formally deduce what the actual ET response is, with
other environmental factors held fixed. This ``other environmental
factors held fixed'' component is, in our mind, nearly impossible to
deduce from pure data, because the environment is always changing and
direct analogues are either impossible to find or result in too small
of a sample size to form a meaningful conclusion... Essentially we are
saying that to the degree we trust arithmetic, uWUE, Penman-Monteith,
and the Medlyn stomatal conductance model, our evaluation of the
derivative is robust.''

\medskip

You also mention possible net radiation dependence on VPD, which we
address in the minor comments.

\subsection{(ii) Leaf scale discussion}
\label{leaf}
As in our
\href{https://www.hydrol-earth-syst-sci-discuss.net/hess-2018-553/hess-2018-553-AC2-supplement.pdf}{previous
  response}, we agree that we need more discussion connecting and
contrasting leaf level results with our ecosystems scale approach. We
have summarized the discussion from our reviewer response and added it
to the introduction.

\subsection{(iii) Additional uWUE parameter}

We agree that are PFT-based formulation was too easy to be
misinterpreted as an assertion that g1 and uWUE are constant and never
vary in time or space. As described in Section \ref{pft}, we have
removed the observational and PFT-focused component of the analysis,
and instead focus exclusively on using previously published constants
(and their range) to interpret results. We would also like to
reiterate that we do not view our use of uWUE as an introduction of an
extra parameter. Our use of uWUE replaces a photosynthesis model and
stomatal conductance reduction factor, which will generally have many
more parameters. Using uWUE is instead a simplifying approximation
that relies on assumptions (based on optimal theory), but one that
certainly reduces the total number of parameters used in a more
traditional estimate of ET.

\subsection{(iv) Bad language}

Yes, we should not have used ``leading-order'' language. We removed
that confusing and inappropriate language. Also, by removing the
empirical results and focusing exclusively theory framed by observational
results from the literature, we have clarified the manuscript along
lines you suggest (e.g. better separation of theory vs empirical
data).

\subsection{Minor Comments}

\begin{itemize}
\item P 1 L 17: We removed ``stress'' language.
\item P 1 L 22-24: Discussion of PET was removed to simplify
  manuscript (it seemed less relevant). Otherwise the Roderick
  citation is a good one, thanks.
\item P 2 L 17: Discussion of precipitation removed to simplify
  manuscript.
\item P 2 L 33-35: You're right, and this was speculation; removed
  this language and replaced with: ``However these researchers did not
  explore explicitly the sensitivity of ET to VPD, including VPD's
  effect on stomatal conductance and plant function.''
\item P 3 L 2: Added citation.
\item P 3 L 9: Replaced novel with useful.
\item P 5 L 4: Yes it is related to water use efficiency, but we think
  that could add more confusion given the ``water use efficiency''
  language in uWUE; by expressing it in terms of $\lambda$ hopefully
  it is clear to the reader that g$_1$ is related to water use
  efficiency.
\item P 6 L 4: Yes, that equation was mislabeled. Should be fixed now.
\item P 7 L 14: We no longer fit our own values, so this is no
  longer relevant.
\item P 7 Eq. (10): Yes, a change in ET/VPD would induce indirect
  effects like changes in temperature and (be extension) net
  radiation. We added discussion guiding the reader on how to avoid
  misuse partial derivatives, because it seems like readers may not
  have sufficient understanding, and included explicit discussion of
  net radiation as an example: ``Care must also be taken with possible
  indirect effects associated with a change in VPD: for example, a
  change in ET induced by a change in VPD can cause a change in
  surface temperature, which would drive a change in net
  radiation. These types of indirect effects and feedbacks are not
  considered in Eq. (6): temperature (a variable) is
  mathematically fixed.''
\item P 10 LL 2 and LL 5: Yes, ``physiological controls'' can be
  misinterpreted and doesn't tell the full story. This language is
  removed.
\item P 10 Eq. (12): Yes there should have been a $\sigma$, but
  $\sigma$ is now removed from the manuscript.
\item P 10. L 11: Yes, both g$_1$ and uWUE can vary with soil moisture
  as both are dependent on $\lambda$. This section is however removed
  from the updated manuscript, but we also added language reflecting
  this ``These parameters will however vary with plant species
  and characteristics \citep[e.g. wood density, ][]{Lin_2015}, as well
  as environmental conditions including soil water content and
  temperature \citep{Lin_2015, Manzoni2013}.''
\item P 10 Equation (13): $\sigma$ should not have been removed, but
  it is now no longer in the manuscript.
\item P 10 L 29-30 Yes, we added more language on both parameters, and
  how they are related through $\lambda$.
\item P 11 L 14-17 : this discussion removed from new version of
  paper.
\item P 13 L 7-9: this discussion removed from new version of paper.
\item Figure 4: removed from paper.
\item P 16 L 16: Yes this was somewhat circular and limited: it only
  allowed us to test if/how uWUE, given a prescribed model form
  and the data, would systematically vary with VPD within a PFT. There
  were definitely some issues with our PFT-focused methods and how we
  handled uncertainty, which led us to remove this portion of the
  analysis (see Section \ref{pft}).
\item P 16, L 15: Agreed, and thankfully this whole analysis is now
  removed from the paper.
\item P 18 L 32 / P 21 L 1-3: These sections have been
  removed. However, we do cannot completely follow the feedback logic
  you are using; can you elaborate? We also added some plots of the
  relationship between soil moisture and uWUE in our supplementary
  material to aid the reader, but view those as inconclusive and an
  are of active research as well. Finally, in general the updated  manuscript is
  much better about acknowledging that uWUE and g$_1$ can vary through
  time, but should be fixed with respect to a VPD perturbation.
\item P 23 L 6: This section removed.
\item P 23 LL 16-21 : W added language about the leaf-scale derived
  shape of the ET-VPD curve in the introduction. We prefer the section
  on the ET-VPD shape in the discussion, as we view it as a discussion
  of the implication of our results (which is the derivation itself),
  and how it is sensitive to stomatal model choice.
\item P 24 L 24: Added clarifying language about the exponent.
  ``In general, increasing the exponent of VPD dependence for
  either $uWUE*$ or $g_*$ increases the likelihood ``
\item P 26 L 2-4: Please see discussion in Section \ref{(i)}.
\item P 27 Appendix A: removed use of FLUXNET data in this manuscript.
\end{itemize}

\section{Reviewer 3}

\subsection{Consequences of $\lambda$ in g$_1$ and uWUE}

\subsubsection{Soil moisture and $\lambda$}

Yes, $\lambda$, and by extension uWUE and g$_1$ can vary with
environmental conditions, including soil moisture. In the worst case,
the consequences of this dependency for scientists applying our theory
is that they must conceptually consider the change in ET with respect
to VPD only for scenarios in which soil moisture is fixed. We believe
this more restricted scenario still has utility for disentangling land
atmosphere feedbacks, and understanding observations and more
sophisticated models. However, \cite{Zhou_2015}'s results do faintly
suggest that uWUE (and by extension $\lambda$) may be somewhat
constant with across a range of conditions (including implicitly soil
moisture). The degree to which it is constant influences the degree to
which a reader can relax their interpretation to one in which soil
moisture is not fixed with respect to VPD. We attempted to shed some
more light on this issue by plotting the distribution of uWUE with SWC
for 66 FLUXNET sites, but the signal to noise ratio was too low and
when there was a signal of a relationship, it was inconsistent across
sites. We did include this in supplemental material to aid the reader
and draw attention to uWUE (and $\lambda$) - SWC relationships at the
ecosystem scale as an important area for future research.

\subsubsection{Interpretation issues with the use of the uncertainty
  parameter}

We agree that forcing all uncertainty (and real variability) into a
single parameter made successful interpretation our results extremely
difficult. The general issues with this formulation led us to remove
the observational component of the paper, to focus on the derivation
using plant parameters previously established in observational studies
in the literature \citep{Zhou_2015, Medlyn_2017}. This simplification
and focus also [hopefully] clarifies the manuscript and the
presentation of results.

\subsection{General comments}

\begin{itemize}
  \item \textbf{Symbol definition}: thank you for pointing out these mistakes
    and/or potential sources of confusion. We have gone through the
    manuscript and attempted to clarify/fix these issues.
  \item \textbf{Dependence on $\lambda$}: we did present both uWUE and g$_1$ in
    terms of $\lambda$ in our original manuscript, but we added
    language drawing the readers attention to this, and elaborating
    its importance for interpreting how they co-vary.
  \item \textbf{Presentation of leaf-level stomatal conductance}: we
    thought this might help the reader and emphasize that the theory
    originated at the leaf scale. However we agree with you that it
    seems unnecessary given we do not use it, and makes the manuscript
    less clear. We removed it.
  \item \textbf{g$_a$ calculation}: in the newer idealized version of
    the manuscript we do not calculate g$_a$, but represent a range of
    reasonable values suggested by data and the literature
    \citep[e.g.,][]{Raupach_1995}.
  \item \textbf{Fitting of parameters with FLUXNET data}: as
    mentioned, we removed this part of the analysis.
\end{itemize}

\subsection{Presentation of results}

We agree that the results are more clear when all figures are in terms
of $\frac{\partial ET}{\partial VPD}$. All figures are now presented
in terms of $\frac{\partial ET}{\partial VPD}$, including those where
only parameters in the sign term are varied.

We also agree that our original manuscript was confusing and
disjointed. Hopefully the significant changes to the structure of the
manuscript have clarified, streamlined, and simplified the
results. Regarding Section 3.7, we moved it into a new ``discussion''
section. While it could be placed in the methods, we view it as more
of a discussion on the consequences of using a given stomatal
conductance theory, and think it is hopefully easier for the reader to
digest after seeing the concave up nature of our results.

Yes, we did misuse ``leading order'' language, which we removed. Also many
of the subjective interpretations of quantitative results have been
removed with the removal of the observational component.

\subsection{Minor comments}

\begin{itemize}
  \item \textbf{Introduction}: We added more discussion of previous
    work on transpiration response to VPD, as elaborated on in Section
    \ref{leaf}. We also introduce \cite{Novick_2016}'s results, and
    added language recognizing there are more details to plant strategy
    and response to VPD than our simplified logic in the introduction.
  \item \textbf{P 9 results equations}: These were moved to methods.
  \item \textbf{P 13, L32, Temperature effect}: Yes, the direct effect is
    included in our idealized analysis (although because of CC the
    effect through $\Delta$ should be larger than the ``direct''
    effect).
  \item \textbf{P 15 L 15 ``between''}: removed.
\end{itemize}

% \printbibliography
\bibliography{final-references}

\end{document}