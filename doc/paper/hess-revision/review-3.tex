\documentclass[12pt]{article}
\usepackage{amsmath}
\usepackage{amssymb}
\usepackage{bm}
\usepackage{graphicx}
\usepackage{epstopdf}
\usepackage{xcolor}
\usepackage{hyperref}
\DeclareGraphicsRule{.tif}{png}{.png}{`convert #1 `basename #1 .tif`.png}
\usepackage{color}
\pagestyle{plain}

\graphicspath{{../../../etc/plots/climate_et/test_plots/}}

\RequirePackage{natbib}
\bibliographystyle{agufull08}

\begin{document}

Thank you for a very thoughtful review. This third review adds further
to our feeling that we are extremely lucky to have received such
quality reviews.

We agree with your comments and will respond with some general views
to open the door for further discussion in this discussion phase. We
look forward to reconciling all reviews with a
formal response in the final response phase. Also of relevance to this
review is our \href{https://editor.copernicus.org/index.php/hess-2018-553-AC1.pdf?_mdl=msover_md&_jrl=13&_lcm=oc108lcm109w&_acm=get_comm_file&_ms=72556&c=153745&salt=17166479111051651323}{
  response to RC1}, which we will refer to as needed to avoid
repeating content.

\section{Soil moisture, uWUE and g$_1$}
Environmental conditions, particularly soil moisture, will effect both
uWUE and g$_1$ as you point out. Not accounting for this variability
by using a PFT-focused analysis is problematic. We discuss this in
\href{https://editor.copernicus.org/index.php/hess-2018-553-AC1.pdf?_mdl=msover_md&_jrl=13&_lcm=oc108lcm109w&_acm=get_comm_file&_ms=72556&c=153745&salt=17166479111051651323}{our
  response to RC1}, specifically Sections 2 and 3, and think a strong
argument could be made that we should have framed our analysis in
terms of more general uWUE and g$_1$ variability instead of focusing
primarily on PFT variability. If we let uWUE and g$_1$ vary in space
and time, then we include a measure of soil moisture through its
effect on uWUE and g$_1$. In this case, interpretation of our results
relies on a less restrictive approximation that soil moisture
conditions remain relatively fixed with respect to the VPD
perturbation, which is consistent with the general interpretation of
our results (e.g. we evaluate the ET response to VPD with
other environmental conditions held fixed).

There are also issues with only including uncertainty in uWUE. We
discuss reasons for this approach in our response to
\href{https://editor.copernicus.org/index.php/hess-2018-553-AC1.pdf?_mdl=msover_md&_jrl=13&_lcm=oc108lcm109w&_acm=get_comm_file&_ms=72556&c=153745&salt=17166479111051651323}{RC1}
(specifically the final paragraph), and we think there may be some
tractable approaches to representing uncertainty and variability in
both uWUE and g$_1$ for the final manuscript.

\section{Methods/Introduction}
A more thorough discussion in the introduction about what is known
about the transpiration response to VPD is definitely needed. This is
also discussed peripherally in our response to
\href{https://editor.copernicus.org/index.php/hess-2018-553-AC2.pdf?_mdl=msover_md&_jrl=13&_lcm=oc108lcm109w&_acm=get_comm_file&_ms=72556&c=154058&salt=19427645101542423504}{RC2}. The clarifications you
suggest for the methods section will also improve the manuscript.

\section{Results}
Regarding the presentation of the results, we agree that in its current
form the manuscript can be disorienting. Just discussing the sign of
the derivative, and then discussing the magnitude (while including
what the current manuscript refers to as the ``sign term'') seems to
be a much cleaner approach, possibly also differentiating between
environmental controls versus plant/canopy controls.

In our response to RC1 we discussed a few revision approaches
regarding the results, some of them including significant changes,
which should help both with acknowledging g$_1$ and uWUE variability
and streamlining the results. One example (``Option 2'') is to remove
the PFT focus from the first portion of the results (Section 3.1-3.2;
this could also go in Methods as you suggest), and instead focus
generally on how changes in the environment (e.g. wind speed,
temperature) and land surface terms (e.g. g$_1$, uWUE, canopy height)
influence the ET response. By using a more sophisticated representation
of uncertainty we could then present the distribution of
observation-informed ET responses as a function of the model terms
(g$_a$, uWUE, g$_1$, etc.), and finally tie all this to either PFTs or
specific sites with maps of statistics from the distribution of model
parameters and ET response. While this could add a lot of analysis and
content to the paper, we think in the end it could streamline the
results and their interpretation. Basically we would start with the
idealized ET response as a function of parameters (no uncertainty), to
the observed ET response as a function of parameters (with
uncertainty), to how both the parameters and ET response (with
uncertainty) map to specific sites, years and/or PFTs. Thoughts on
this are welcome.

We also acknowledge that we need to be more precise about some of our
language (e.g. ``leading order behavior'', ``bit more variability''),
and thank you for the other minor comments as well; fixing these will
improve the manuscript.

\end{document}