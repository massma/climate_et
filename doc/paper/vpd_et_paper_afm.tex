% 11/23/2015
\documentclass[draft,linenumbers]{afmjournal}
% \draftfalse
\usepackage{makecell} \usepackage{multirow} %AM for tables
\usepackage{hyperref}
\usepackage[title]{appendix}
\usepackage{booktabs} \drafttrue
\usepackage{longtable}


\journalname{Agricultural and Forest Meteorology: When does VPD drive
  or reduce ET?}

\newcommand{\appropto}{\mathrel{\vcenter{
      \offinterlineskip\halign{\hfil$##$\cr
        \propto\cr\noalign{\kern2pt}\sim\cr\noalign{\kern-2pt}}}}}

\begin{document}

\title{When does vapor pressure deficit drive or reduce
  evapotranspiration?}


\authors{Adam Massmann\affil{a}, Pierre Gentine\affil{a}, Changjie
  Lin\affil{a,b}}

\affiliation{a}{Department of Earth and Environmental Engineering,
  Columbia University, New York, NY 10027} \affiliation{b}{State Key
  Laboratory of Hydroscience and Engineering, Department of Hydraulic
  Engineering, Tsinghua University, Beijing, CN 100084}

\correspondingauthor{Adam Massmann}{206-919-1364;
  akm2203@columbia.edu}

\begin{abstract}
Increasing vapor pressure deficit (VPD) increases atmospheric demand
for water, and vapor pressure deficit is expected to rise with
increasing greenhouse gases. While increased evapotranspiration (ET)
in response to increased atmospheric demand seems intuitive, plants
are capable of reducing ET in response to increased VPD by closing
their stomata, in an effort to conserve water. Here we examine which
effect dominates response to increasing VPD: atmospheric demand and
increases in ET, or plant physiological response (stomata closure) and
decreases in ET. We use Penman-Monteith, combined with semi-empirical
optimal stomatal regulation theory and underlying water use
efficiency, to develop a theoretical framework for understanding how
ET responds to increases in VPD. The theoretical ET response to VPD
over a range of reasonable environmental conditions and plant
characteristics varies from a strong decrease in ET in response to
increasing VPD (water conservative) to a strong increase in ET in
response to VPD (water intensive), highlighting the diversity of plant
water regulation strategies.

The ET response varies due to: 1) climate, with tropical and
temperate climates more likely to exhibit a positive ET response than
boreal and arctic climates; 2) photosynthesis strategy, with C3 plants
more likely to exhibit a positive ET response than C4 plants; and 3)
due to plant type, with crops more likely to exhibit a positive ET
response, and shrubs and gymniosperm trees more likely to exhibit a
negative ET response. These results, derived from previous literature
connecting plant parameters to plant and climate characteristics,
highlight the utility of our simplified framework for understanding
complex land atmosphere systems in terms of idealized scenarios.

\end{abstract}

\begin{keypoints}
\item ecohydrology, evapotranspiration, aridity, vapor pressure
  deficit, plant physiology, stomatal response
\end{keypoints}

\section{Methods}
\label{methods}
The Penman-Monteith equation \cite<hereafter PM,>[]{Penman_1948,
  Monteith_1965} estimates ET as a function of observable atmospheric
variables and surface conductances:
% \begin{linenomath*}
  \begin{equation}
    \label{orig_pen}
    ET = \frac{\Delta R_{net} + g_a \rho_a c_p VPD}{\Delta + \gamma(1 + \frac{g_a}{g_s})},
  \end{equation}
% \end{linenomath*}
  where $\Delta$ is the change in saturation vapor pressure with
  temperature, given by Clausius-Clapeyron
  ($\frac{d \; e_s}{d \; T}$), $R_{net}$ is the net radiation minus
  ground heat flux, $g_a$ is aerodynamic conductance, $\rho_a$ is air
  density, $c_p$ is specific heat of air at constant pressure,
  $\gamma$ is the psychometric constant, and $g_s$ is the stomatal
  conductance (Table \ref{definitions}). The issue with PM is that
  $g_s$ is a function of carbon uptake, which is has a strong
  functionally relation to ET through stomatal function. Therefore,
  when PM is formulated in terms of $g_s$, it is an implicit function
  of ET itself rather than an explicit function of ET. Here we will
  derive a new form of PM in which ET is an explicit function of plant
  parameters and environmental conditions, and use it to assess the
  ecosystem scale response to VPD (by taking a partial derivative).

\begin{table}
  \caption{Definition of symbols and variables, with citation for how
    values are calculated, if applicable.}
  \label{definitions}
  \centering \footnotesize
  \begin{tabular}{l c c c}
    \hline
    Variable & Description & Units & Citation \\
    \hline
    $e_s$  & saturation vapor pressure & Pa  & - \\
    $T$  & temperature  & K & - \\
    $P$  & pressure & Pa  & - \\
    $\Delta$  & $\frac{\partial e_s}{\partial T}$ & Pa K$^{-1}$ & - \\
    $R_{net}$  & net radiation at land surface minus ground heat flux & W m$^{-2}$   & - \\
    $g_a$  & aerodynamic conductance & m s$^{-1}$  & \citeA{Shuttleworth_2012} \\
    $\rho_a$  & air density & kg m$^{-3}$  & - \\
    $c_p$  & specific heat capacity of air at constant pressure & J K$^{-1}$ kg$^{-1}$ & - \\
    $VPD$  & vapor pressure deficit & Pa  & - \\
    $\gamma$  & psychometric constant & Pa K$^{-1}$   & - \\
    $g_{s}$  &  stomatal conductance & m s$^{-1}$
                                   & \citeA{Medlyn_2017} \\
    $g_{1}$  & ecosystem-scale slope parameter & Pa$^{0.5}$ & \citeA{Medlyn_2017} \\
    $R$ & universal gas constant & J mol$^{-1}$ K$^{-1}$ & - \\
    $R_{air}$ & gas constant of air & J  K$^{-1}$ kg$^{-1}$ & - \\
    $\sigma$ & uncertainty parameter & -& - \\
    $c_a$ & CO$_2$ concentration & $\mu$ mol CO$_2$ mol$^{-1}$ air& - \\
    $\lambda = \frac{\partial \; transpiration}{\partial\; CO_2\; assimilation}$ & marginal water cost of leaf carbon & mol H$_2$O mol$^{-1}$ CO$_2$ & - \\
    $\Gamma$ & CO$_2$ compensation point & - & - \\
    $\Gamma^*$ & CO$_2$ compensation point without dark respiration &
                                                                      - & - \\
    GPP & gross primary production & $\mu$-mol C s$^{-1}$ m$^{-2}$ & -
    \\
    ET & evapotranspiration & W m$^{-2}$ & - \\
   uWUE & underlying water use efficiency & $\mu$-mol C Pa$^{0.5}$
                                            J$^{-1}$ ET  &
                                                           \citeA{Zhou_2015} \\
    \hline
  \end{tabular}
\end{table}


\citeA{MEDLYN_2011} developed a model for stomatal conductance ($g_s$)
by combining an optimal photosynthesis theory \cite{Cowan_1977} with
an empirical approach, which describes the dependence of $g_s$ to
VPD. They also extended this model to the ecosystem scale
\cite{Medlyn_2017}:

% \begin{linenomath*}
  \begin{equation}
    g_s = \frac{R \,T}{P} \; 1.6 \left(1 + \frac{g_1}{\sqrt{VPD}}\right) \frac{GPP}{c_a},
    \label{medlyn}
  \end{equation}
% \end{linenomath*}

  where $g_{1}$ is a VPD ``slope'' parameter, GPP is the ecosystem
  scale gross primary production, and $c_a$ is the atmospheric CO$_2$
  concentration at the canopy. \cite{MEDLYN_2011} relate the slope
  parameter ($g_{1}$) to physical parameters as:
% \begin{linenomath*}

  \begin{equation}
    g_{1}  \propto \sqrt{\frac{3 \, \Gamma^* \,
        \lambda}{1.6}},\footnote{Note this expression has units of of
      (mmol mol$^{-1}$)$^{1/2}$, but this can be converted to
      Pa$^{1/2}$ using the ideal gas law.}
    \label{slope}
  \end{equation}
% \end{linenomath*}

where $\Gamma^*$ is the CO$_2$ compensation point for photosynthesis
(without dark respiration), and $\lambda$ is the marginal water cost
of leaf carbon
($\frac{\partial \; \text{transpiration}}{\partial \; CO_2 assimilation}$)
\cite{Farquhar_1980, Katul_2009}. So,
$g_{1}$ is a leaf-scale term reflecting the trade-off of water for
carbon uptake. The higher $g_{1}$, the more open the stomata and
the more they release water in exchange for carbon.


While Eq. (\ref{medlyn}) can be used in PM (Eq. (1)), it will make
analytical work with the function intractable because GPP is
functionally related to ET itself. Additionally, a perturbation to VPD
should induce a physiological plant response that will alter GPP and
cause an indirect change in stomatal conductance, in addition to the
direct effect of VPD in Eq. (\ref{medlyn}). Therefore, in order to
derive the full plant response of ET to VPD, we must account for the
functional relationship between GPP, ET, and VPD, and its effect on
stomatal conductance. We can use aforementioned semi-empirical results
of \citeA{Zhou_2015}, which were inspired by optimal photosynthesis
theory, as a tool to approach this problem. \citeA{Zhou_2015}, showed
that underlying Water Use Efficiency (uWUE):

% \begin{linenomath*}
  \begin{equation}
    uWUE = \frac{GPP \cdot \sqrt{VPD}}{ET}
    \label{uwue}
  \end{equation}
% \end{linenomath*}

  is relatively constant across time within a plant functional type,
  and correctly captures a constant relationship between GPP, ET and
  VPD over a diurnal cycle during the growing season
  \cite{Zhou_2014}. The theoretical derivation of the square root VPD
  dependence in uWUE leverages the same assumptions used in
  \citeA{MEDLYN_2011} to derive the square-root VPD dependence of the
  stomatal conductance model (Eq.  (\ref{medlyn})).  We can use uWUE
  to remove the $GPP$ dependence in $g_s$ in a way that makes PM
  analytically tractable:

% \begin{linenomath*}
  \begin{equation}
    g_s = \frac{R \, T}{P} 1.6 \left(1 + \frac{g_1}{\sqrt{VPD}}\right) \frac{uWUE \; ET}{c_a \; \sqrt{VPD}}.
    \label{new_g_s}
  \end{equation}
% \end{linenomath*}

Plugging Eq. (\ref{new_g_s}) into Eq. (\ref{orig_pen}) and
rearranging gives a new explicit expression for PM, in which
dependence on $GPP$ is removed:

% \begin{linenomath*}
  \begin{equation}
    ET = \frac{\Delta R_{net} + \frac{g_a\; P}{T} \left( \frac{ c_p VPD}{R_{air}} -  \frac{\gamma c_a \sqrt{VPD} }{ R \; 1.6\; \text{ uWUE } (1 + \frac{g_1}{\sqrt{VPD}})} \right) }{ \Delta + \gamma}
    \label{et}
  \end{equation}
% \end{linenomath*}

  By accounting for photosynthesis changes in ecosystem conductance,
  with Eq. (\ref{et}) we have used recent results
  \cite{MEDLYN_2011, Zhou_2014, Zhou_2015, Medlyn_2017} to derive
  ET explicitly as function of environmental variables and two
  plant-specific constants, the slope parameter ($g_1$), and uWUE. For
  the first time we have removed the implicit dependence of ET on
  itself through the stomatal conductance term, and we have also
  replaced the added complexity of a stomatal conductance reduction
  factor and a photosynthesis model with a single parameter (uWUE). Both
  the plant parameters reflect water conservation strategy. The
  slope parameter is related to the willingness of stomata to trade
  water for CO$_2$ and to keep stomata open (carbon cost in terms of
  water)). uWUE is a semi-empirical ecosystem-scale constant related
  to how WUE changes with VPD (specifically $VPD^{-1/2}$). It is also
  roughly proportional to physical constants:

\[uWUE \appropto \sqrt{\frac{c_a - \Gamma}{1.6 \lambda}},\]

where $\Gamma$ is the CO$_2$ compensation point \cite<Eq. (5)
in>[]{Zhou_2014}. So uWUE is related to atmospheric CO$_2$
concentration and compensation point, and is inversely proportional to
the marginal water cost of leaf carbon ($\lambda$). This relationship
with $\lambda$ ($\propto \lambda^{-1/2}$) is important as it is the inverse of g$_1$'s
relationship with $\lambda$ ($\propto \sqrt{\lambda}$). So, uWUE
should increase as $\lambda$ decreases, and g$_1$ should decrease as
$\lambda$ decreases.

With Eq. (\ref{et}) we can take the partial derivative of ET with respect
to VPD to understand whether VPD drives or reduces ecosystem-scale ET:

% \begin{linenomath*}
  \begin{equation}
    \frac{\partial \;  ET}{\partial \; VPD} = \frac{2\; g_a \;
      P}{T(\Delta + \gamma)}   \left(\frac{ c_p}{R_{air}} -
      \frac{\gamma c_a }{1.6 \; R\; \; \text{ uWUE }} \left(
        \frac{2 g_1 + \sqrt{VPD}}{2 (g_1 + \sqrt{VPD})^2}\right)
    \right),
    \label{d_et}
  \end{equation}
% \end{linenomath*}

  providing analytical framework for ecosystem response to
  atmospheric demand with environmental conditions held
  fixed. There are a few subtleties to taking the derivative in
  Eq. (\ref{d_et}): $\Delta$ ($\frac{d e_{s}}{d T}$) and $VPD$ are
  functionally related, so while taking the derivative we evaluate
  $\frac{\partial \; ET}{\partial \; VPD} = \frac{\partial \; ET}
  {\partial \; e_s} \frac{\partial \; e_s}{\partial \; VPD}
  \Big|_{\text{RH fixed}} + \frac{\partial \; ET}{\partial \; RH}
  \frac{\partial \; RH}{\partial \; VPD} \Big|_{\text{$e_s$
      fixed}}$. $RH$ and $e_s$ are assumed to be approximately
  independent, which is supported by data (not shown).

  This derivation relied either implicitly or explicitly on several
  assumptions. First, we assume that VPD at the leaf surface is the
  same as VPD at measurement height; physically this implies that
  leaves are perfectly coupled to the atmosphere. In reality, for some
  conditions and plant types the leaves can become decoupled from the
  boundary layer \cite{De_2017, Medlyn_2017}. Therefore, our
  derivation will be most applicable in times like the growing season
  (when we also expect uWUE to be most valid), when relatively high
  insolation induces instability and convective boundary layers, and
  we would expect the surface to be generally well coupled. An
  additional assumption in the formulation of uWUE \cite{Zhou_2014,
    Zhou_2015} and \citeA{Medlyn_2017}'s stomatal conductance model is
  that direct soil evaporation contributions to ET remain small
  relative to transpiration. Again, this should be more true during
  the growing season. The ratio of evaporation to transpiration may
  increase immediately after rainfall events due to high soil moisture
  and ponding, but VPD is generally low anyways during these
  times. However, some plant types allow for systematically larger
  contributions of evaporation in ET, particularly those with sparse
  canopies and smaller relative amounts of transpiration. We therefore
  might expect that the theory will be most applicable to forest PFTs,
  which will be most strongly coupled to the boundary layer due to
  larger surface roughness, and will also generally have the highest
  ratios of transpiration to evaporation. Finally, we assume that
  $g_1$ and uWUE are constant with respect to the conceptual VPD
  perturbation. Both quantities have been shown to be relatively
  constant with respect to changes in VPD \cite{Franks_2017,
    Zhou_2014}. These parameters will however vary with plant species
  and characteristics \cite<e.g. wood density, >[]{Lin_2015}, as well
  as environmental conditions including soil water content and
  temperature \cite{Lin_2015, Manzoni2013}. Exploring possible soil
  moisture (in)dependence of the plant parameters (g$_1$ and uWUE) is
  particularly interesting because soil moisture only enters the
  partial derivative directly through these plant parameters. If the
  plant parameters are weak functions of soil moisture then the theory
  can be directly applied to a broader range of conceptual VPD
  scenarios, including observed compound events between high VPD and
  low soil moisture \cite{Zhou_2019}. Supplementary material for this
  manuscript explores the soil moisture dependence of the plant
  parameters, but excessive noise and inconsistencies preclude
  conclusions about the nature of any dependence. We provide it in
  case it is of interest to the reader, and to motivate future
  research.

  Because Eq. (\ref{d_et}) is a partial derivative, it's utility is as
  a conceptual model for the change in ET in response to VPD with all
  other variables held fixed. This provides a useful tool for
  identifying the effect of VPD on ET through atmospheric demand and
  plant response, and allows a practitioner to disentangle complicated
  feedbacks when many quantities co-vary. However users of
  Eq. (\ref{d_et}) should take care that their interpretation matches
  the assumptions inherent in a partial derivative. For example,
  results will only be as valid as the assumption that g$_1$ and uWUE
  (and by extension $\lambda$) are fixed with respect to the user's
  conceptual change in VPD. Care must also be taken with possible
  indirect effects associated with a change in VPD: for example, a
  change in ET induced by a change in VPD can cause a change in
  surface temperature, which would drive a change in net
  radiation. These types of indirect effects and feedbacks are not
  considered in Eq. (\ref{d_et}): temperature (a variable) is
  mathematically fixed.

\subsection{Framing Eq. \ref{d_et} with previous research }


  To account for both the variability in plant parameters and the
  environment, we will systematically analyze how the ET response to
  VPD (Eq. (\ref{d_et})) varies. Eq. (\ref{d_et}) includes a
  ``sign'' term that determines the sign of the response in addition
  to magnitude:

\begin{equation}
  \label{sign}
  \frac{c_p}{R_{air}} - \frac{\gamma c_a }{1.6 \; R\; \text{ uWUE }} \left( \frac{2 g_1 + \sqrt{VPD}}{2 (g_1 + \sqrt{VPD})^2}\right),
\end{equation}

and a ``scaling'' term multiplying the ``sign'' term:

\begin{equation}
  \frac{g_a \; P}{T(\Delta + \gamma)}.
\end{equation}

In the ``sign term'' most of the quantities are relatively constant,
except for VPD and the plant parameters g$_1$ and uWUE. In the scaling
term, most of the terms are relatively constant with the exception of
aerodynamic conductance ($g_a$) and temperature (especially its effect
through $\Delta$). To determine the range of probable ET responses to
VPD we will systematically vary these parameters according to Table
\ref{param_varying}, while all other parameters are held fixed (Table
\ref{param_fixed}). Physical variables ($g_a$, $T$) and plant
physiological parameters (g$_1$, uWUE) are varied according to
literature-based expectations for a range of growing season conditions
and plant types \cite{Zhou_2015, Medlyn_2017}. Using this previous
literature we can connect the effect of varying plant parameters to
specific plant types and characteristics.

All code and data used in this analysis, including those used to
generate the figures and tables, are publicly available at
\url{https://github.com/massma/climate_et}.
%\hyperref[https://github.com/massma/climate\_et]{https://github.com/massma/climate_et}.

\begin{table}
  \caption{Variable quantities in the ET response to VPD. Each value
    is varied to determine the effect of a range of expected plant and
    environmental conditions on ET response to VPD. A citation is
    provided in cases where the quantities are directly derived from
    previous literature. Conceptually, min. values are extracted from
    literature to correspond to approximately the 15th percentile of
    observed conditions during the growing season, med. values
    correspond to approximately the 50th percentile, and max. values
    correspond to approximately the 85th percentile.}
  \label{param_varying}
  \centering
  \footnotesize
  \begin{tabular}{l c c c c c}
    \hline
    Symbol & units [units in citation] & min  & med & max
    & citation  \\
    \hline
    \input{param_varying.tex}
    \hline
  \end{tabular}
\end{table}

\begin{table}
  \caption{Quantities that are fixed in the ET response to VPD
    (relative to those in Table \ref{param_varying}).}
  \label{param_fixed}
  \centering
  \begin{tabular}{l c c}
    \hline
    Symbol & units & value \\
    \hline
    \input{param_fixed.tex}
    \hline
  \end{tabular}
\end{table}

\section{Results}
\label{results}

By varying four parameters (g$_a$, T, uWUE, g$_1$) at three different
values (Table \ref{param_varying}) we generate nine different values
for the scaling term, nine different curves (as a function of VPD) for
the sign term, and 81 different curves for the ET response to VPD
($\frac{\partial \; ET}{\partial \; VPD}$). To aid visualization we
can examine a subset of nine of these 81 curves, defined by the
minimum, median and maximum values for both the scaling and sign term
(Figure \ref{full}). The range of ET responses to VPD vary from those
where ET almost always decreases with increasing VPD (water
conservative), to those where ET almost always increases with
increasing VPD (water intensive). Additionally, for some parameters
whether ET will increase or decrease with increasing VPD depends on
atmospheric demand (Figure \ref{full}).

\begin{figure}
  \centering \includegraphics{./fully_idealized.pdf}
  \caption{The functional form of $\frac{\partial \; ET}{\partial
      \; VPD}$ for minimum, median and maximum values of both the sign
    term and the scaling term.}
  \label{full}
\end{figure}

\begin{figure}
  \centering \includegraphics{./fully_idealized_sign.pdf}
  \caption{The functional form of $\frac{\partial \; ET}{\partial
      \; VPD}$ evaluated at the median value of the scaling term, for
    varying values of g$_1$ and uWUE as given in Table \ref{param_varying}.}
  \label{sign}
\end{figure}

Examining the sign term independent of the scaling term illuminates
the role of the two plant parameters, g$_1$ and uWUE, in determining
the degree to which the response is water conservative
($\frac{\partial \; ET}{\partial \; VPD} < 0$) or water intensive
($\frac{\partial \; ET}{\partial \; VPD} > 0$) (Figure \ref{sign})
. Higher g$_1$ and uWUE shift the curve towards increasing ET
responses with VPD (water intensive), and smaller g$_1$ values lead to
a higher non-linear VPD dependence of the response. However care
should be exercised when interpreting the range of ET responses,
because g$_1$ and uWUE should generally be anti-correlated due to
their dependencies on $\lambda$. As $\lambda$ increases, g$_1$ should
increase and uWUE should decrease. Because high and low values of uWUE
and g$_1$ were determined independently from each other in previous
literature, plant types with anomalously high or low values for
\textit{both} g$_1$ and uWUE should be relatively rare. However, some
plant types do exhibit co-occuring high values for g$_1$ and uWUE. For
example, C3 crops have both the highest g$_1$ value in
\citeA{Franks_2017}, and the highest uWUE value in \citeA{Zhou_2015}.

\begin{figure}
  \centering \includegraphics{./fully_idealized_scaling.pdf}
  \caption{The functional form of $\frac{\partial \; ET}{\partial
      \; VPD}$ evaluated at the median value of the sign term, for
    varying values of g$_a$ and T as given in Table \ref{param_varying}.}
  \label{scaling}
\end{figure}

Examining the scaling term independent of the sign term shows how
aerodynamic conductance (communication between the atmosphere and the
surface) and temperature (controls the efficiency of energy conversion
to latent heat through $\Delta$), amplify or suppress the plant
response represented in the sign term (Figure \ref{scaling}). Both
lower temperatures and higher aerodynamic conductance lead to
amplified ET response to VPD, with the variability of aerodynamic
conductance resulting in a slightly higher variability of ET response
relative to temperature.

\section{Discussion}
\label{discussion}

Interpreting both the derivation and results of this manuscript hinges
on an understanding and appreciation for the usefulness of partial
derivatives for understanding the behavior of complex systems in a
simplified framework. Our results are a conceptual tool for answering
the question, ``What is the ET response to VPD with all other
quantities held fixed?'' This is useful, and we would argue is
critical, for understanding complex responses when many quantities are
varying simultaneously \cite{Zhou_2019}. If we cannot understand the
contribution of one term to observed variability, independent of other
confounding factors, there is little hope of disentangling and
understanding the relative role of many co-varying quantities in
determining observed variability.  Eq. (\ref{d_et}) explicitly
provides an estimate of the ET response to VPD, given assumptions
about ecosystem plant characteristics through the parameters g$_1$ and
uWUE. Eq. (\ref{d_et}) is generic to external environmental factors
and the timescale over which a VPD response acts, and has a
straightforward conceptual interpretation. It provides the ET
response, given other quantities and parameters held fixed, subject to
the assumptions outlined in Sect. \ref{methods}. So, while the plant
parameters g$_1$ and uWUE may vary with plant species and
environmental conditions like soil moisture, we can still assess the
ET response to VPD for a given ecosystem state, or for a given soil
moisture condition.

However, given that g$_1$ and uWUE can vary, and they both have a
large impact on the ET response (Figure \ref{full}, \ref{sign}), it is
useful to examine how these quantities vary in previous literature,
framed by our results on $\frac{\partial \; ET}{\partial \; VPD}$. In
\citeA{Zhou_2015} most plant types have uWUE similar to our median
value, under which the sign of the ET response depends on
VPD. However, crops, which we expect to prioritize carbon uptake over
water conservation, have a higher uWUE value closer to our maximum
value. This higher uWUE results in a higher likelihood for an
increasing ET response to VPD, which matches intuition given that we
expect crops to keep stomata open for access to carbon, at the cost of
increased water loss during high VPD. Shrubs, and to a lesser extent
grass, have an uWUE closer to the minimum uWUE. For these plant types,
we would then expect a decrease of ET in response to VPD. It is
important to note that, while \citeA{Zhou_2015} did not examine the
role of soil moisture for within plant-type variability of uWUE, we
might expect some variation, especially in extreme cases when soil
water becomes a limiting factor (see supplementary material).

\citeA{Medlyn_2017}'s results did not display the robust relationship
between plant type and and plant parameter (g$_1$) that was
demonstrated by \citeA{Zhou_2015} for uWUE. Given that there is a
robust relationship between g$_1$ and plant type in \citeA{Lin_2015},
and that \citeA{Zhou_2015}'s ecosystem-scale results rely on the same
underlying theory as \citeA{Medlyn_2017}'s results, the ambiguous
results in \citeA{Medlyn_2017} could be due to noise in ecosystem
scale observations, the consequences of imposing a model structure in
some of their g$_1$ estimations, or real ecosystem-scale within-plant
type variability (e.g. multiple species impacting the relationship
within the eddy-covariance footprint). These differences between
leaf-scale behavior and ecosystem scale behavior highlight the
importance of understanding how and why slope parameters show
increased variability at the ecosystem scale, as argued by
\citeA{Medlyn_2017}. However, if we assume that there is some real
analogy between leaf scale results in \citeA{Lin_2015} and ecosystem
scale behavior, and that some of the ambiguity in \citeA{Medlyn_2017}
is due to model and/or observational error, then we can use the
relationships between g$_1$ and plant characteristics defined by \citeA{Lin_2015} to
frame the ET response to VPD. This is subject
to the strong caveat that we still do not fully understand g$_1$
behavior at the ecosystem scale. Extrapolating \citeA{Lin_2015}'s
results (e.g. Fig. 2) to the ecosystem scale would reveal the
following relationships between g$_1$ and plant types:

\begin{itemize}
  \item C3 plants would have a generally higher g$_1$ than C4 plants,
  \item Crops would have a larger g$_1$ than shrubs, grass, and
    angiosperm trees, which have a higher g$_1$ than gymnosperm
    trees.
  \item Tropical and temperate climates would generally be
    characterized by plants with a higher g$_1$ than arctic and boreal
    climates.
\end{itemize}
For all of the above relationships, a higher g$_1$ means a higher
likelihood of a positive ET response to increasing VPD (water
intensive), and a smaller VPD dependence of the ET response.

Our theoretical results highlight the variability of ET
response to VPD as a function of both climate  and plant
characteristics, especially water usage strategy. Whether an
ecosystem increases or decreases ET in response to VPD will vary
depending on vegetation and climate. Generalizations about ET response
to VPD therefore require thoughtful consideration of ecosystem
characteristics; statements such as ``ET increases with warming due to
increases in VPD'' may be false depending on the water conservation
strategy of a given ecosystem \cite{Lemordant_2018}.

Additionally, the sensitivity of ET response to plant parameters
highlights the importance of understanding and developing stomatal
conductance models and parameters for the ecosystem scale. In
particular, why and how g$_1$ behavior at the leaf scale is not
analogous to the ecosystem scale must be understood. While the
derivation here was explicit about the assumption of a constant g$_1$
term with respect to the VPD perturbation, many sophisticated land
surface models and earth system models employing stomatal conductance
models make similar assumptions about the stationarity of VPD slope
terms \cite{Niu_2011, Franks_2017, Rogers_2017, Lawrence_2019}. Many
models will assume a constant VPD slope term (e.g. g$_1$) within
a given plant functional type, and research derived from these models
often does not acknowledge limitations of this assumption, or the
difficulty in the literature with quantifying an ecosystem scale slope
parameter \cite{Medlyn_2017}. Given that something so fundamental as
the ET response to VPD varies strongly with g$_1$, we must invest more
in understanding g$_1$'s behavior at the ecosystem scale, and the
efficacy of models using constant slope terms for ecosystem-scale
fluxes. Fortunately, our framework, and the new approach of using
robust ecosystem-scale ratios (uWUE) to remove the implicit dependence
of stomatal conductance on ET, is flexible enough to be applied to any
stomatal conductance model that contains a dependence on GPP.

So far we have discussed the role of different parameters in
modulating the ET response. However, the form of the ET response is
imposed by our choice of stomatal conductance model. We now examine
how the stomatal model choice, and specifically the exponent of the
VPD dependence in the stomatal conductance model, can impact the
general form of the ET-VPD relationship. There is a theoretical basis
for the square root VPD dependence in both the stomatal conductance
model and uWUE based on the assumption that stomata behave to maximize
carbon gain while minimizing water loss, which observations also
generally support \cite{Lloyd_1991, MEDLYN_2011, Lin_2015, Zhou_2014,
  Zhou_2015, Medlyn_2017}. However, some purely empirical results that
fit the exponent of the VPD dependence to data have shown that it may
vary slightly from 1/2, suggesting that stomata, as well as
ecosystem-scale quantities based on stomata theory, may not always
function optimally \cite{Zhou_2015, Lin_2018}. Specifically with
regards to uWUE, one would not expect that this ecosystem scale WUE
quantity will respond to VPD exactly analogously to stomata. Direct
soil evaporation's contributions to ET should shift the exponent of
the VPD dependence. \citeA{Zhou_2015}'s results corroborate this: they
found a mean empirically fit exponential VPD dependence of 0.55,
varying slightly from the theoretically optimal value of of 0.5 for
AmeriFlux sites. \citeA{Lin_2018}'s results also show variance in the
empirical exponent of the VPD dependence in the stomatal conductance
model, but interpretation of this variance is more difficult as
\citeA{Lin_2018} do not handle the GPP dependence of stomatal
conductance in a directly analogous manner to the optimal theory in
\citeA{MEDLYN_2011} and \citeA{Medlyn_2017}. Regardless, given that
these recent results on the relationship between VPD, GPP, and ET
\cite{MEDLYN_2011, Zhou_2014, Zhou_2015, Medlyn_2017} form the
backbone of our analysis and are what allowed us to derive an explicit
ET expression for the first time (Eq. (\ref{et})), we will analyze if
and how assumptions about the exponent of the VPD dependence impact
the shape of the ET-VPD dependence. This analysis is also important to
understand whether the choice of stomatal conductance model alters the
fundamental behavior of the ET-VPD relationships, as many commonly
used models utilize a VPD exponent other than the 1/2 suggested by
optimal theory \cite<e.g. >[ which uses an exponent of
1]{Leuning_1990}. It is useful to understand how the choice of model
imposes the functional form of the ET-VPD relationship, given its
fundamental importance for land-atmosphere interactions.


\subsection{Functional form of ET dependence on VPD and its relation
  to the VPD exponent}
\label{functional_form}

The results presented in Sect. \ref{results} indicates that for a
given uWUE and $g_1$, the ET dependence on VPD should be concave
upward. In other words, there should be some local minimum in ET at a
critical VPD$_{crit}$, assuming the scaling and plant terms
(e.g. aerodynamic conductance, $\Delta$, $g_1$ and uWUE) are held
fixed. This result warrants further investigation, because to our
knowledge no earlier work has derived the theoretical ecosystem-scale
relationship between ET and VPD in an energy balance framework, while
controlling for other environmental conditions. In particular there is
an apparent lack of consensus over whether the shape of the ET-VPD
curve should be concave upward (our result) or concave downward in the
absence of dramatic water stress. Given that understanding the ET-VPD
relationship of the one-way plant response is fundamental to
hypothesizing about any feedbacks between the land surface and the
atmosphere, we analyze why our derived ET-VPD relationship is concave
upward, particularly with respect to the exponent of VPD dependence in
uWUE and the Medlyn unified stomatal conductance model, as other
models and empirical results have suggested exponents varying from 1/2
\cite{Leuning_1990, Zhou_2015, Lin_2018}.

By introducing $n$, the exponent of VPD in uWUE, and $m$, the exponent
of VPD in the stomatal conductance model, we can free our theory from
assumptions about VPD dependence:
% \begin{linenomath*}
  \begin{equation}
    g_s = \frac{R \, T}{P} 1.6 \left(1 + \frac{g*}{VPD^m}\right) \frac{*WUE \; ET}{c_a \; VPD^n},
    \label{m_n}
  \end{equation}
% \end{linenomath*}
where:
\[*WUE = \frac{GPP}{ET}VPD^n,\] and $g*$ is a generic slope parameter
of units $VPD^m$. To determine how the exponent $n$ and $m$ alter the
shape of the ET-VPD dependence we find the roots of the second
derivative of ET, using Eq. (\ref{m_n}) for stomatal conductance
($g_s$), with respect to VPD:

\input{d2_solutions.tex}

With this
result we have defined the family of curves separating concave up from
concave down ET solutions (Fig. \ref{concave}). These curves are
only functions of the exponent of the VPD dependence and a quantity we
call non-dimensional VPD ($VPD^m/g*$). Several important
relations reveal themselves from Eq. (\ref{curves}):

\begin{figure}
  \centering
  \centerline{\includegraphics[width=0.75\textwidth]{./concave.pdf}}
  \caption{ Solutions corresponding to inflection points between
    concave up and concave down ET-VPD curves (Eq. (\ref{curves}))
    for three specific scenarios. Solutions are defined in terms of a
    non-dimensional VPD ($VPD^m/g_*$), but to aide physical
    interpretation the horizontal axis is additionally provided in terms of dimensionalized VPD assuming $m=1/2$ and
    $g_*=110\; Pa^{1/2}$ (average of all PFT $g_1$). The vertical axis
    has a different interpretation depending on the solution
    curve. For the blue line ($m$ varying), it corresponds to $m$,
    for the orange line ($n$ varying) it corresponds to $n$, and for
    the green line it corresponds to the value of both $n$ and $m$
    ($n=m$). Regions of the parameter space that correspond to
    concave up and concave down results are labeled.}
  \label{concave}
\end{figure}

\begin{itemize}
  \item For optimal behavior (n, m = 1/2) the ET-VPD curve will be
    concave up regardless of the magnitude of the plant constants
    $g_1$ and uWUE. Therefore, the general concave up nature of our
    results, given an assumption of optimal behavior, is insensitive
    to plant type.
  \item For all physically possible exponents of VPD dependence ($n,
    m$), whether the solution is concave up or concave down does not
    depend on $*WUE$.
  \item In general, increasing the exponent of VPD dependence for
    either $*WUE$ or $g_*$ increases the likelihood of a concave down
    result. Additionally, as the exponent of VPD dependence increases
    from the optimum value of 1/2, whether the curve is concave upward
    or concave downward becomes a function of the plant specific slope
    parameter $g_*$, through non-dimensional VPD
    ($VPD^m/g_*$). Because the exponent of the VPD dependencies is
    capable of altering the fundamental shape of ET-VPD dependence,
    future research investment in understanding the exact VPD
    dependence of stomatal conductance, and further reconciliation of
    empirical and theoretical stomatal and ecosystem behavior should
    be prioritized.
\end{itemize}

While it is possible that in the future some other form of VPD
dependence is derived, at present \cite{MEDLYN_2011} and
\cite{Zhou_2014} established n=m=1/2 as the most likely candidate
given current theory and empirical data. Additionally, we argue that a
concave up result matches physical intuition more than a concave down
result. Plants must maintain nutrient and sugar transport through the
phloem and xylem. To accomplish this, stomata must remain slightly
open \cite{De_2013, Nikinmaa_2013, Ryan_2014}. Furthermore, even if
complete stomatal closure were possible, cuticular water loss and [at
the ecosystem-scale] direct soil evaporation are still sources of ET
which increase with VPD, independent of stomatal closure. Therefore,
in the limit as VPD becomes large and we assume plants are exercising
all strategies to reduce ET, any further increase in VPD should result
in an increase in ET through cuticular water loss and/or direct soil
evaporation. This inevitable transition from conditions when stomata
respond strongly to VPD to conditions when stomata response is
asymptoting towards full closure would cause a concave up ET-VPD
curve, which is matched by our theory. In short, plant response
becomes more limited as VPD increases, while atmospheric demand
monotonically increases with VPD, leading to the result that
atmospheric demand dominates plant response when atmospheric demand is
high.

This analysis allows us to understand the theoretical shape of the ET
response to VPD with environmental conditions held
fixed. Accomplishing this with purely statistical methods applied to
flux observations would be very difficult, given the relatively fast
time scale of plant response and the non-stationarity of [solar
forced] environmental conditions over the relatively coarse (half
hourly) flux estimates (which is required to obtain robust
eddy-covariance statistics). Our results on the shape of the ET-VPD
curve with environmental conditions held fixed can be built upon with
future work examining how changes in VPD and environmental conditions
(e.g. soil water storage) feedback upon one another. In the soil water
storage example, over very long time scales extremely high VPD
perturbations coupled with no precipitation could result in decreases
in soil water storage such that water becomes limiting. This could be
represented by an extension of our framework in which $\lambda$ is
allowed to decrease with decreasing soil water, increasing uWUE and
decreasing g$_1$. Here, we focus our results by framing them with
previous literature to build baseline intuition for ET-VPD dependence.

