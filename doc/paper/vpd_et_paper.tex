% 11/23/2015
%%%%%%%%%%%%%%%%%%%%%%%%%%%%%%%%%%%%%%%%%%%%%%%%%%%%%%%%%%%%%%%%%%%%%%%%%%%%
% AGUJournalTemplate.tex: this template file is for articles formatted with LaTeX
%
% This file includes commands and instructions
% given in the order necessary to produce a final output that will
% satisfy AGU requirements. 
%
% You may copy this file and give it your
% article name, and enter your text.
%
%%%%%%%%%%%%%%%%%%%%%%%%%%%%%%%%%%%%%%%%%%%%%%%%%%%%%%%%%%%%%%%%%%%%%%%%%%%%
% PLEASE DO NOT USE YOUR OWN MACROS
% DO NOT USE \newcommand, \renewcommand, or \def, etc.
%
% FOR FIGURES, DO NOT USE \psfrag or \subfigure.
% DO NOT USE \psfrag or \subfigure commands.
%%%%%%%%%%%%%%%%%%%%%%%%%%%%%%%%%%%%%%%%%%%%%%%%%%%%%%%%%%%%%%%%%%%%%%%%%%%%
%
% Step 1: Set the \documentclass
%
% There are two options for article format:
%
% 1) PLEASE USE THE DRAFT OPTION TO SUBMIT YOUR PAPERS.
% The draft option produces double spaced output.
% 
% 2) numberline will give you line numbers.

%% To submit your paper:
\documentclass[draft,linenumbers]{agujournal}
% \draftfalse
\drafttrue

%% For final version.
% \documentclass{agujournal}

% Now, type in the journal name: \journalname{<Journal Name>}

% ie, \journalname{Journal of Geophysical Research}
%% Choose from this list of Journals:
%
% JGR-Atmospheres
% JGR-Biogeosciences
% JGR-Earth Surface
% JGR-Oceans
% JGR-Planets
% JGR-Solid Earth
% JGR-Space Physics
% Global Biochemical Cycles
% Geophysical Research Letters
% Paleoceanography
% Radio Science
% Reviews of Geophysics
% Tectonics
% Space Weather
% Water Resource Research
% Geochemistry, Geophysics, Geosystems
% Journal of Advances in Modeling Earth Systems (JAMES)
% Earth's Future
% Earth and Space Science
%
%

\journalname{Agricultural and Forest Meteorology}

\begin{document}

%% ------------------------------------------------------------------------ %%
%  Title
% 
% (A title should be specific, informative, and brief. Use
% abbreviations only if they are defined in the abstract. Titles that
% start with general keywords then specific terms are optimized in
% searches)
%
%% ------------------------------------------------------------------------ %%

% Example: \title{This is a test title}

\title{When does vapor pressure deficit drive or reduce evapotranspiration?}

%% ------------------------------------------------------------------------ %%
%
%  AUTHORS AND AFFILIATIONS
%
%% ------------------------------------------------------------------------ %%

% Authors are individuals who have significantly contributed to the
% research and preparation of the article. Group authors are allowed, if
% each author in the group is separately identified in an appendix.)

% List authors by first name or initial followed by last name and
% separated by commas. Use \affil{} to number affiliations, and
% \thanks{} for author notes.  
% Additional author notes should be indicated with \thanks{} (for
% example, for current addresses). 

% Example: \authors{A. B. Author\affil{1}\thanks{Current address, Antartica}, B. C. Author\affil{2,3}, and D. E.
% Author\affil{3,4}\thanks{Also funded by Monsanto.}}

\authors{A. Massmann\affil{1}, P. Gentine\affil{1}, C. Lin\affil{2}}


% \affiliation{1}{First Affiliation}
% \affiliation{2}{Second Affiliation}
% \affiliation{3}{Third Affiliation}
% \affiliation{4}{Fourth Affiliation}

\affiliation{1}{Department of Earth and Environmental Engineering, Columbia University, New York, NY 10027}
\affiliation{2}{Department of Hydraulic Engineering, Tsinghua University, Beijing, CN}

  % (repeat as many times as is necessary)

%% Corresponding Author:
% Corresponding author mailing address and e-mail address:

% (include name and email addresses of the corresponding author.  More
% than one corresponding author is allowed in this LaTeX file and for
% publication; but only one corresponding author is allowed in our
% editorial system.)  

% Example: \correspondingauthor{First and Last Name}{email@address.edu}

\correspondingauthor{Adam Massmann}{akm2203@columbia.edu}

%% Keypoints, final entry on title page.

% Example: 
% \begin{keypoints}
% \item	List up to three key points (at least one is required)
% \item	Key Points summarize the main points and conclusions of the article
% \item	Each must be 100 characters or less with no special characters or punctuation 
% \end{keypoints}

%  List up to three key points (at least one is required)
%  Key Points summarize the main points and conclusions of the article
%  Each must be 100 characters or less with no special characters or punctuation 

\begin{keypoints}
\item = enter point 1 here = 
\item = enter point 2 here = 
\item = enter point 3 here = 
\end{keypoints}

%% ------------------------------------------------------------------------ %%
%
%  ABSTRACT
%
% A good abstract will begin with a short description of the problem
% being addressed, briefly describe the new data or analyses, then
% briefly states the main conclusion(s) and how they are supported and
% uncertainties. 
%% ------------------------------------------------------------------------ %%

%% \begin{abstract} starts the second page 

\begin{abstract}
= enter abstract here =
\end{abstract}


%% ------------------------------------------------------------------------ %%
%
%  TEXT
%
%% ------------------------------------------------------------------------ %%

%%% Suggested section heads:
% \section{Introduction}
% 
% The main text should start with an introduction. Except for short
% manuscripts (such as comments and replies), the text should be divided
% into sections, each with its own heading. 

% Headings should be sentence fragments and do not begin with a
% lowercase letter or number. Examples of good headings are:

% \section{Materials and Methods}
% Here is text on Materials and Methods.
%
% \subsection{A descriptive heading about methods}
% More about Methods.
% 
% \section{Data} (Or section title might be a descriptive heading about data)
% 
% \section{Results} (Or section title might be a descriptive heading about the
% results)
% 
% \section{Conclusions}


\section{Introduction}\explain{This section needs to be fleshed out, and I definitely need to add more citations}


Changes to vapor pressure deficit (VPD) alter the atmospheric demand for water from the land surface. Traditionally, atmospheric scientists and hydrometeorologists generally think that an increase in atmospheric demand induces an increases in evapotranspiration (ET) (citations?). This possible misconception developed in part due to the proliferation of studies examing potential ET (PET) rather than estimates of ET itself (citations?).  In contrast, plant physiologists know that stomata have evolved to optimally regulate the exchange of water and carbon, and tend to close in response to increased atmospheric dryness \citep{Ball_1987, Leuning_1990, MEDLYN_2011}.  Therefore, an increase (decrease) in VPD may not correspond to an increase (decrease) in ET because stomatal closure (opening) can cancel the effects of shifts to atmospheric demand.

Quantifying the plant response to a perturbation to atmospheric VPD increases our understanding of land surface response to shifts in atmospheric conditions. If plant response reduces ET in response to atmospheric drying then soil moisture will be conserved. An increase in ET in reponse to atmospheric drying will reduce soil moisture, but contribute increased moisenting to the atmosperhere. Clearly, the sign and magnitude of land-surface responsedrives the co-evolution of the atmosphere and land-surface at many timescales, from diurna to interdecadal.

We hypothesize that for most plant types a common resonse to increase in VPD will actually be a decrease in ET. The excpetion would be plants such as crops that are evolved (or bred) to prioritize gross primary production (GPP) over water conservation. However, for all otehr plants types, our hypothesis calls into question the validity of PET-based drought metrics deceloped by hydrometeorologists and used extensively in operations \citep[e.g. PDSI, P-PET,][]{2002, Alley_1984} \explain{more citations needed, including recent PET climate studies like Jack Scheff}. These metrics ignore the role of plants as gatekeepers for surface water loss to the atmosphere and have limited physical meaning for drought of vegetated land types. Additionally, plants evolved in arid climates should prioritize water conservation and we would expect a very negative ET response to increase in VPD. Therefore, vegetated locations most likely to experience droughts should show the strongest deviation between reality and a PET-based approximation. 

In order to quantify plant response to perturbations to atmospheric demand for water, we apply a Penman-Monteith framework to eddy-covariance observations spanning various biomes and climates. Section 2 describes the data used, Section 3 derives the framework, Section 4 presents results, and Section 5 discusses conclusions. The goal of this paper is to use reasonable approximations as a tool to increase intuition for plant response to atmospheric drying. This intuition will aid interpretation of observations and full complexity climate models. 

\section{Data}
\label{data}
We use data from FLUXNET2015. Because $g_1$ coefficients \citep{Lin_2015} and uWUE were only both available for five plant functional types (PFTs - see Table \ref{pft}),  only 56 of the 77 sites were used. Figure \ref{map_fig} \explain{map needs to be improved - it's a placeholder for now} presents each site and its plant functional type.

\section{Methods}
n
The Penman-Monteith equation (hereafter PM) estimates ET as a function of atmospheric and land-surface variables:
\begin{linenomath*}
  \begin{equation}
      ET = \frac{\Delta R + g_a \rho_a c_p VPD}{\Delta + \gamma(1 + \frac{g_a}{g_s})},
  \end{equation}
\end{linenomath*}

 where variable definitions are given in Table 1. \citet{MEDLYN_2011} developed a model for $g_s$ by combining optimal photosynthesis theory with empirical approaches. The result for leaf-scale stomatal resistance was:

\begin{linenomath*}
  \begin{equation}
  g_{l-s} = g_0 + 1.6 \left(1 + \frac{g_1}{\sqrt{VPD}}\right) \frac{A}{c_s}
  \end{equation}
\end{linenomath*}

This can be adapted to an ecosystem-scale stomatal resistance by multiplying by leaf area index (LAI) and converting units to m s$^{-1}$

\begin{linenomath*}
  \label{medlyn}
  \begin{equation}
  g_s = \text{LAI} \frac{R \,T}{P} \left( g_0 + 1.6 \left(1 + \frac{g_1}{\sqrt{VPD}}\right) \frac{A}{c_s}\right)
  \end{equation}
\end{linenomath*}

While Equation 3 can be used in PM, it will make analytical work with the function intractable because $A$ is a function of ET itself. To remove dependence of ET on $A$ we can use the semi-empirical results of \citet{Zhou_2015}. \citet{Zhou_2015} showed that:

\begin{linenomath*}
  \begin{equation}
    \label{uwue}
uWUE = \frac{GPP \cdot \sqrt{VPD}{ET}
  \end{equation}
\end{linenomath*}
is relatively constant across time and space (within plant functional type). If, following \citet{Lin_2015}, we approximate $g_0$ as $0$, we can use uWUE to remove $A$ from $g_s$ in a way that makes PM analytically tractable:

\begin{linenomath*}
  \begin{equation}
  g_s = \frac{R \, T}{P} 1.6 \left(1 + \frac{g_1}{\sqrt{VPD}}\right) \frac{uWUE \; ET}{c_s \; \sqrt{VPD}}
  \end{equation}
\end{linenomath*}

Note that $uWUE$ is fit on the ecosystem scale in \citet{Zhou_2015} so GPP in \ref{uwue} is really $A\cdot \text{LAI}$. This leads to the cancelation of LAI in addition to uWUE in Equation \ref{medlyn}. Plugging Equation 5 into Equation 1 and rearranging gives:

\begin{linenomath*}
  \begin{equation}
    ET = \frac{\Delta R + \frac{g_a\; P}{T} \left( \frac{ c_p VPD}{R_{air}} -  \frac{\gamma c_s \sqrt{VPD} }{ R* \; 1.6\; \text{ uWUE } (1 + \frac{g_1}{\sqrt{VPD}})} \right) }{ \Delta + \gamma}
    \label{et}
  \end{equation}
\end{linenomath*}

Given FLUXNET data described in Section \ref{data}, every term in Equation \ref{et} is known. However, our sampling of sites and our focus on the growing season may intoduce some deviations of $uWUE$ from those observed in \citet{Zhou_2015}. Also, we wish to include some measure of uncertainty in our analysis to guide if our many assumptions and simplifications are reasonable. To account for both mean deviations of $uWUE$ and uncertainty, we will introduce an uncertainty parameter $\sigma$ modifying $uWUE$:

\begin{linenomath*}
  \begin{equation}
    ET = \frac{\Delta R + \frac{g_a\; P}{T} \left( \frac{ c_p VPD}{R_{air}} -  \frac{\gamma c_s \sqrt{VPD} }{ R* \; 1.6\; \sigma \; \text{ uWUE } (1 + \frac{g_1}{\sqrt{VPD}})} \right) }{ \Delta + \gamma}
    \label{et_sigma}
  \end{equation}
\end{linenomath*}

Now, from each FLUXNET observation we can calculate single value of $\sigma$:

\begin{linenomath*}
  \begin{equation}
\sigma = - \frac{g_a \gamma c_s \sqrt{VPD} L_v P }{ \left(\text{ ET } ( \Delta + \gamma) - \Delta R - g_a \rho_a c_p VPD\right) 1.6 \; R\; T\; \text{ uWUE } (1 + \frac{g_1}{\sqrt{VPD}})}
    \label{et_sigma}
  \end{equation}
\end{linenomath*}

The variablility of $\sigma$ across sites and time will provides some measure of uncertainty in our model, assumptions, as well as the fluxnet observations themselves. Mean shifts in $\sigma$ can be attributed to differences in out sampling from those used to calcualte $uWUE$ in \citet{Zhou_2015}. The influence of $\sigma$ will propogate through any uncertainty to our derivative of Equation \ref{et_sigma}:

\begin{linenomath*}
  \begin{equation}
    \frac{\partial \;  ET}{\partial \; VPD} = \frac{2\; g_a \; P}{T(\Delta + \gamma)}   \left(\frac{ c_p}{R_{air}} -  \frac{\gamma c_s }{1.6 \; R*\; \sigma \; \text{ uWUE }} \left( \frac{2 g_1 + \sqrt{VPD}}{2 (g_1 + \sqrt{VPD})^2}\right) \right)
    \label{d_et}
  \end{equation}
\end{linenomath*}

With Equation \ref{d_et} we have provided an analytical franework for ecosystem reponse to atmospheric demand perturbations. There are a few subtelties to taking the derivative in Equation \ref{d_et}: $\Delta$ ($\frac{d e_{s}}{d T}$) and $VPD$ are functionally related, so while taking the derivative we evaluate $\frac{\partial \; ET}{\partial \; VPD} = \frac{\partial \; ET} {\partial \; e_s} \frac{\partial \; e_s}{\partial \; VPD} \Big|_{\text{RH fixed}} + \frac{\partial \; ET}{partial \; RH} \frac{\partial \; RH}{\partial \; VPD} \Big|_{\text{$e_s$ fixed}}$. $RH$ and $e_s$ are assumed to be approximately orthogonal.



The $VPD$ dependence in Equation \ref{d_et} is a little opaque. However, mean $VPD$ is 1062 Pa, so $\sqrt{VPD}$ is 32.6 Pa$^{1/2}$, which is much less than $g_1$ (with the exception of ENF; Table \ref{pft}). So a series expansion in the limit $\frac{\sqrt{VPD}}{g_1} \to 0$ gives an approximation which makes the functional form more clear: \explain{Should I even include the series expansion?}

\begin{linenomath*}
  \begin{equation}
    \frac{\partial \;  ET}{\partial \; VPD} \approx \frac{g_a \; P}{T(\Delta + \gamma)}   \left(\frac{ c_p}{R_{air}} -  \frac{\gamma c_s }{1.6 \; R*\; \sigma \; \text{ uWUE }} \left( \frac{1}{g_{1}} - \frac{3 \sqrt{VPD}}{2 g_{1}^{2}} + \frac{2 \sqrt{VPD}^{2}}{g_{1}^{3}} - \frac{5 \sqrt{VPD}^{3}}{2 g_{1}^{4}} + \mathcal{O}\left(\left(\frac{\sqrt{VPD}}{g_1}\right)^{4}\right) \right) \right)
    \label{d_et_approx}
  \end{equation}
\end{linenomath*}

One final comment on our derivation which will not be discussed further but is relevant for future analysis: if we  approximage $c_s$ at a global mean CO$_2$ concentration, then the RHS of Equation \ref{et} is fully defined using commonly available weather station data and the constants published in \citet{Zhou_2015, Lin_2015}. This then begs the question, why use PET for drought metrics in vegetated areas? It appears a much more physically realistic estimate of ET can be had with the same information reqiured to calculate PET. 

\begin{table}
\caption{Definition of symbols and variables}
\centering
\begin{tabular}{l c c}
\hline
 Variable & Description & Units  \\
\hline
$e_s$  & saturation vapor pressure & Pa  \\ 
$T$  & temperature  & K \\
$\Delta$  & $\frac{\partial e_s}{\partial T}$ & Pa K$^{-1}$ \\
$R$  & net radiation at land surface minus ground heat flux & W m$^{-2}$   \\
  $g_a$  & aerodynamic conductance & m s$^{-1}$  \\
  $\rho_a$  & air density & kg m$^{-3}$  \\
  $c_p$  & specific heat capacity of air at constant pressure & J K$^{-1}$ kg$^{-1}$ \\
  $VPD$  & VPD & Pa  \\
  $\gamma$  & psychrometric constant & Pa K$^{-1}$   \\
  $g_s$  & stomatal conductance & m s$^{-1}$  \\
  $g_{l-s}$  & leaf-scale stomatal conductance & mol m$^{-2}$ s$^{-1}$  \\
  $R*$ & universal gas constant & J mol$^{-1}$ K$^{-1}$ \\
  $LAI$ & leaf area index & -\\
  $\sigma$ & uncertainty parameter & -\\
  $c_s$ & CO$_2$ concentration & $\mu$ mol CO$_2$ mol$^{-1}$ air\\
\hline
\multicolumn{2}{l}{$^{a}$Footnote text here.}
\end{tabular}
\end{table}




\begin{table}
  \label{pft}
\caption{Plant functional types, their abbreviation, Medlyn coefficient \citep[from ][]{Lin_2015}, and uWfUE \citep[from ][]{Zhou_2015}. Note that units are converted such that the quantities fit into Equations 1-8 with the variables in Table 1.}
\centering
\begin{tabular}{l c c c}
  \hline
  Abbreviation & PFT & $g_1$ (Pa$^{0.5}$) & uWUE ($\mu$-mol [C] Pa$^{0.5}$ J$^{-1}$ [ET])  \\
  \hline
  CRO & cropland & 183.1 & 3.80 \\
  CSH & closed shrub & 148.6 & 2.18 \\
  DBF & deciduous broadleaf forest & 140.7 & 3.12 \\
  ENF & evergreen needleleaf forest & 74.3 & 3.30 \\
  GRA & grassland (C3) & 166.0 & 2.68 \\
\hline
\multicolumn{2}{l}{$^{a}$Footnote text here.}
\end{tabular}
\end{table}

\begin{figure}[h]
\centering
\includegraphics[width=20pc]{./fig01.pdf}
\caption{Plant functional type and location of sites used in analysis. ***This is just a placeholder for now and needs to be improved i.e. with lat lon, better placement of continents, etc.)***}
\label{map_fig}
 \end{figure}

We restrict our analysis to the daytime (sensible heat > 5 W m$^{-1}$ and shortwave radiation > 50 W m$^{-2}$) when there is no precipitation and the plants are growing (GPP > 10\% of the 95th percentile) \explain{This GPP thresholding was used by Changjie, do you know if there is a citation for it? Otherwise it seems like something a reviewer would have issue with as it is arbitray}. Also, because some sites use half hourly data but some use hourly, we aggregate all data to hourly averages. Only times with good quality control flags are used.

\section{Results}
\label{results}

By construction, the variability in the $\sigma$ term (Equation \ref{lai}) contains all model and observational uncertainties. For an observation that perfectly matches our model and assumptions $\sigma$ will be one. Therefore, if for our assumptions and framework to be reasonable $\sigma$ should be $O(1)$. Figure \ref{lai_fig} presents the histogram of calculated $\sigma$s with outliers (lowest and highest 5\% percent) and nonphysical values ($\sigma$ $<$ 0.) removed. All remaining $\sigma$ values are $O(1)$ which provides confidence in model framework.

\begin{figure}[h]
\centering
\includegraphics[width=20pc]{./fig02.png}
\caption{Histogram of $\sigma$ values calculated for each site and time according to Equation \ref{lai}. The lowest and highest 5\% are removed as outliers, as well as any values below 0. The curve is normalized such that its area is 1. }
\label{lai_fig}
\end{figure}

An additional concern is that the $\sigma$ term may in fact be some function of $VPD$, in which case the dependence would need to be accounted for when taking the derivative. Figure \ref{lai_vpd_fig} plots the joint distribution of $\sigma$ and VPD, and shows that $\sigma$ is very weakly a function of VPD. Given this weak dependence, we argue that Equation \ref{d_et} is a valid approximation for ET response to $VPD$.

% below show histogram of site specific spearmannr? think would be useful 
\begin{figure}[h]
\centering
\includegraphics[width=20pc]{./fig03.png}
\caption{The joint distribution of $VPD$ and $\sigma$. $\sigma$ has only a weak dependence on $VPD$. ***This plot could proabably benefit from a box plot of site specific correlations, because some sites do have stronger depednence than others. Note also Figs 3 and 2 can probably be combined because this figure shoes $\sigma$'s histogram.***}
\label{lai_vpd_fig}
\end{figure}

Before calculating the sensitivity of ET to VPD, it is useful to consider the functional form of Equation \ref{d_et}. There are three terms: a scaling term for the full expression we will call Term 1 ($\frac{g_a \; P}{T(\Delta + \gamma)}$), a relatively constant offset we will call Term 2 ($\frac{c_p}{R_{air}}$), and a variable term we will call Term 3 ($\frac{\gamma c_s }{1.6 \; R\; \text{ uWUE }} \left( \frac{2 g_1 + \sqrt{VPD}}{2 (g_1 + \sqrt{VPD})^2}\right)$). All variables are positive, so the relative magnitude between Term 2 and Term 3 will determine the sign of the derivative, while Term 1 will scale the expression larger or smaller.


Term 2 minus Term 3's role in determining the sign of the sensitivty of ET to VPD makes it crucuial for answering our question ``When does VPD drive or reduce ET?'' Exploring these terms more, $c_s$ and $\gamma$ variability is relatively less than $\sigma$ and $VPD$ variability, so variability within PFT will be solely determined by $\sigma$ and $VPD$. If we fix uncertainty $\sigma$ at PFT averages, then Term 2 minus Term 3 is just a function of $VPD$. We can further dertmine a $VPD_{crit}$ where $\frac{\partial \; ET}{\partial \; VPD} = 0$:

\begin{linenomath*}
  \begin{equation}
VPD_{crit} = \frac{R_{air}}{4 c_p} \left( \frac{\gamma c_s}{1.6\; R \; \overline{\sigma} uWUE} + \sqrt{\frac{\gamma c_s}{1.6\; R \; \overline{\sigma} uWUE}\left( \frac{\gamma c_s}{1.6\; R \; \overline{\sigma} uWUE} + 8 g_1 \frac{c_p}{R_{air}}\right)} - 4 g_1 \frac{c_p}{R_{air}} \right)
\label{vpd_min_et}
  \end{equation}
\end{linenomath*}

Values of $VPD_{crit}$ as a function of PFT are shown in Table \ref{vpd_crit}. For any values of $VPD$ less than $VPD_{crit}$, $\frac{\partial \; ET}{\partial \; VPD}$ will be negative, and for values of $VPD$ greater than $VPD_{crit}$, $\frac{\partial \; ET}{\partial \; VPD}$ will be positive.


\begin{table}
  \label{vpd_crit}
\caption{Values of $VPD_{crit}$, where $\frac{\partial \; ET}{\partial \; VPD} = 0$, evaluated at PFT average values for $R_{air}$, $\sigma$, $\gamma$, and $c_s$. For reference, these values are also provided. For values of $VPD$ less than $VPD_{crit}$, $\frac{\partial \; ET}{\partial \; VPD}$ will be negative, and for values of $VPD$ greater than $VPD_{crit}$, $\frac{\partial \; ET}{\partial \; VPD}$ will be positive.}
\centering
\begin{tabular}{l c c c c c c}
  \hline
  PFT & $R_{air}$ & $c_s$ (ppm) & $\gamma$ & $\overline{\sigma}$ & $\overline{\sigma} \cdot uWUE$ & \textbf{$VPD_{crit}$ (Pa)} \\
  \hline
  CRO &  288.680920 & 372.567691& 65.351523& 0.684394&       2.602873&  \textbf{133.165438} \\
  CSH &   289.067152& 381.593622& 67.613172& 0.997224&       2.175278& \textbf{4439.564212} \\
  DBF &   288.624437& 377.449849& 63.421812& 0.881061&       2.746393&  \textbf{888.773243} \\
  ENF &  288.183849& 377.676463& 61.559242& 1.217892&       4.015362&  \textbf{978.084845} \\
  GRA &  288.425651& 377.264645& 61.598768& 0.850869&       2.281074& \textbf{1141.630778} \\
\hline
\multicolumn{2}{l}{$^{a}$Footnote text here.}
\end{tabular}
\end{table}

Figure \ref{term3} provides greater description for how (Term 2 - Term 3) varies with $VPD$, as a function of PFT. Equation \ref{d_et_approx} aids interpretation of Figure \ref{term3}. Larger $uWUE\cdot\overline{\sigma}$, and $g1$ shift the leading-order constant term ($\frac{1}{g_1}$) towards smaller values, and (Term 2 - Term 3) towards positive values. $uWUE$ and $g1$ are both water-use effiency type constants. Higher values corresponde to plants that are more willing to spend water on primary production and less evolved to conserve water. Figure \ref{term3} confirms our physical intuition: CROs are the least water conservative so have the smallest constant portion of Term 3, while CSH are the most water conservative and have the largest constant portion of Term 3. For the VPD-dependent terms in Equation \ref{d_et_approx}, differences in $g1$ between PFTs exert a greater influence than difference $uWUE$, as the power of $g1$ increases. Increasing $uWUE$ and $g1$ decreases the VPD-dependence, but $g1$ has the bigger effect due to its increasing powers. ENF ($g1 = 74.31$) has by far the largest VPD dependence of response, while CRO ($g1 = 183.1$) has the smallest VPD dependence.  
 
\begin{figure}[h]
\centering
\includegraphics[width=20pc]{./fig05.pdf}
\caption{Sources of variability for Term 2 - Term 3. Top: Term 2 - Term 3 as a function of VPD, with $\sigma$ held fixed at PFT averages. The observed range of VPD for each PFT is also shown below the x-axis. Line extent corresponds to 5th and 95th percentiles, while stars denote the location of the 25th, 50th, and 75th percentiles.

Bottom: The location of the minima of ET, as a function of VPD and $\sigma$. Lines and stars denote the distribution of VPD and $\sigma$ next to each axis, following the same percentiles as above.}
\label{term3}
\end{figure}

Figure \ref{term3}b shows the location of the minima of ET, as a function of $\sigma$ and $VPD$. For any $\sigma$ or VPD less (more) than these curves, Term 2 - Term 3 will be negative (positive). It is clear that the portion of VPD observations below/above these curves will be a strong function of $\sigma$. However, we can see some general trends. For CSH, $\frac{\partial \; ET}{\partial \; VPD}$ should be negative for the vast majority of observed $\sigma$ and VPD. The fraciton of positive $\frac{\partial \; ET}{\partial \; VPD}$ appears to be more even for ENF, GRA, and DBF, and we might expect a greater frequency of positive $\frac{\partial \; ET}{\partial \; VPD}$ for CRO. 

While the above discussion of the sign of $\frac{\partial \; ET}{\partial \; VPD}$ is important to answer our research question, the magnitude of $\frac{\partial \; ET}{\partial \; VPD}$ will also impact statistics of $\frac{\partial \; ET}{\partial \; VPD}$ and the importance of $VPD$ variablility for $ET$ variability. So we will more closely examine the scaling Term 1: $\frac{P}{T} \propto \rho$, so this should vary little relative to aerodynamic conductance and $\Delta$. $\gamma$ should also be relatively constant, so Term 1 should be primarily a function of aerodynamic conductance and temperature (through the function $\Delta$). This makes sense, as aerodynamic conductance represents how efficently response at the surface is communicated to the atmosphere. As it increases, any plant response will be communicated more strongly to the atmosphere (and vice-versa).

$\Delta$'s presence in the scaling term also matches physical intition. Evaporative cooling will dampen the ability of the atmosphere to take more moisture, because $e_{s}$ decreases with temperature. The decrease in $e_{s}$ is proportional to $\Delta$ ($\delta e_{s} = \Delta \delta T$). So as $\Delta$ increases, you will get a larger damping of ET due to evaporative cooling.  The functional from of $\Delta$ will be the same across PFT, but the temperature range may vary slightly. In contrast, aerodynamic conductance will vary strongly with PFT due to the importance of surface roughness. So most of the differences in scaling between PFT should be in the aerodynamic conductance term. One interesting side note is that the coefficient of variability for both aerodynamic conductance and Term 1 is relatively constant across PFT, suggesting that the influence of aerodynamic conductance on the relative (to the PFT mean) variability of Term 1 is approximately similar across PFT.

Figure \ref{scale_vary}A shows Term 1 normalized by mean aerodynamic conductance (calculated for each plant functional type), and confirms that much of the relative variability of Term 1 is contained in the aerodynamic conductance variability. Generally, $T$ has less of a role. Additionally, the impact of $T$ on the relative variability increases with increasing aerodynamic conductance. \explain{plot every PFT and show they collapse onto the same curve? - might just be too messy though}.

While the relative variability of Term 1 is similar across PFT, the absolute value of Term 1 varies strongly across PFT. Figure \ref{scale_vary}B shows Term 1 evaluated with the mean aerodynamic conductance for each PFT, and at the range of observed temperatures for each PFT. As expected, for the tree PFTs (DBF, ENF) Term 1 is much larger and the temperature dependence is much stronger. Systematic differences in observed temperatures also cause differences in the average magnitude of Term 1. For example, ENF experiences on average colder temperatures and is thus more likely to have a larger scaling term. Additionally, because the variability of aerodynamic conductance increases proportionally to the mean, the spread of Term 1 due to aerodynamic conductance variability will be larger for the tree PFTs, although this is not shown for simplicity. To summarize, the variability of Term 1 within each PFT will look like Figure \ref{scale_vary}A for each PFT, but the scale of the y-axis will increase or decrease according to mean aerodynamic conductance oberved in Figure \ref{scale_vary}B.
 
\begin{figure}[h]
\centering
\includegraphics[width=20pc]{./fig04.pdf}
\caption{Primary sources of variability for Term 1. A) Variability within each PFT: Term 1 normalized by mean $g_a$ for each PFT. B) Variability between each PFT: Term 1 evaluated at mean $g_a$ for each PFT. Temperature range is 5-95th percentile for each PFT. Additionally, stars denote the location of the 25th, 50th, and 75th percentiles.}
\label{scale_vary}
\end{figure}

%%%%% replace below with idea:
\subsection{Bulk statistics of $\frac{\partial \; ET}{\partial \; VPD}$}

Table 3 confirms our expectations for PFT behavior of $\frac{\partial \; ET}{\partial \; VPD}$. For all PFTs except for CRO, average $\frac{\partial \; ET}{\partial \; VPD}$ is less than zero. However, $\frac{\partial \; ET}{\partial \; VPD}$ evaluated at the average of all variables (e.g. $\sigma$, $T$, $c_s$, $VPD$) is only negative for CSH and GRA. And, DBF in addition to CRO experiences $\frac{\partial \; ET}{\partial \; VPD}$ < 0 less than half the time. These observations highlight the effect of the nonlinear function in Figure \ref{term3}: $\frac{\partial \; ET}{\partial \; VPD}$ has a much steeper slope when the function is negative, and is thus more likely to be large.

The units of $\frac{\partial \; ET}{\partial \; VPD}$ make it difficult to interpret if $VPD$ is even a meaningful contributor to ET's variability. To better understand $VPD$'s contribution, we normalize $\frac{\partial \; ET}{\partial \; VPD}$ with $VPD$'s standard deviation to define a (linearized) relative change in ET for variations in $VPD$ . $VPD$'s contribution to ET's variability ranges between 30 - 40 W m$^{-2}$ for all PFTs except for CSH, which is about 100 W m$^{-2}$. Another meaningful comparison is to $\frac{\partial \; ET}{\partial \; R} \cdot std(R)$, as net radiation is generally the driver of ET (cite joe berry here). For all PFTs except for CSH $VPD$ contributes between 30 - 40 \% of $R$'s contribution to variability. For CSH the portion is much larger, about 88 \%. $VPD$'s variability is certiantly a primary contributor to $ET$'s variability.

So far, idealized plots and statistics have illuminated the form of $\frac{\partial \; ET}{\partial \; VPD}$ and how it varies with PFT. Large mean $\sigma$ and uWUE shifts CRO and DBF towards positive $\frac{\partial \; ET}{\partial \; VPD}$. However, the strongly nonlinear function of $\frac{\partial \; ET}{\partial \; VPD}$ at $\frac{\partial \; ET}{\partial \; VPD} < 0$ pushes $\overline{\frac{\partial \; ET}{\partial \; VPD}}$ negative for DBF (it does not do this for CRO because of CRO's high $g1$). ENF's low $g1$ value increases the dependence of $\frac{\partial \; ET}{\partial \; VPD}$ on $VPD$, and makes the function more strongly nonlinear. This has the side effect of pushing $\overline{\frac{\partial \; ET}{\partial \; VPD}}$ negative further than other PFTs for a given fraction $\frac{\partial \; ET}{\partial \; VPD} < 0$ and magnitude $\frac{\partial \; ET}{\partial \; VPD}(\overline{T,\ldots,VPD})$. GRA shows the opposite behavior; a relatively high $g1$  makes the function more linear, decreasing the magnitude of $-\overline{\frac{\partial \; ET}{\partial \; VPD}}$ for a given  [large] fraction $\frac{\partial \; ET}{\partial \; VPD} < 0$ and negative $\frac{\partial \; ET}{\partial \; VPD}(\overline{T,\ldots,VPD})$ (although aerodynamic conductance and Term 1 also probably have a role in this). Finally, low $uWUE$ of CSH pushes to toward by far the lowest values $\frac{\partial \; ET}{\partial \; VPD}$ (Figure \ref{term3}). Variability in $VPD$ accounts for the largest about of $ET$ variability for CSH. For the other PFTs, $VPD$ contributes less to $ET$ variability, but still represents about 15-20 \% of $R$'s contributions to ET variability.

\begin{table}
\caption{Statistics of $\frac{\partial \; ET}{\partial \; VPD}$ as a function of PFT.}
\centering
\begin{tabular}{l c c c c c}
  \hline
PFT & $\overline{\frac{\partial \; ET}{\partial \; VPD}}$ & $\frac{\partial \; ET}{\partial \; VPD}\left(\overline{T, \ldots , VPD}\right)$ & $\frac{\partial \; ET}{\partial \; VPD}\left(\overline{T, \ldots , VPD}\right)*\text{std}(VPD)$ & $\frac{\frac{\partial \; ET}{\partial \; VPD}\left(\overline{T, \ldots , VPD}\right)*\text{std}(VPD)}{ \frac{\partial \; ET}{\partial \; R}\left(\overline{T, \ldots , VPD}\right)*\text{std}(R)}$ & fraction $\frac{\partial \; ET}{\partial \; VPD} < 0.$ \\
  \hline
CRO & 0.000853 & 0.026241 & 18.523659 & 0.203022 & 0.473311\\
CSH & -0.108234 & -0.091526 & 50.861613 & 0.439379 & 0.931660\\
DBF & -0.012727 & 0.013794 & 19.734435 & 0.164241 & 0.461674\\
ENF & -0.034087 & 0.000706 & 16.611852 & 0.148548 & 0.534425\\
GRA & -0.019637 & -0.000921 & 16.798083 & 0.173552 & 0.631735\\
\hline
\multicolumn{2}{l}{$^{a}$Footnote text here.}  

  
\end{tabular}
\end{table}


\subsection{Testing the theory - does the data match the model?}

So far we have developed a theory for $\frac{\partial \; ET}{\partial \; VPD}$'s behavior. In particular, we determined a critical threshold $VPD_{crit}$ for $\frac{\partial \; ET}{\partial \; VPD}$. For $VPD < VPD_{crit}$, $\frac{\partial \; ET}{\partial \; VPD}  < 0$, and for $VPD > VPD_{crit}$, $\frac{\partial \; ET}{\partial \; VPD} > 0$. This $VPD_{crit}$ is only a function of PFT. However, many assumptions, including constant uWUE, were made in deriving $VPD_{crit}$. So, we need to test if our theory and $VPD_{crit}$ hold up in the face of uncertainty.

Figure \ref{real} presents calculated $\frac{\partial \; ET}{\partial \; VPD}$ where, unless otherwise noted, all variables in Equation \ref{d_et} are allowed to vary, including uncertainty. Each column is a different quantity related to $\frac{\partial \; ET}{\partial \; VPD}$, and each row is a different PFT. 

The full observations generally confirm expectations from Section \ref{results}. CRO has the most positive values of $\frac{\partial \; ET}{\partial \; VPD}$, $\frac{\partial \; ET}{\partial \; VPD}$ is almost always negative for CSH, and response depends more with the environmental conditions for the other PFTs (especially ENF). Through the columns of Figure \ref{real} we can see the impact of $\sigma$ and $g_a$ on the variability of $\frac{\partial \; ET}{\partial \; VPD}$. $g_a$'s scaling (included in columns 1 and 3) alters the magnitude considerably. $\sigma$ variability (included in columns 1 and 2) adds a lot of additional noise to the signal of $\frac{\partial \; ET}{\partial \; VPD}$, which is slightly undesirable given $\sigma$'s role in representing model and observational uncertainty. However, the general story with the noise appears to match the cleaner signal when $\sigma$ is help constant and $VPD_{crit}$ is clearly visible. One exception is possibly with GRA, for which uncertainty represented in $\sigma$ is high and causes the full complexity plots (Columns 1 and 2) to not match well with $\sigma$ held fixed (Columns 3 and 4).

For ENF and GRA $VPD_{crit}$ does not appear to be only a function of $\sigma$ (most observable in Column 4). It turns out that the site to site variability in $\gamma$ causes $VPD_{crit}$ to vary, which is not discussed in the previous section. The impact is observable in both ENF and GRA, but especially for ENF which has a larger $\frac{\partial^2 \; ET}{\partial^2 \; VPD}$ than the other PFTs. 

In general the full complexity plots of $\frac{\partial \; ET}{\partial \; VPD}$ match our expectations, even with the large sensitivity to $\sigma$ measures of uncertainty observed in Figure \ref{term3}. Our $\sigma$-based method of uncertainty propagation blurs the idealized expectations the most for GRA, and also has a considerable effect for CRO. We therefor have the most confidence in our conclusion based on Equation \ref{d_et} for PFTS CSH, DBF, and ENF, as the full complexity plots with uncertainty included closely match the story when $\sigma$ is held fixed. **see somewhat preferred alternate figure \ref{real2} \explain{I think I like the alternate plot much  more as thinking in terms of T and RH is easier, and it makes the story easier to see at relatively low temperatures. However, I used the other plot because Fig 5 does not discuss things in terms of temperature, as this would make things more complicated (adding another dimension). I could just include both versions of figure 6 though (using figure 6a as a bridge to figure 6b)}.
\begin{figure}[h]
\centering
\includegraphics[width=\textwidth]{./fig06.png}
\caption{Scatter plots of $\frac{\partial \; ET}{\partial \; VPD}$. Each row is a different PFT, and each column is a different quantity related to $\frac{\partial \; ET}{\partial \; VPD}$, as labeled: Column 1 - $\frac{\partial \; ET}{\partial \; VPD}$; Column 2 - $\frac{\partial \; ET}{\partial \; VPD}$ normalized by $g_a$; Column 3 - $\frac{\partial \; ET}{\partial \; VPD}$ with $\sigma$ held fixed at PFT average; and Column 4 - $\frac{\partial \; ET}{\partial \; VPD}$ normalized by $g_a$ and with $\sigma$ held fixed. For reference, lines corresponding to RH = 20\% and RH = 90 \% are drawn. Please note differences in the colorbar scale. ***see alternate (or additional) plot below.***}
\label{real}
\end{figure}

\begin{figure}[h]
\centering
\includegraphics[width=\textwidth]{./fig06b.png}
\caption{****alternate Fig 06****  Scatter plots of $\frac{\partial \; ET}{\partial \; VPD}$. Each row is a different PFT, and each column is a different quantity related to $\frac{\partial \; ET}{\partial \; VPD}$, as labeled. If I end up using this, I could also draw on the curve of $VPD_{crit}$ with $\overline{\frac{\text{LAI}}{\text{LAI$_{ref}$}}}$. }
\label{real2}
\end{figure}


\section{Conclusions} \explain{also need to flesh this section out}

The idealized representation of ET used here is successful in developing intuition for how ET responds to changes in $VPD$. This intuition will aid the community in interpreting observations and output from sophisticated full complexity climate models.

The idealized framework leads to the following general conclusions:
\begin{itemize}
  \item Aerodynamic resistance plays an important role of scaling $\frac{\partial \; ET}{\partial \; VPD}$. This is a leading order effect for observing higher magnitude responses in DBF and ENF.
  \item In general, CSH has the most negative (i.e. ET reduced) response to increases in $VPD$ (atmospheric drying). So CSH plants will almost always try and conserve water, effectively reducing ET with dry atmospheric perturbation.
    \item Additionally for CSH, $VPD$ variability contributes the most to $ET$ variability.
\item CRO has the most positive response (i.e. ET increased) in response to increases in $VPD$. This is consistent with CROs that may be evolved or bred to thrive in non-water-limited environments.
\item The response is more a function of the environment for DBF, ENF, and GRA. Because as VPD increases the response is more likely to be positive, if RH is fixed then the response will be more likely to be positive at warmer T, or if T is fixed the response is more likely to be positive with decreasing RH.
\item ENF has the strongest dependence on environmental conditions due to its small $g1$.
\item Model and observational uncertainty is highest for GRA and CRO, so conclusions about those PFTs should be tempered.
\item However, inclusion of uncertainty doesn't alter conclusions about DBF, ENF, and CSH.
\end{itemize}

The intuition developed using this framework can be used to understand how the land surface will respond and contribute to changes in the environment. Additionally, Equation \ref{et} gives provides an estimate of ET that requires no additional information beyond that required to calculate PET. Given that for all PFTs, with the exception of CRO, we found a high frequency and magnitude of negative $\frac{\partial \; ET}{\partial \; VPD}$, PET is a physically unrealisitic representation of ET for vegetated surfaces and PET-based drought metrics are not usefull. We advocate for drought metrics using Equation \ref{et} instead of PET. 


  

%%
%% Enter Figures and Tables near as possible to where they are first mentioned:
%
% DO NOT USE \psfrag or \subfigure commands.
%
% Figure captions go below the figure.
% Table titles go above tables;  other caption information
%  should be placed in last line of the table, using
% \multicolumn2l{$^a$ This is a table note.}
%
%----------------
% EXAMPLE FIGURE
%
% \begin{figure}[h]
% \centering
% when using pdflatex, use pdf file:
% \includegraphics[width=20pc]{figsamp.pdf}
%
% when using dvips, use .eps file:
% \includegraphics[width=20pc]{figsamp.eps}
%
% \caption{Short caption}
% \label{figone}
%  \end{figure}
%
% ---------------
% EXAMPLE TABLE
%
% \begin{table}
% \caption{Time of the Transition Between Phase 1 and Phase 2$^{a}$}
% \centering
% \begin{tabular}{l c}
% \hline
%  Run  & Time (min)  \\
% \hline
%   $l1$  & 260   \\
%   $l2$  & 300   \\
%   $l3$  & 340   \\
%   $h1$  & 270   \\
%   $h2$  & 250   \\
%   $h3$  & 380   \\
%   $r1$  & 370   \\
%   $r2$  & 390   \\
% \hline
% \multicolumn{2}{l}{$^{a}$Footnote text here.}
% \end{tabular}
% \end{table}

%% SIDEWAYS FIGURE and TABLE 
% AGU prefers the use of {sidewaystable} over {landscapetable} as it causes fewer problems.
%
% \begin{sidewaysfigure}
% \includegraphics[width=20pc]{figsamp}
% \caption{caption here}
% \label{newfig}
% \end{sidewaysfigure}
% 
%  \begin{sidewaystable}
%  \caption{Caption here}
% \label{tab:signif_gap_clos}
%  \begin{tabular}{ccc}
% one&two&three\\
% four&five&six
%  \end{tabular}
%  \end{sidewaystable}

%% If using numbered lines, please surround equations with \begin{linenomath*}...\end{linenomath*}
%\begin{linenomath*}
%\begin{equation}
%y|{f} \sim g(m, \sigma),
%\end{equation}
%\end{linenomath*}

%%% End of body of article

%%%%%%%%%%%%%%%%%%%%%%%%%%%%%%%%
%% Optional Appendix goes here
%
% The \appendix command resets counters and redefines section heads
%
% After typing \appendix
%
%\section{Here Is Appendix Title}
% will show
% A: Here Is Appendix Title
%
%\appendix
%\section{Here is a sample appendix}

%%%%%%%%%%%%%%%%%%%%%%%%%%%%%%%%%%%%%%%%%%%%%%%%%%%%%%%%%%%%%%%%
%
% Optional Glossary, Notation or Acronym section goes here:
%
%%%%%%%%%%%%%%  
% Glossary is only allowed in Reviews of Geophysics
%  \begin{glossary}
%  \term{Term}
%   Term Definition here
%  \term{Term}
%   Term Definition here
%  \term{Term}
%   Term Definition here
%  \end{glossary}

%
%%%%%%%%%%%%%%
% Acronyms
%   \begin{acronyms}
%   \acro{Acronym}
%   Definition here
%   \acro{EMOS}
%   Ensemble model output statistics 
%   \acro{ECMWF}
%   Centre for Medium-Range Weather Forecasts
%   \end{acronyms}

%
%%%%%%%%%%%%%%
% Notation 
%   \begin{notation}
%   \notation{$a+b$} Notation Definition here
%   \notation{$e=mc^2$} 
%   Equation in German-born physicist Albert Einstein's theory of special
%  relativity that showed that the increased relativistic mass ($m$) of a
%  body comes from the energy of motion of the body—that is, its kinetic
%  energy ($E$)—divided by the speed of light squared ($c^2$).
%   \end{notation}




%%%%%%%%%%%%%%%%%%%%%%%%%%%%%%%%%%%%%%%%%%%%%%%%%%%%%%%%%%%%%%%%
%
%  ACKNOWLEDGMENTS
%
% The acknowledgments must list:
%
% •	All funding sources related to this work from all authors
%
% •	Any real or perceived financial conflicts of interests for any
%	author
%
% •	Other affiliations for any author that may be perceived as
% 	having a conflict of interest with respect to the results of this
% 	paper.
%
% •	A statement that indicates to the reader where the data
% 	supporting the conclusions can be obtained (for example, in the
% 	references, tables, supporting information, and other databases).
%
% It is also the appropriate place to thank colleagues and other contributors. 
% AGU does not normally allow dedications.


\acknowledgments
This work used eddy covariance data acquired and shared by the FLUXNET community, including these networks: AmeriFlux, AfriFlux, AsiaFlux, CarboAfrica, CarboEuropeIP, CarboItaly, CarboMont, ChinaFlux, Fluxnet-Canada, GreenGrass, ICOS, KoFlux, LBA, NECC, OzFlux-TERN, TCOS-Siberia, and USCCC. The ERA-Interim reanalysis data are provided by ECMWF and processed by LSCE. The FLUXNET eddy covariance data processing and harmonization was carried out by the European Fluxes Database Cluster, AmeriFlux Management Project, and Fluxdata project of FLUXNET, with the support of CDIAC and ICOS Ecosystem Thematic Center, and the OzFlux, ChinaFlux and AsiaFlux offices.


%% ------------------------------------------------------------------------ %%
%% Citations

% Please use ONLY \citet and \citep for reference citations.
% DO NOT use other cite commands (e.g., \cite, \citeyear, \nocite, \citealp, etc.).


%% Example \citet and \citep:
%  ...as shown by \citet{Boug10}, \citet{Buiz07}, \citet{Fra10},
%  \citet{Ghel00}, and \citet{Leit74}. 

%  ...as shown by \citep{Boug10}, \citep{Buiz07}, \citep{Fra10},
%  \citep{Ghel00, Leit74}. 

%  ...has been shown \citep [e.g.,][]{Boug10,Buiz07,Fra10}.



%%  REFERENCE LIST AND TEXT CITATIONS
%
% Either type in your references using
%
% \begin{thebibliography}{}
% \bibitem[{\textit{Kobayashi et~al.}}(2003)]{R2013} Kobayashi, T.,
% Tran, A.~H., Nishijo, H., Ono, T., and Matsumoto, G.  (2003).
% Contribution of hippocampal place cell activity to learning and
% formation of goal-directed navigation in rats. \textit{Neuroscience}
% 117, 1025--1035.
%
% \bibitem{}
% Text
% \end{thebibliography}
%
%%%%%%%%%%%%%%%%%%%%%%%%%%%%%%%%%%%%%%%%%%%%%%%
% Or, to use BibTeX:
%
% Follow these steps
%
% 1. Type in \bibliography{<name of your .bib file>} 
%    Run LaTeX on your LaTeX file.
%
% 2. Run BiBTeX on your LaTeX file.
%
% 3. Open the new .bbl file containing the reference list and
%   copy all the contents into your LaTeX file here.
%
% 4. Run LaTeX on your new file which will produce the citations.
%
% AGU does not want a .bib or a .bbl file. Please copy in the contents of your .bbl file here.
% \bibliography{references.bib}
% 11/23/2015
\documentclass[draft,linenumbers]{agujournal}
% \draftfalse
\usepackage{makecell}
\usepackage{multirow} %AM for tables
\usepackage{booktabs}
\drafttrue

\journalname{Agricultural and Forest Meteorology: When does VPD drive or reduce ET?}

\newcommand{\appropto}{\mathrel{\vcenter{
  \offinterlineskip\halign{\hfil$##$\cr
    \propto\cr\noalign{\kern2pt}\sim\cr\noalign{\kern-2pt}}}}}

\begin{document}

\title{When does vapor pressure deficit drive or reduce evapotranspiration?}


\authors{Adam Massmann\affil{a}, Pierre Gentine\affil{a}, Changjie Lin\affil{a,b}}

\affiliation{a}{Department of Earth and Environmental Engineering, Columbia University, New York, NY 10027}
\affiliation{b}{State Key Laboratory of Hydroscience and Engineering, Department of Hydraulic Engineering, Tsinghua University, Beijing, CN 100084}

\correspondingauthor{Adam Massmann}{206-919-1364; akm2203@columbia.edu}

\begin{keypoints}
\item ecohydrology, evapotranspiration, aridity, vapor pressure deficit, Clausius-Clapeyron relation, plant physiology, stomatal response
\end{keypoints}


\begin{abstract}
  Increases to vapor pressure deficit (VPD) and atmospheric demand for water are expected with rising atmospheric CO$_2$. While increased evapotranspiration (ET) in response to increased atmospheric demand seems intuitive, plants are capable of reducing ET in response to increased VPD by closing their stomata, in an effort to conserve water. Here we examine which effect dominates response to increasing VPD: atmospheric demand and increases in ET, or plant physiological response (stomata closure) and decreases in ET. We use Penman-Monteith, combined with semi-empirical optimal stomatal regulation theory and underlying water use efficiency, to develop a theoretical framework for understanding how ET responds to increases in VPD.
  The theory suggests that for most environmental conditions and plant types, plant physiological response dominates and ET decreases with increasing VPD. Plants that are evolved or bred to prioritize primary production over water conservation (e.g. crops) exhibit a higher likelihood of atmospheric demand-driven response (ET increasing). However for forest, grass, and shrub plant types, ET more frequently decreases than increases with rising VPD. Therefore, we may expect reductions in ET due to CO$_2$-induced increases in VPD. This work serves as an example of the utility of our simplified framework for disentangling land-atmosphere feedbacks at multiple scales, including the characterization of ET response in an atmospherically drier, enriched CO$_2$ world. 
 
\end{abstract}

\section{Introduction}

Vapor pressure deficit (VPD) is expected to rise over continents in the future due to the combination of increased temperature and, depending on region, decreased relative humidity \citep{Byrne_2013}. Increases in VPD increase the atmospheric demand for evapotranspirated water \citep{Penman_1948, Monteith_1965}, but also stress plant stomata \citep{Leuning_1990, MEDLYN_2011}.

The opposing effects of increased atmospheric demand and higher stomatal stress lead to two possible perspectives for how evapotranspiration (ET) responds to shifts in VPD. The first, a hydrometeorological perspective, is that higher VPD increases atmospheric demand for water from the land surface, and this drives an increase in evapotranspiration (ET). This perspective is particularly relevant because potential evapotranspiration (PET), which is used in many drought indices and hydrometeorological studies \citep[e.g.,][]{Heim_2002, Scheff_2015}, typically only quantifies changes in atmospheric demand and fails to account for ecosystem response \citep{Swann_2016}. In reality, plants' stomata have evolved to optimally regulate the exchange of water and carbon, and tend to partially close in response to increased atmospheric dryness \citep{Farquhar_1978, Ball_1987, Leuning_1990, MEDLYN_2011}. This leads to a plant physiology perspective, in which an increase in VPD, particularly in well-watered soil conditions, may actually correspond to a decrease in ET because of stomatal closure.  In other words, the  question ``When does VPD drive or reduce ET?'' can be related to whether plant regulation or atmospheric demand dominates ET response.

The ET response to changes in VPD alters water partitioning between the soil and atmosphere. If plant response reduces ET with atmospheric drying then soil moisture will be better conserved. This would seem a sensible evolutionary strategy to cope with aridity. If stomata were fully passive \citep [similar to soil pores, e.g. ][]{Or_2013}, increased atmospheric aridity would strongly reduce soil moisture \citep{Berg_2017}. In turn, this would further increase aridity as low soil moisture levels increase the Bowen ratio, and cause increased temperature and atmospheric drying \citep[][]{Bouchet_1963, Morton_1965, Brutsaert_1999, Ozdogan_2006, Salvucci_2013, Gentine_2016, Berg_2016}. This however would not seem to be a sensible strategy for plants from an evolutionary standpoint.

We can use intuition about plant water conservation strategy to hypothesize about ET response to changes in VPD. Plants that evolved to conserve water, such as arid shrubs, should be more likely to reduce ET with increasing VPD, and plants that have evolved or have been engineered to care little about water, such as crops, will be more likely to increase ET with increasing VPD. Atmospheric conditions must matter as well. At the ecosystem scale, there are limits to plant water conservation strategies. As atmospheric demand for water (VPD) increases, ecosystems should begin to reach their water conservation limits and might not be able to entirely limit ET flux to the atmosphere. At this stage any further increase in VPD will most likely drive a (limited) increase in ET, because the increase in atmospheric demand for water overwhelms the limited plant response to conserve water.

The objective of the present manuscript is to evaluate the VPD dependence of ET, in non-extreme soil drought conditions. The goal of this paper is to use reasonable approximations as a tool to develop intuition for plant response to atmospheric drying. This intuition will aid interpretation of observations and full complexity models. In the past, similiar approaches were used to understand interactions between stomatal conductance, evapotranspiration and the environment \citep[e.g.,][]{Jarvis_1986, Mcnaughton_1991}. However, at the time researchers' understandng of the form of VPD's effect on plant physiology was limited, so they could not explore the sensitivity of ET to VPD, including VPD's effect on stomatal conductance and plant function.

Recent results have drastically improved our understanding of VPD's impact on physiology. \citet{MEDLYN_2011} developed a model for leaf-scale stomatal conductance ($g_s$), including VPD response, by combining an optimal photosynthesis theory \citep{Farquhar_1980, Katul_2010} with an empirical approach, and extended this model to the ecosystem scale in \citet{Medlyn_2017}. Additionally, \citet{Zhou_2014} demonstrated that a new quantity $uWUE$ ($\frac{GPP\; \sqrt{vpd}}{ET}$) properly captures a constant relationship between GPP, ET, and VPD over a diurnal cycle. uWUE is also remarkably well conserved in the growing season across space and time, within a PFT \citep{Zhou_2015}. These results provide novell tools to examine how ET varies in response to VPD, while explicitly including VPD's effect on stoamtal conductance for the first time. In this manuscript, we derive the theoretical response of ET to VPD using a Penman Monteith framework. Our theory is validated and tested at multiple eddy-covariance stations spanning various climates and plant functional types. 


\section{Materials and Methods}
\label{methods}
\subsection{Data}
\label{data}
We use both meteorological and eddy-covariance data from the FLUXNET2015 database, including all sites with at least four years of data (data available at \sloppy https://fluxnet.fluxdata.org/data/fluxnet2015-dataset/ \sloppy). Each site's plant functional type (PFT) was classified using the International Geosphere-Biosphere Programme vegetation classification scheme \citep{Loveland_1999}. The physical constants used in the stomatal conductance model in the Methods section (Section \ref{methods}) are only published for six plant functional types (PFTs): crops (CRO), deciduous broadleaf forest (DBF), evergreen broadleaf forests (EBF), evergreen needleleaf forest (ENF), grass (GRA), and closed shrub (CSH) (see Table \ref{pft}). There are 62 sites with these plant functional types, and their location is shown in  Figure \ref{map_fig}.

\begin{table}
  \caption{Plant functional types, their abbreviation, Medlyn coefficient (from \citet{Lin_2015}, via \citet{Franks_2017}), and calculated uWUE. uWUE from \citet{Zhou_2015}, including the  observed standard deviation is shown for comparison. Note that uWUE from \citet{Zhou_2015} is calculated from a different set of sites, and that units are converted such that the quantities work with Equations 1-8 with the variables defined Table \ref{definitions}.}
    \small
\label{pft}
\centering
\begin{tabular}{l c c @{\qquad} c c}
  \hline
  \multirow{2}[3]{*}{Abbreviation} & \multirow{2}[3]{*}{PFT} & \multirow{2}[3]{*}{$g_1$ (Pa$^{0.5}$)} & \multicolumn{2}{c}{uWUE ($\mu$-mol [C] Pa$^{0.5}$ J$^{-1}$ [ET])}  \\
  \cmidrule{4-5}

  & & & fitted & \citet{Zhou_2015} \\

  \hline
  DBF & deciduous broadleaf forest & 140.7 & 2.63 & 3.12 $\pm$ 0.52 \\
  EBF & evergreen broadleaf forest & 130.3 & 2.56 &  N/A \\
  ENF & evergreen needleleaf forest & 74.3 & 3.45 & 3.30 $\pm$ 0.91 \\
  CRO & cropland & 183.1 & 2.27 & 3.80 $\pm$ 1.01 \\
  CSH & closed shrub & 148.6 & 1.66 & 2.18 $\pm$ 0.44 \\
  GRA & grassland (assumed C3) & 166.0 & 1.92 & 2.68 $\pm$ 0.61 \\
\hline
\end{tabular}
\end{table}


\begin{figure}
\centering
\includegraphics[width=\textwidth]{./map.pdf}
\caption{Plant functional type and location of FLUXNET2015 sites used in this analysis.}
\label{map_fig}
 \end{figure}


The purpose of this study is to examine ecosystem response to atmospheric drying, focusing on the growing season. To accomplish this, we filter and quality control the data using a similar procedure as \cite{Zhou_2015}:
\begin{itemize}
\item Only measured or highest (``good'') quality gapfilled data, according to quality control flags, are used.
\item To isolate the growing season, we only use days in which the average Gross Primary Productivity (GPP) exceeds 10\% of the observed 95th percentile of GPP for a given site. GPP is calculated using the nighttime respiration partitioning method.
\item We remove days with rain and the day following to avoid issues with rain interception and sensor saturation at high relative humidity (\cite{MEDLYN_2011}).
\end{itemize}
Additionally, as in \citet{Lin_2018}, we restrict data to the daytime, which is identified when downwelling shortwave radiation is greater than 50 W m$^{-2}$ and sensible heat flux is greater than 5 W m$^{-2}$. To reduce the chance of sensor saturation at high relative humidity, we remove all time steps for which VPD is less than .01 kPa, and to reduce errors at low windspeeds we remove all periods with wind magnitudes less than 0.5 m s$^{-1}$. Timesteps with negative observed GPP or ET are also removed, and we aggregate half hourly data to hourly averages to reduce noise \citep{Lin_2018}. After these quality control procedures, 335,939 upscaled hourly observations remain. 

\subsection{Methods}
\label{methods}
The Penman-Monteith equation \citep [hereafter PM,][]{Penman_1948, Monteith_1965} estimates ET as a function of observable atmospheric variables and surface conductances:
\begin{linenomath*}
  \begin{equation}
    \label{orig_pen}
    ET = \frac{\Delta R_{net} + g_a \rho_a c_p VPD}{\Delta + \gamma(1 + \frac{g_a}{g_s})},
  \end{equation}
\end{linenomath*}
where $\Delta$ is the change in saturation vapor pressure with temperature, given by Clausius-Clapeyron ($\frac{d \; e_s}{d \; T}$), $R_{net}$ is the net radiation minus ground heat flux, $g_a$ is aerodynamic conductance, $\rho_a$ is air density, $c_p$ is specific heat of air at constant pressure, $\gamma$ is the psychometric constant, and $g_s$ is the stomatal conductance (Table \ref{definitions}).

\begin{table}
  \caption{Definition of symbols and variables, with citation for how values are calculated, if applicable.}
  \label{definitions}
\centering
\small
\begin{tabular}{l c c c}
\hline
 Variable & Description & Units & Citation \\
\hline
$e_s$  & saturation vapor pressure & Pa  & - \\ 
  $T$  & temperature  & K & - \\
  $P$  & pressure & Pa  & - \\
$\Delta$  & $\frac{\partial e_s}{\partial T}$ & Pa K$^{-1}$ & - \\
$R_{net}$  & net radiation at land surface minus ground heat flux & W m$^{-2}$   & - \\
  $g_a$  & aerodynamic conductance & m s$^{-1}$  & \citet{Shuttleworth_2012} \\
  $\rho_a$  & air density & kg m$^{-3}$  & - \\
  $c_p$  & specific heat capacity of air at constant pressure & J K$^{-1}$ kg$^{-1}$ & - \\
  $VPD$  & vapor pressure deficit & Pa  & - \\
  $\gamma$  & psychometric constant & Pa K$^{-1}$   & - \\
  $g_s$  & stomatal conductance & m s$^{-1}$  & \citet{MEDLYN_2011} \\
  $g_{l-s}$  & leaf-scale stomatal conductance & mol m$^{-2}$ s$^{-1}$  & \citet{MEDLYN_2011} \\
  $R$ & universal gas constant & J mol$^{-1}$ K$^{-1}$ & - \\
  $R_{air}$ & gas constant of air & J  K$^{-1}$ kg$^{-1}$ & - \\
  $LAI$ & leaf area index & -& - \\
  $\sigma$ & uncertainty parameter & -& - \\
  $c_a$ & CO$_2$ concentration & $\mu$ mol CO$_2$ mol$^{-1}$ air& - \\
  $\lambda$ & marginal water cost of leaf carbon & mol H$_2$O mol$^{-1}$ CO$_2$ & - \\
  $\Gamma$ & CO$_2$ compensation point & - & - \\
  $\Gamma^*$ & CO$_2$ compensation point without dark respiration & - & - \\
\hline
\end{tabular}
\end{table}


\citet{MEDLYN_2011} developed a model for stomatal conductance ($g_s$) by combining an optimal photosynthesis theory \citep{Farquhar_1980, Katul_2010} with an empirical approach, which describes the dependence of $g_s$ to VPD. This resulted in the following model for leaf-scale stomatal conductance:

\begin{linenomath*}
  \begin{equation}
  g_{l-s} = g_0 + 1.6 \left(1 + \frac{g_1}{\sqrt{VPD}}\right) \frac{A}{c_a}
  \end{equation}
\end{linenomath*}
Where $g_1$ is a leaf-scale ``slope'' parameter, A is the net CO$_2$ assimilation rate, and $c_a$ is the atmospheric CO$_2$ concentration at the leaf surface. \cite{MEDLYN_2011} relate the slope parameter ($g_1$) to physical parameters as:
\begin{linenomath*}
  \label{slope}
  \begin{equation}
  g_1 = \sqrt{\frac{3 \, \Gamma^* \, \lambda}{1.6}},\footnote{Note this expression has units of of (mmol mol$^{-1}$)$^{1/2}$, but this can be converted to Pa$^{1/2}$ using the ideal gas law.}
  \end{equation}
\end{linenomath*}

where $\Gamma^*$ is the CO$_2$ compensation point for photosynthesis (without dark respiration), and $\lambda$ is the marginal water cost of leaf carbon ($\frac{\partial \; \text{transpiration}}{\partial \; A}$). So, $g_1$ is a leaf-scale term reflecting the tradeoff of water for carbon uptake and growth. The higher $g_1$, the more open the stomata and the more they release water in exchange for carbon.

The Medlyn model for stomatal conductance has been shown to behave very well across PFTs \citep[][]{Lin_2015}, and can be  adapted to the ecosystem scale by multiplying $g_{l-s}$ by leaf area index (LAI) and converting units to m s$^{-1}$ with the ideal gas law:

\begin{linenomath*}
  \begin{equation}
    g_s = \text{LAI} \frac{R \,T}{P} \left( g_0 + 1.6 \left(1 + \frac{g_1}{\sqrt{VPD}}\right) \frac{A}{c_a}\right)
      \label{medlyn}
  \end{equation}
\end{linenomath*}

While Equation 3 can be used in PM (Equation 1), it will make analytical work with the function intractable because $A$, net CO$_2$ assimilation rate, is functionally related to ET itself. To remove the dependence of ET on $A$ we use fundamental semi-empirical results of \citet{Zhou_2015}, who showed that the underlying Water Use Efficiency (uWUE):

\begin{linenomath*}
  \begin{equation}
    uWUE = \frac{GPP \cdot \sqrt{VPD}}{ET}
    \label{uwue}
  \end{equation}
\end{linenomath*}
is relatively constant across time and moisture conditions within a plant functional type. If, following \citet{Lin_2015}, we approximate $g_0$ as $0$ (i.e. we neglect cuticular and epidermal losses - a reasonable assumption except in very dry conditions), we can use uWUE to remove the $A$ dependence from $g_s$ in a way that makes the PM equation analytically tractable:

\begin{linenomath*}
  \begin{equation}
    g_s = \frac{R \, T}{P} 1.6 \left(1 + \frac{g_1}{\sqrt{VPD}}\right) \frac{uWUE \; ET}{c_a \; \sqrt{VPD}}.
    \label{new_g_s}
  \end{equation}
\end{linenomath*}

Note that uWUE is fit on the ecosystem scale as in \citet{Zhou_2015}, so GPP in Equation \ref{uwue} is really $A \cdot \text{LAI}$. This leads to the cancellation of LAI in the step from Equation \ref{medlyn} to Equation \ref{new_g_s}. Plugging Equation \ref{new_g_s} into Equation \ref{orig_pen} and rearranging gives a new explicit expression for PM, in which dependence on $A$ is removed:

\begin{linenomath*}
  \begin{equation}
    ET = \frac{\Delta R_{net} + \frac{g_a\; P}{T} \left( \frac{ c_p VPD}{R_{air}} -  \frac{\gamma c_a \sqrt{VPD} }{ R \; 1.6\; \text{ uWUE } (1 + \frac{g_1}{\sqrt{VPD}})} \right) }{ \Delta + \gamma}
    \label{et}
  \end{equation}
\end{linenomath*}

With Equation \ref{et}, ET is explicitly a function of environmental variables and two plant-specific constants, the slope parameter ($g_1$), and uWUE, both reflecting water conservation strategy. The slope parameter is a leaf-scale term related to the willingness of stomata to trade water for CO$_2$ and to keep stomata open. uWUE is a semi-empirical ecosystem-scale constant related to how WUE changes with VPD (specifically $VPD^{-1/2}$). Is is also roughly proportional to physical constants:

\[uWUE \appropto \sqrt{\frac{c_a - \Gamma}{1.6 \lambda}},\]

where $\Gamma$ is the CO$_2$ compensation point \citep[Equation 5 in][]{Zhou_2014}. So uWUE is related to atmospheric CO$_2$ concentration and compensation point, and is inversely proportional to the marginal water cost of leaf carbon. 

Given eddy-covariance FLUXNET2015 data (Section \ref{data}), every term in our new version of PM (Equation \ref{et}) is observed, except for the two parameters $g_1$, and uWUE. $g_1$ has been measured and reported for the PFTs considered here \citep{Lin_2015, Franks_2017}, and we fit uWUE by calculating its expectation, given the model and FLUXNET2015 data.

However, eddy-covariance data are inherently noisy so we include a measure of uncertainty in our analysis. To account for observational error, as well as model uncertainty (e.g. temporal and spatial variations of uWUE), we introduce an uncertainty parameter $\sigma$ modifying uWUE:

\begin{linenomath*}
  \begin{equation}
    ET = \frac{\Delta R_{net} + \frac{g_a\; P}{T} \left( \frac{ c_p VPD}{R_{air}} -  \frac{\gamma c_a \sqrt{VPD} }{ R \; 1.6\; \sigma \; \text{ uWUE } (1 + \frac{g_1}{\sqrt{VPD}})} \right) }{ \Delta + \gamma}
    \label{et_sigma}
  \end{equation}
\end{linenomath*}

Now, from each FLUXNET2015 observation (i.e. for each hourly observation at every time step) we can evaluate $\sigma$:

\begin{linenomath*}
  \begin{equation}
\sigma = - \frac{g_a \gamma c_a \sqrt{VPD} L_v P }{ \left(\text{ ET } ( \Delta + \gamma) - \Delta R_{net} - g_a \rho_a c_p VPD\right) 1.6 \; R\; T\; \text{ uWUE } (1 + \frac{g_1}{\sqrt{VPD}})},
    \label{sigma}
  \end{equation}
\end{linenomath*}

Thus, with this uncertainty analysis we can evaluate departure from our theory in observations, as a departure of $\sigma$ from unity. The variability of $\sigma$ across sites and time provides a measure of uncertainty in our model, assumptions, as well as in the FLUXNET2015 observations themselves. The variability of $\sigma$ then propagates through any uncertainty to our derivative of Equation \ref{et_sigma}:

\begin{linenomath*}
  \begin{equation}
    \frac{\partial \;  ET}{\partial \; VPD} = \frac{2\; g_a \; P}{T(\Delta + \gamma)}   \left(\frac{ c_p}{R_{air}} -  \frac{\gamma c_a }{1.6 \; R\; \sigma \; \text{ uWUE }} \left( \frac{2 g_1 + \sqrt{VPD}}{2 (g_1 + \sqrt{VPD})^2}\right) \right)
    \label{d_et}
  \end{equation}
\end{linenomath*}

With Equation \ref{d_et} we have an analytical framework for ecosystem response to atmospheric demand perturbations, with an implicit assumption of non-extreme drought conditions. There are a few subtleties to taking the derivative in Equation \ref{d_et}: $\Delta$ ($\frac{d e_{s}}{d T}$) and $VPD$ are functionally related, so while taking the derivative we evaluate $\frac{\partial \; ET}{\partial \; VPD} = \frac{\partial \; ET} {\partial \; e_s} \frac{\partial \; e_s}{\partial \; VPD} \Big|_{\text{RH fixed}} + \frac{\partial \; ET}{\partial \; RH} \frac{\partial \; RH}{\partial \; VPD} \Big|_{\text{$e_s$ fixed}}$. $RH$ and $e_s$ are assumed to be approximately independent, which is supported by the data (not shown). 

We note one final comment on our derivation which is relevant for drought indices. If we approximate $c_a$ at a global mean CO$_2$ concentration, then the RHS of Equation \ref{et} is fully defined using commonly available weather station data and the constants published in \citet{Zhou_2015} and \citet{Lin_2015}. This makes Equation \ref{et} a useful alternative to PET in drought indices and hydrometeorological analysis for vegetated surfaces. Equation \ref{et} better reflects the physics of water exchange at the land surface and would only require fitting of uWUE and $g_1$, and can also account for changes in CO$_2$ concentration, contrary to typical drought indices such as the Palmer Drought Severity Index (PDSI) \citep{Swann_2016}.




\section{Results}
\label{results}

By construction, the variability in the $\sigma$ term (Equation \ref{sigma}) contains all model and observational uncertainties. For an observation that perfectly matches our model and constant uWUE assumption, $\sigma$ will be one. Therefore, for our assumptions and framework to be reasonable $\sigma$ should be close to 1. An additional concern is that $\sigma$ may in fact be correlated with $VPD$, in which case the dependence would need to be accounted for when taking the derivative. Fortunately, there is a very weak dependence of $\sigma$ on VPD in their joint distrubution, and $\sigma$ is indeed close to unity i.e. $O(1)$ (Figure \ref{joint_vpd_sigma}). Given this weak dependence and the distribution of $\sigma$ we have confidence in our model framework and the data quality.

\begin{figure}
\centering
\includegraphics[width=\textwidth]{./joint_vpd_sigma.pdf}
\caption{The joint distribution of $VPD$ and $\sigma$, with outliers removed (defined as lowest and highest 5\% of $\sigma$). $\sigma$ exhibits a weak dependence on $VPD$, and $\sigma$ is $O(1)$ for the bulk of the observations.}
\label{joint_vpd_sigma}
\end{figure}

Before calculating the sensitivity of ET to VPD, we will consider the functional form of Equation \ref{d_et}. There are two main terms: a ``scaling'' term, which modifies the magnitude but not the sign of the ET response to VPD  ($\frac{\partial \; ET}{\partial \; VPD}$):

\begin{equation}
  \frac{g_a \; P}{T(\Delta + \gamma)},
\end{equation}

and a ``sign'' term, which determines whether ET increases or decreases with VPD (i.e. atmospheric demand driven or physiologically controlled):

\begin{equation}
  \label{sign}
  \frac{c_p}{R_{air}} - \frac{\gamma c_a }{1.6 \; R\; \text{ uWUE }} \left( \frac{2 g_1 + \sqrt{VPD}}{2 (g_1 + \sqrt{VPD})^2}\right).
\end{equation}
All variables are positive, so the relative magnitude between the first term and the second term in the sign term (Equation \ref{sign}) will determine whether ET increases or decreases with increasing VPD. If the second term is larger then plant control dominates and ET decreases with increasing VPD. However, if the first term is larger, then atmospheric demand dominates and ET increases with increasing VPD. 

\subsection{Functional Form of the Sign Term}
\label{sign_term}
First, we explore the variables within the sign term to gain better intuition on the driver of either the increase or reduction of ET with VPD. CO$_2$ concentration ($c_a$) and the psychometric constant ($\gamma$) are relatively constant over the dataset considered here so that the variability is dominated by $\sigma$ and $VPD$. uWUE could vary with soil moisture but has been shown to be relatively constant \citep{Zhou_2015}. This then means that the sign term only depends on VPD for a given PFT and is approximately just a function of $VPD$. We can further determine a critical threshold separating an increase from a decrease in ET, i.e. the threshold $VPD_{crit}$ such that the derivative vanishes $\frac{\partial \; ET}{\partial \; VPD} = 0$:
\small
\begin{linenomath*}
  \begin{equation}
VPD_{crit} = \frac{R_{air}}{4 c_p} \left( \frac{\gamma c_a}{1.6\; R \;  uWUE} + \sqrt{\frac{\gamma c_a}{1.6\; R \;  uWUE}\left( \frac{\gamma c_a}{1.6\; R \;  uWUE} + 8 g_1 \frac{c_p}{R_{air}}\right)} - 4 g_1 \frac{c_p}{R_{air}} \right),
\label{vpd_min_et}
  \end{equation}
\end{linenomath*}
\normalsize
noting that $VPD_{crit}$ mostly depends on the PFT parameters uWUE and $g_1$, and only varies weakly with climate as most other parameters related to the environment are nearly constant. The calculated value of $VPD_{crit}$ for each PFT is shown in Table \ref{vpd_crit}. For any values of $VPD$ less than $VPD_{crit}$, ET will thus decrease with increasing VPD ($\frac{\partial \; ET}{\partial \; VPD} < 0$), and for values of $VPD$ greater than $VPD_{crit}$, ET will increase with increasing VPD ($\frac{\partial \; ET}{\partial \; VPD} > 0$). In other words, ecosystems regulate and mitigate evaporative losses up to the VPD limit, $VPD_{crit}$, above which atmospheric demand is just too high to be entirely compensated by stomatal and ecosystem regulation. We note however that even though ET increases again above the critical threshold, $VPD_{crit}$, ET is still much lower that potential evaporation as stomata are still strongly regulating vapor fluxes to the atmosphere. Indeed, even in the absence of soil pore evaporation, stomata do not shut down entirely at very high VPD so that ET does not go to zero, because stomata are still slightly open to perform some photosynthesis \citep{Ball_1987, Leuning_1990, MEDLYN_2011}. In addition, upward xylem transport is necessary to maintain phloem transport, as well as nutrient transport and thus carbon allocation \citep{De_2013, Nikinmaa_2013, Ryan_2014}.



\begin{table}  
\caption{Values of $VPD_{crit}$, where $\frac{\partial \; ET}{\partial \; VPD} = 0$, evaluated at PFT average values for $R_{air}$, $\gamma$, and $c_a$. PFT-specific constants ($g_1$, uWUE) are provided in Table \ref{pft}. For values of $VPD$ less than $VPD_{crit}$, $\frac{\partial \; ET}{\partial \; VPD}$ will be negative, and for values of $VPD$ greater than $VPD_{crit}$, $\frac{\partial \; ET}{\partial \; VPD}$ will be positive.}
\centering
\begin{tabular}{l c c c c c}
  \hline
  PFT & $R_{air}$ & $c_a$ (ppm) & $\gamma$  & \textbf{$VPD_{crit}$ (Pa)} \\
  \hline
CRO & 288.6 & 376.0 & 65.2 &   \textbf{936.6} \\
CSH & 289.0 & 383.6 & 67.5 & \textbf{12852.5} \\
DBF & 288.6 & 379.5 & 63.6 &  \textbf{1266.6} \\
EBF & 288.4 & 368.5 & 61.6 &  \textbf{1454.8} \\
ENF & 288.28 & 379.8 & 61.0 &  \textbf{1853.7} \\
GRA & 288.4 & 378.0 & 61.0 &  \textbf{3261.1} \\
\hline
\end{tabular}
\label{vpd_crit}
\end{table}


Differences in $VPD_{crit}$ are exclusively determined by uWUE and the slope parameter ($g_1$) related to the plant functional type. A larger uWUE means a smaller $VPD_{crit}$, and an ET response to increases VPD that is more likely to be positive. At first glance this result is somewhat counter-intuitive; we expect that plants with a higher water use efficiency would be more water conservative. However, in reality uWUE determines how $WUE$ changes with VPD:

\[WUE = \frac{GPP}{ET} = \frac{uWUE}{\sqrt{VPD}}.\]
\[\frac{\partial \; WUE}{\partial \; VPD} = -\frac{uWUE}{2 \; VPD^{3/2}}\]

So, plants with a higher uWUE will have a greater \textit{decrease} in ecosystem-scale $WUE$ in response to increases in VPD. This decrease in $WUE$ causes more water loss per unit carbon production, and explains the relationship between high uWUE and high likelihood of increases of ET in response to increasing atmospheric drying (increases in VPD).

A tendency towards increasing ET response with increasing VPD can also be caused by a high slope parameter ($g_1$), characteristic of plants that at the leaf scale are more willing to trade water for access to atmospheric CO$_2$. Plants that are less conservative will be thus be more likely to increase ET with increasing VPD. Both the aforementioned effects (large uWUE, $g_1$) can amplify each other, and generally conspire to shift the sign term towards a positive value for a given PFT.

This effect of uWUE and $g1$ on the sign term is most apparent by comparing two extreme PFTs: water intensive crops (CRO) and water conservative closed shrub (CSH). CRO has high slope parameters and uWUE ($g_1 = 183.1$ Pa$^{1/2}$; $2.27$ $\mu$-mol [C] Pa$^{0.5}$ J$^{-1}$ [ET]) compared to CSH ($g1 = 148.6 \, Pa^{1/2}$, $uWUE=1.66$ $\mu$-mol [C] Pa$^{0.5}$ J$^{-1}$ [ET]). These differences in PFT parameters cause opposite ET responses to changes in VPD between CRO and CSH. ET theoretically always decreases with increasing VPD for the more water conservative CSH, while ET frequently  increases with increasing VPD for the more water intensive CRO (Figure \ref{idealized_sign}). CRO evolved or were bred to prioritize GPP and yield and are thus not water conservative. They are very willing to trade water for photosynthesis and productivity, despite changes in VPD, while CSH are very unwilling to trade water for more photosynthesis.

\begin{figure}
\centering
\includegraphics[width=\textwidth]{./idealized_sign.pdf}
\caption{The functional form of the sign term, with $\sigma$ held fixed at 1, and all terms except VPD set to PFT averages. For comparison, the observed range of VPD for each PFT is plotted below the x-axis. Stars denote 25th, 50th, and 75th percentiles, and the range of the line spans the 5th-95th percentiles of observed VPD. Vertical black lines denote the location of $VPD_{crit}$ for each PFT, with the exception of CSH, for which $VPD_{crit}$ is off-scale.}
\label{idealized_sign}
\end{figure}


As expected, the slope parameter ($g_1$) is a primary determinant of the VPD dependence for the sign term shown in Figure \ref{idealized_sign}. Plants that are more conservative (small $g_1$) will tend to reduce ET with increasing VPD, and will be very effective at reducing ET, especially at low VPD. However, at very high VPD, gradients in vapor pressure at the leaf scale will be so strong that they will begin to dominate leaf-scale plant strategies of stomatal (partial) closure (parameterized with $g_1$). As a result, ET response will begin to asymptote towards its ecosystem-scale values irregardless of leaf-scale strategy and stomatal regulation (parameterized with $g_1$). Therefore, plants with a low $g_1$ will have the largest VPD dependence of ET response because the difference in ET response at low VPD (leaf stomatal response dominates) and high VPD (VPD gradient dominates) is largest. This is apparent in the strong VPD dependence of ENF, which has the lowest slope parameter ($g_1=74.31$ Pa$^{1/2}$) (Figure \ref{idealized_sign}).

To summarize our theoretical insights (Figure \ref{idealized_sign} and Table \ref{vpd_crit}), CROs are the least water conservative and have the strongest overall tendency to increase ET with increasing VPD, while CSH are the most water conservative and have the strongest tendency to decrease ET with increasing VPD.  ENF ($g_1 = 74.31$ Pa$^{1/2}$) has by far the largest VPD response dependence, while CRO ($g_1 = 183.1$ Pa$^{1/2}$) has the smallest VPD dependence. Figure \ref{idealized_sign} clearly shows, according to our theory, that for all PFTs except for crops there is frequent occurrence of a negative (plant dominating) ET response to increases in VPD. Therefore, plants are able in most atmospheric conditions to reduce ET in response to increased VPD and thus to reduce water loss. To better illustrate this, the ranges of observed environmental VPDs at the FLUXNET sites are plotted parallel to the x-axis. For CSH, VPD is always less than VPD$_{crit}$ (off scale) so that the plant response dominates in typical environmental conditions, emphasizing the water conservative strategy of those plants. For CRO on the other hand, VPD is higher than VPD$_{crit}$ for more than 50\% of observations, emphasizing that those plants operate with an aggressive water usage strategy, are water intensive and were actually engineered for photosynthesis rather than water saving. For forests and grass,  more than half of the observed VPD are less than VPD$_{crit}$, i.e. in conditions where plant response dominates. It is also important to note that even when atmospheric demand dominates, ET response to VPD is still far more negative than it would be for potential evaporation $\partial PET/\partial VPD$, i.e. atmospheric demand only, emphasizing that there is still a strong regulation of evaporative flux by stomata and though the plant xylem. The sign term in the PET case would just be a constant ($\frac{c_p}{R_{air}} \approx 3.5$), which is far larger than any part of the curves for any PFT. This highlights the deficiencies of PET's ability to capture land response to changes in atmospheric dryness (VPD). Plants are always regulating water exchange from the land surface, even when they reach the limits of they ability to do so. 

\subsection{Functional Form of the Scaling Term}
\label{scale_term}
While the above discussion of the sign of $\frac{\partial \; ET}{\partial \; VPD}$ is important to answer our question of when ET response increases or decreases with VPD, understating the overall magnitude of the ET response is important to soil-plant-atmosphere water budgeting. So we now more closely examine the terms that affect how the sign term is scaled:

\begin{equation}
  \frac{g_a \; P}{T(\Delta + \gamma)}:
\end{equation}

$\frac{P}{T}$ is an air-density term, which varies little compared to aerodynamic conductance and Clausius-Clapeyron ($\Delta$). The psychometric constant ($\gamma$) is also relatively constant, so the scaling term should be primarily a function of aerodynamic conductance and temperature, through the Clausius-Clapeyron relationship $\Delta$. This is as expected, given that the aerodynamic conductance represents the efficiency of exchange between the surface and the atmosphere. As aerodynamic conductance  increases, any plant response will be communicated more strongly to the atmosphere (and vice-versa).

$\Delta$'s presence in the scaling term also matches physical intuition. $\Delta$ (and also the approximately constant $\gamma$) control the efficiencies with which surface energy is converted to latent and sensible heat \citep{Monteith_1965}. The functional from of $\Delta$ will be the same across PFTs, but the temperature range may vary slightly. In contrast, aerodynamic conductance will vary strongly with PFT due to the importance of surface roughness for aerodynamic conductance. So most of the differences in scaling between PFT should be in the aerodynamic conductance term. 

The control of the scaling term variability between PFTs by aerodynamic conductance is confirmed by data (Figure \ref{scale_vary}). Differences between PFT are almost entirely due to differences in aerodynamic conductance, rather than differences in observed temperature ranges. The scaling term for the tree PFTs (DBF, EBF, ENF) is generally about double the scaling terms for the other PFTs which have lower surface roughness and generally smaller aerodynamic conductance.

\begin{figure}
\centering
\includegraphics[width=\textwidth]{./idealized_scale.pdf}
\caption{Primary sources of variability for the scaling term, as a function of PFT. The 5th-95th percentile range of temperature is plotted at the 5th, 25th, 50th, 75th, and 95th percentiles of aerodynamic conductance, as observed for each PFT.}
\label{scale_vary}
\end{figure}

Within each PFT, the scaling term variability is controlled both by environmental temperature and aerodynamic conductance variability (Figure \ref{scale_vary}). While the observed variability of the aerodynamic conductance contributes more to the scaling term variability than temperature, the temperature contribution is non-negligible. Specifically, the scaling term is generally larger at low temperatures when latent heat is relatively inefficient at moving energy away from the surface. This effect amplifies the role of aerodynamic conductance variability at low temperatures.

To summarize, variability between PFTs is mostly controlled by systematic differences in aerodynamic conductance, due to differences in surface roughness between each PFT, and possibly to a lesser extent wind conditions. In contrast, variability within PFT is also controlled by temperature, through Clausius-Clapeyron. But, aerodynamic conductance variability generally impacts the scaling term more than temperature, even within PFT.

\subsection{Bulk statistics of ET response to VPD}
\label{stats_sec}

In this section we consider direct observations of ET response with eddy-covariance data, while including uncertainty with the $\sigma$ term (Section \ref{methods}). These observational results of ET response largely confirm our theoretical analysis, presented above (Table \ref{stats}). For all PFTs, mean ET response to increasing VPD is negative. However, ET response evaluated at the average of all variables (e.g. $\sigma$, $T$, $c_a$, $VPD$) is positive for CRO, and negative for all other PFTs. This difference in mean ET response as compared to the ET response at mean environmental conditions is due to the non-linear nature of the response, in which negative responses are generally larger magnitude than positive responses (Figure \ref{idealized_sign}). Therefore, both the mean ET response as well as the ET response at mean environmental conditions matches our expectations from the theory (Section \ref{sign_term}), with the exception that CRO observations are shifted more towards a negative ET response than we expect.

\begin{table}
\caption{Statistics of $\frac{\partial \; ET}{\partial \; VPD}$ as a function of PFT.}
\centering
\begin{tabular}{l c c c}
  \hline
PFT & $\overline{\frac{\partial \; ET}{\partial \; VPD}}$ & $\frac{\partial \; ET}{\partial \; VPD}\left(\overline{env}\right)$ & fraction $\frac{\partial \; ET}{\partial \; VPD} < 0.$ \\
  \hline
  CRO & -0.05 & 0.008 &  0.533 \\
  DBF & -0.10 & -0.013 &  0.609 \\
  EBF & -0.13 & -0.037 &  0.666 \\
  GRA & -0.07 & -0.024 &  0.694 \\
  ENF & -0.17 & -0.081 &  0.740 \\
  CSH & -0.29 & -0.224 &  0.973 \\
  \hline
\end{tabular}
  \label{stats}
\end{table}

Regarding the frequency of negative and positive ET response, all PFTs exhibit a  decreasing ET response with increasing VPD (physiologically controlled, water conservative response) for the majority of observations. When listed by increasing frequency of negative response, the PFTs are generally ordered as we would expect them to be, from most water intensive to most water conservative: CRO (53\%), DBF (61\%), EBF (67\%), GRA (69\%), ENF (74\%), and CSH (97\%). 

In general the bulk statistics match our theoretical expectations well, with the caveat that inclusion of uncertainty shifts crops towards a slightly more negative response than the theory and Figure \ref{idealized_sign} would suggest. The bulk statistics motivate a more thorough examination of the structure of uncertainty and more sophisticated validation of our theory's performance against observations. 

\subsection{Validation of theory at eddy-covariance sites}
\label{testing}
We now compare more sophisticated distributions of the observed response to our simplified theory (Section \ref{sign_term}). The observed distribution of the sign term, as compared to what the theory would predict, is provided in Figure \ref{test_sign}. Our goal was to capture the leading order behavior of the ET dependence on VPD. Given the assumptions we have made, and the uncertainties of flux tower observations themselves, we expect a relatively large amount of noise when reproducing the derivatives of ET. However, the data largely reproduces our theoretical analysis.

\begin{figure}
\centering
\includegraphics[width=\textwidth]{./test_sign.pdf}
\caption{Comparison of the sign term with model uncertainty included (box plots) to the sign term as calculated with simplifying assumptions (blue line, as in Figure \ref{idealized_sign}). Each box plot corresponds to 5\% of the data, and the 5-95\% range of VPD is plotted.}
\label{test_sign}
\end{figure}

For DBF our theory is successful towards this goal: the theory matches the leading order behavior of the function when uncertainty is included, and the observations match the theory with the addition of noise. The VPD dependence, given by the slope parameter ($g_1$), follows the median values of each bin. Perhaps most importantly, the x-intercept, and thus VPD$_{crit}$, matches nearly exactly between the theory and the observations. Therefore the sign of the ET response to increases in VPD should be well matched, subject to the unavoidable constraints of noise, much of which comes from the observations themselves. The uncertainty is non negligible; there are many observations in each bin for which the the sign of observation is opposite the response predicted by the theory, but to leading order our theory matches the observations well.

While CSH has a much different functional form of the sign term than DBF, CSH observations also match our theory to leading order. Again, the VPD dependence mostly determined from the slope parameter ($g_1$) closely matches the medians in the observation bins and the theoretical threshold is well reproduced (albeit off-scale). The VPD-independent, strongly negative response, which is determined by uWUE, is also captured. For CSH, there is extremely rare occurrence of observed positive ET response with VPD ($\approx 3$\%), even with uncertainty, so the sign of the observations almost always matches the sign of the theory, which states that ET response should always be negative.

As Table \ref{stats} suggested, the theory does not match as well for CRO as for DBF and CSH. At low VPD the VPD dependence (due to $g_1$) is similar to the leading order behavior of the observations, but the theory's curve is shifted towards more positive values relative to the observations, likely due to some bias in uWUE. Most strikingly, the observations show a tendency to move back towards negative values at high VPD, which given the functional form of VPD dependence, is behavior that is impossible for the theory to capture. This general positive bias of the theory explains the shift towards negative ET response observed in the bulk statistics in Table \ref{stats}. The divergence between the theory and the observations for CRO could have been expected. For practical reasons, our theory does not include important sources of plant type heterogeneity within and across sites for CRO. In agricultural landscape, the footprint of an eddy-covariance towers tend to not only observe a single species but many different ones, with potentially varying water use strategies. We also do not account for seasonal changes in LAI, variability in crop height and surface roughness, differences in C3 vs C4 photosynthesis and water strategies, or differences in irrigation practices between the sites and years. These deficiencies should largely explain the inability of our theory to exactly match the observations in croplands. For example, a superposition of sites with C3 plants (at low environmental VPD) and C4 plants (at high environmental VPD) would explain observed shift back towards negative ET response at high VPD when all sites are binned together, as in Figure \ref{test_sign}.  

The theory for GRA suffers from similar limitations as for CRO. GRA observations are characterized by a consistent trend back towards negative ET response at higher VPD, which the functional form of our theory is incapable of accounting for. As compared to CRO, the divergence between the theory and the observations is even greater for GRA, but these differences are explainable using similar logic as we did for CRO. GRA also tends to exhibit substantial within-PFT variability, such as C3 vs C4 photosynthesis, as well as strong seasonal variability in height and LAI. As with CRO, the trend back towards negative ET response at higher VPD could be explained by a superposition of sites with C3 vs C4 plants. Unfortunately, because detailed site-level data of C3 vs C4 plant type is not available, we could not bin CRO and GRA further by irrigation and photosynthesis strategy. However, we hypothesize that the theory would validate against observations much better when photosynthesis pathways are accounted for. 

The comparison between the theory and observations for ENF and EBF is unique from the rest of the PFTs. The VPD dependence of the theory, determined by the slope parameter ($g_1$), is not as well captured by observations. However, the theory still captures the general behavior and magnitude of the response, with some increase in error at low and/or high VPD.  We note that we used published constants for $g_1$ \citep[from ][]{Lin_2015, Franks_2017}, which determines the VPD dependence, and that deviations should be expected across sites \citep{Lin_2015}. We could have fit our own constants so that the theory would best match the observations, but our goal was to use a priori results whenever possible in order to avoid over-fitting and to have a theoretical approach as independent to the observations as possible for further confirmation of the validity of the approach.

While the above discussion shows that our theory has some limitations when applied to some PFTs, especially for crops and grasslands, it does well for DBF and CSH PFTs, and captures the response at non-extreme VPD for ENF and EBF. It is useful to also compare our theory, the observed uncertainty in the ET response, and PET, as PET is often used as a proxy for ET in hydrometeorological studies and drought indices. Again, as in our discussion of the theoretical results, the PET sensitivity to VPD will be constant at about 3.5. In Figure \ref{test_sign} the ET response to shifts in dryness is always far smaller than 3.5 in the observations, even with all uncertainty included. So the PET response to atmospheric dryness is outside the range of all observable responses. This casts doubt on PET's usefulness for capturing the leading order behavior of water budget and dryness response at the land surface, as is done in several drought indices. However our theory (Equation \ref{et}) could be used with published uWUE and $g1$ data, as an alternative to PET to estimating land response and dryness, and would more accurately account for the response to atmospheric drying.

\subsection{Observed ET response to VPD}

Most of the results presented so far focus on the sign term, so now we turn to observations of ET response with the scaling term included (Figure \ref{data_scatter}). Until now, in the interest of developing leading order intuition for ET response, and to be conservative in our acknowledgment of model and observational error, we've considered $\sigma$ variability to be a measure of uncertainty. An alternative viewpoint is that $\sigma \cdot uWUE$ represents spatial and temporal variability of uWUE, which may be expected within bounds \citep[see Table \ref{pft}, also ][]{Zhou_2015}. This is a less conservative view; some of the $\sigma \cdot uWUE$ variability will be due to model and observational error, so by viewing $\sigma \cdot uWUE$ as \textit{real} variability we run the risk of mistaking noise for signal. However, the advantage of this viewpoint is that, from \citet{Zhou_2014}, we have very high confidence that uWUE fit at the hourly timescale (as we do with $\sigma \cdot uWUE$) correctly captures the relationship between ET, GPP, and VPD, and that our form of PM introduces minimal error with its use of uWUE.

\begin{figure}
\centering
\centerline{\includegraphics[width=\textwidth]{./data_scatter.png}}
\caption{Scatter plots of observed $\frac{\partial \; ET}{\partial \; VPD}$, including $\sigma$ variability, as a function of PFT, temperature and VPD. Please note differences in the colorbar scale.}
\label{data_scatter}
\end{figure}

Therefore, with the caveat that some of the signal presented in Figure \ref{data_scatter} may in fact be noise, we can interpret the observed distribution of ET response to VPD. In general, the observed response matches the intuitive theory. ET response to VPD shifts towards positive values as VPD increases (atmospheric demand dominating). CRO exhibit the highest occurrence of positive ET response, and the observations confirm that CRO are the most water intensive. CSH are the most water conservative, with a strong negative response. DBF, EBF, ENF, and GRA are also water conservative, but show some occurrence of positive ET response to VPD, particularly at higher observed VPD, as the theory predicts.

Figure \ref{data_scatter} also includes the impact of the scaling term. For a given VPD, the magnitude of ET response does not vary strongly with temperature, confirming that any impact of the scaling term on the magnitude of the response is primarily due to changes in aerodynamic conductance. Intuitively this is reasonable; aerodynamic conductance will control how dominant balances at the land surface are communicated to the boundary layer. While $\Delta$ controls the efficiency of energy conversion to latent heat, it appears this is a second order term, relative to $g_a$, for scaling ET response. 

As in Figure \ref{test_sign}, Figure \ref{data_scatter} matches our expectation based on simplified theory. The sign term is most strongly scaled by $g_a$, and in general the sign term shifts towards positive values as VPD increases. The willingness with which a given plant type evolved to use water dictates the occurrence of positive versus negative ET response. Water conservative plants are highly effective at mitigating the effects of atmospheric demand, and can store water for later use by reducing ET in response to increasing VPD.

\subsection{Limitations of theory - very dry soil moisture conditions}
In formulating our theory with Penman Monteith, we implicitly did not account for very dry soil moisture conditions. For the majority of environmental conditions  observed at the eddy covariance sites used here, soil conditions were not extremely dry so that we could assume a constant uWUE and $g_1$. We posit that ecosystems will generally optimize to host plants living in conditions which they evolved for. However, in extreme conditions and drought scenarios soil water content (SWC) could become the limiting factor for ET response to VPD, which our theory does not account for. In addition, low soil moisture conditions themselves increase VPD through land-atmosphere feedback \citep[][]{Bouchet_1963, Morton_1965, Brutsaert_1999, Ozdogan_2006, Salvucci_2013, Gentine_2016, Berg_2016}.

We first note that in the data used low SWC conditions are very infrequent. Within our framework, any systematic bias due to the  failure to account for SWC's effects in very dry conditions should manifest itself in a functional relationship between $\sigma$, our uncertainty measure, and SWC, which is observed at the FLUXNET sites. Examining the relationship between $\sigma$ and SWC will test to what extent our theory breaks down in very dry soil moisture conditions.

If we again view $\sigma \cdot uWUE$ as, in addition to a measure of uncertainty, a short time-scale observation of uWUE, we would expect $\sigma \cdot uWUE$ to decrease at low SWC. If $\sigma \cdot uWUE$ is a very strong function of SWC, then our theory should be conditioned more strongly on well watered soil conditions. If $\sigma \cdot uWUE$ is weakly a function of SWC, then our theory is more universal and independent of soil moisture conditions.

Indeed, for all PFTs, there is some slight dependence of $\sigma \cdot uWUE$ on SWC, especially at low SWC (Figure \ref{swc_boxplot}). The portion of observations for which our theory is biased by SWC-limitations varies by PFT, due to the nonlinear threshold at which soil moisture availability limits plant function. For CRO and DBF, soil water content only matters for about the lowest 5\% of observations (each box is 5\% of observations in Figure \ref{swc_boxplot}). And, CRO, relative to DBF, has a very weak dependence on soil moisture, which is reflective of the high likelihood that CRO sites are optimally irrigated and not water stressed. For ENF and EBF, systematic SWC-induced biases in $\sigma \cdot uWUE$ emerge in about the lowest 30\% of SWC conditions. In crontast to ENF, CRO, and DBF, for GRA the relationship between $\sigma \cdot uWUE$ and SWC is more linear and affects a greater portion of observations; about 60\% of observations. Clearly, for GRA soil water frequently impacts plant function and alters ET response, and our theory is limited. It is therefore not suprising that our theory tested poorly againt the data for GRA, relative to ENF, EBF, DBF, and CSH (Section \ref{testing}). For all other PFTs occurence of soil mosture impacts was rare enough to not manifest itself in bulk statistics and plots. A possible exception is CSH, for which there are only two sites, which precludes comment on the exact PFT-wide relationship.

\begin{figure}
\centering
\includegraphics[width=\textwidth]{./swc_boxplot.pdf}
\caption{$\sigma \cdot uWUE$ as a function of PFT and SWC. Each box plot represents 5\% of all observations.}
\label{swc_boxplot}
\end{figure}

The observed dependence of $\sigma \cdot uWUE$ on SWC for GRA would explain the deficiencies of our theory compared to the observations in Figure \ref{idealized_sign}, specifically the trend back towards negative soil moisture at high VPD. Aforementioned feedbacks between the land surface and the atmosphere, which are not accounted for due to our focus on the one-way response of the land surface to atmospheric conditions, would cause high VPD to be correlated with low SWC \citep[][]{Gentine_2016, Berg_2016}. So, at high VPD observations of low SWC are more likely, and this low SWC causes a lower uWUE (Figure \ref{swc_boxplot}). The lower uWUE at low SWC/high VPD then leads to the observed downturn towards decreasing ET with increasing VPD, and the deviation between our theory (based on a constant uWUE assumption) and observations.

To summarize, our theory is limited by its inability to account for soil water impacts on land surface response, and feedbacks between SWC and VPD. Fortunately, for most PFTs SWC's effect on ET is relatively rare (<30\% of observations) and does not manifest itself in the majority of observations and bulk statistics. However, for GRA SWC has a strong effect decreasing water use efficiency for the majority of the observations. Soil moisture effects explain the deficiencies of our theory in Section \ref{testing}, particularly for GRA.

\section{Discussion} 

We derived a new form of Penman Monteith using the concept of semi-empirical optimal stomatal regulation \citep{Lin_2015, MEDLYN_2011} and near constant uWUE \citep{Zhou_2015} to remove the implicit dependence of stomatal conductance on GPP and ET. With our new form of Penman Monteith we developed a theory for when an ecosystem will tend to reduce or increase ET with increasing VPD, which we evaluated against a range of eddy-covariance data spanning different climates and plant functional types. The goal was to capture the leading order behavior of the system to gain some intrinsic knowledge for its behavior. This intuition can be used to disentangle land atmosphere feedbacks in more complicated scenarios, and will aid interpretation of observations and sophisticated models.

Our theory suggests that for a majority of environmental conditions, plants will tend to conserve water and reduce ET with increasing VPD. Stomatal regulation and plant physiological response strongly regulate ET, and this regulation varies by PFT. CROs are the least water conservative, while DBF, EBF, ENF, GRA and CSH are progressively more water conservative (more likely to reduce ET in response to increasing VPD). Observations of ET response to VPD exhibit the same general behavior as the theory, with ET response becoming more positive (atmospheric demand dominating) as environmental VPD increases within a PFT, and more negative for PFTs that are adapted to arid conditions and prioritize water conservation over primary production.

The consequences of our theory our twofold. First, the ET response to VPD is always far more negative than it would be if only atmospheric demand is considered and plant physiological response is ignored (as with PET). Therefore,  any drought indices or analysis using PET will not capture how  vegetated surfaces respond and react to increases in atmospheric demand. However, the new uWUE-version of Penman Monteith we derived (Equation \ref{et}) could be used as a complement to PET in drought indices. One could calculate and compare indices  when plant response is ignored (PET) or included (Equation \ref{et}). 

Lastly, our paper builds important intuition for how plants will respond to global change-induced shifts in VPD, which should increase in the future. Plant physiological responses to direct CO$_2$ effects \citep[e.g.,][]{Swann_2016} receives more attention than physiological response to indirect effects like increased VPD. Here, we provide broad PFT-focused results showing a likely decrease in ET in response to positive VPD perturbations (atmospheric drying). Additionally, we derive an explicit conceptual framework for ET response as a function of two plant type parameters, $g_1$ and uWUE (e.g. Equation \ref{et}). This framework can be used to understand ET response to a range of environmental change, and forms a foundation for resolving varying response to more nuanced consideration of plant type and climate. Any plant physiological heterogeneity that can be conceptualized with shifts in $g_1$ \citep[e.g.][]{Lin_2015} and/or uWUE \citep[e.g.][]{Zhou_2014} is representable within our framework, which opens the door for a hierarchy of more sophisticated climate- and plant-specific analysis of ET sensitivity to environmental variables (including VPD). We argue that such simplified conceptual frameworks are critical tools for disentangling land-atmosphere feedbacks at various scales, from diurnal to seasonal and beyond, and to characterize ET response in a warmer, atmospherically drier, and enriched CO$_2$ world. 


\acknowledgments
This work used eddy covariance data acquired and shared by the FLUXNET community, including these networks: AmeriFlux, AfriFlux, AsiaFlux, CarboAfrica, CarboEuropeIP, CarboItaly, CarboMont, ChinaFlux, Fluxnet-Canada, GreenGrass, ICOS, KoFlux, LBA, NECC, OzFlux-TERN, TCOS-Siberia, and USCCC. The ERA-Interim reanalysis data are provided by ECMWF and processed by LSCE. The FLUXNET eddy covariance data processing and harmonization was carried out by the European Fluxes Database Cluster, AmeriFlux Management Project, and Fluxdata project of FLUXNET, with the support of CDIAC and ICOS Ecosystem Thematic Center, and the OzFlux, ChinaFlux and AsiaFlux offices. This material is based upon work supported by the National Science Foundation Graduate Research Fellowship under Grant No. DGE 16-44869. All code and data used in this analysis, including those used to generate the figures and tables, is publicly available at https://github.com/massma/climate\_et.


\bibliography{references.bib}


\end{document}








%% After you run BibTeX, Copy in the contents of the .bbl file here:


%%%%%%%%%%%%%%%%%%%%%%%%%%%%%%%%%%%%%%%%%%%%%%%%%%%%%%%%%%%%%%%%%%%%%
% Track Changes:
% To add words, \added{<word added>}
% To delete words, \deleted{<word deleted>}
% To replace words, \replace{<word to be replaced>}{<replacement word>}
% To explain why change was made: \explain{<explanation>} This will put
% a comment into the right margin.

%%%%%%%%%%%%%%%%%%%%%%%%%%%%%%%%%%%%%%%%%%%%%%%%%%%%%%%%%%%%%%%%%%%%%
% At the end of the document, use \listofchanges, which will list the
% changes and the page and line number where the change was made.

% When final version, \listofchanges will not produce anything,
% \added{<word or words>} word will be printed, \deleted{<word or words} will take away the word,
% \replaced{<delete this word>}{<replace with this word>} will print only the replacement word.
%  In the final version, \explain will not print anything.
%%%%%%%%%%%%%%%%%%%%%%%%%%%%%%%%%%%%%%%%%%%%%%%%%%%%%%%%%%%%%%%%%%%%%

%%%
\listofchanges
%%%

\end{document}

%%%%%%%%%%%%%%%%%%%%%%%%%%%%%%%%%%%%%
%% Supporting Information
%% (Optional) See AGUSuppInfoSamp.tex/pdf for requirements 
%% for Supporting Information.
%%%%%%%%%%%%%%%%%%%%%%%%%%%%%%%%%%%%%



%%%%%%%%%%%%%%%%%%%%%%%%%%%%%%%%%%%%%%%%%%%%%%%%%%%%%%%%%%%%%%%

More Information and Advice:

%% ------------------------------------------------------------------------ %%
%
%  SECTION HEADS
%
%% ------------------------------------------------------------------------ %%

% Capitalize the first letter of each word (except for
% prepositions, conjunctions, and articles that are
% three or fewer letters).

% AGU follows standard outline style; therefore, there cannot be a section 1 without
% a section 2, or a section 2.3.1 without a section 2.3.2.
% Please make sure your section numbers are balanced.
% ---------------
% Level 1 head
%
% Use the \section{} command to identify level 1 heads;
% type the appropriate head wording between the curly
% brackets, as shown below.
%
%An example:
%\section{Level 1 Head: Introduction}
%
% ---------------
% Level 2 head
%
% Use the \subsection{} command to identify level 2 heads.
%An example:
%\subsection{Level 2 Head}
%
% ---------------
% Level 3 head
%
% Use the \subsubsection{} command to identify level 3 heads
%An example:
%\subsubsection{Level 3 Head}
%
%---------------
% Level 4 head
%
% Use the \subsubsubsection{} command to identify level 3 heads
% An example:
%\subsubsubsection{Level 4 Head} An example.
%
%% ------------------------------------------------------------------------ %%
%
%  IN-TEXT LISTS
%
%% ------------------------------------------------------------------------ %%
%
% Do not use bulleted lists; enumerated lists are okay.
% \begin{enumerate}
% \item
% \item
% \item
% \end{enumerate}
%
%% ------------------------------------------------------------------------ %%
%
%  EQUATIONS
%
%% ------------------------------------------------------------------------ %%

% Single-line equations are centered.
% Equation arrays will appear left-aligned.

Math coded inside display math mode \[ ...\]
 will not be numbered, e.g.,:
 \[ x^2=y^2 + z^2\]

 Math coded inside \begin{equation} and \end{equation} will
 be automatically numbered, e.g.,:
 \begin{equation}
 x^2=y^2 + z^2
 \end{equation}


% To create multiline equations, use the
% \begin{eqnarray} and \end{eqnarray} environment
% as demonstrated below.
\begin{eqnarray}
  x_{1} & = & (x - x_{0}) \cos \Theta \nonumber \\
        && + (y - y_{0}) \sin \Theta  \nonumber \\
  y_{1} & = & -(x - x_{0}) \sin \Theta \nonumber \\
        && + (y - y_{0}) \cos \Theta.
\end{eqnarray}

%If you don't want an equation number, use the star form:
%\begin{eqnarray*}...\end{eqnarray*}

% Break each line at a sign of operation
% (+, -, etc.) if possible, with the sign of operation
% on the new line.

% Indent second and subsequent lines to align with
% the first character following the equal sign on the
% first line.

% Use an \hspace{} command to insert horizontal space
% into your equation if necessary. Place an appropriate
% unit of measure between the curly braces, e.g.
% \hspace{1in}; you may have to experiment to achieve
% the correct amount of space.


%% ------------------------------------------------------------------------ %%
%
%  EQUATION NUMBERING: COUNTER
%
%% ------------------------------------------------------------------------ %%

% You may change equation numbering by resetting
% the equation counter or by explicitly numbering
% an equation.

% To explicitly number an equation, type \eqnum{}
% (with the desired number between the brackets)
% after the \begin{equation} or \begin{eqnarray}
% command.  The \eqnum{} command will affect only
% the equation it appears with; LaTeX will number
% any equations appearing later in the manuscript
% according to the equation counter.
%

% If you have a multiline equation that needs only
% one equation number, use a \nonumber command in
% front of the double backslashes (\\) as shown in
% the multiline equation above.

% If you are using line numbers, remember to surround
% equations with \begin{linenomath*}...\end{linenomath*}

%  To add line numbers to lines in equations:
%  \begin{linenomath*}
%  \begin{equation}
%  \end{equation}
%  \end{linenomath*}



