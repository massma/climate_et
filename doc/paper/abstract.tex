Increasing vapor pressure deficit (VPD) increases atmospheric demand
for water. While increased evapotranspiration (ET) in response to
increased atmospheric demand seems intuitive, plants are capable of
reducing ET in response to increased VPD by closing their stomata. We
examine which effect dominates the response to increasing VPD:
atmospheric demand and increases in ET, or plant response (stomata
closure) and decreases in ET. We use Penman-Monteith, combined with
semi-empirical optimal stomatal regulation theory and underlying water
use efficiency, to develop a theoretical framework for assessing ET
response to VPD. The theory suggests that depending on the environment
and plant characteristics, ET response to increasing VPD can vary from
strongly decreasing to increasing, highlighting the diversity of plant
water regulation strategies.

The ET response varies due to: 1) climate, with tropical and temperate
climates more likely to exhibit a positive ET response to increasing
VPD than boreal and arctic climates; 2) photosynthesis strategy, with
C3 plants more likely to exhibit a positive ET response than C4
plants; and 3) plant type, with crops more likely to exhibit a
positive ET response, and shrubs and gymniosperm trees more likely to
exhibit a negative ET response. These results, derived from previous
literature connecting plant parameters to plant and climate
characteristics, highlight the utility of our simplified framework for
understanding complex land atmosphere systems in terms of idealized
scenarios in which ET responds to VPD only. This response is otherwise
challenging to assess in an environment where many processes co-evolve
together.