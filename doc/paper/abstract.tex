Increasing vapor pressure deficit (VPD) increases atmospheric demand for
water, and vapor pressure deficit is expected to rise with increasing
greenhouse gases. While increased evapotranspiration (ET) in response to
increased atmospheric demand seems intuitive, plants are capable of
reducing ET in response to increased VPD by closing their stomata, in an
effort to conserve water. Here we examine which effect dominates
response to increasing VPD: atmospheric demand and increases in ET, or
plant physiological response (stomata closure) and decreases in ET. We
use Penman-Monteith, combined with semi-empirical optimal stomatal
regulation theory and underlying water use efficiency, to develop a
theoretical framework for understanding how ET responds to increases in
VPD.

The theory suggests that for most environmental conditions and plant
types, plant physiological response dominates and ET decreases with
increasing VPD. Plants that are evolved or bred to prioritize primary
production over water conservation (e.g. crops) exhibit a higher
likelihood of atmospheric demand-driven response (ET
increasing). However for forest, grass, savannah, and shrub plant types,
ET more frequently decreases than increases with rising VPD. This work
serves as an example of the utility of our simplified framework for
disentangling land-atmosphere feedbacks at multiple scales, including
the characterization of ET response in an atmospherically drier,
enriched CO$_2$ world.
