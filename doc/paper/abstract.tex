Increasing vapor pressure deficit (VPD) increases atmospheric demand
for water, and vapor pressure deficit is expected to rise with
increasing greenhouse gases. While increased evapotranspiration (ET)
in response to increased atmospheric demand seems intuitive, plants
are capable of reducing ET in response to increased VPD by closing
their stomata, in an effort to conserve water. Here we examine which
effect dominates response to increasing VPD: atmospheric demand and
increases in ET, or plant physiological response (stomata closure) and
decreases in ET. We use Penman-Monteith, combined with semi-empirical
optimal stomatal regulation theory and underlying water use
efficiency, to develop a theoretical framework for understanding how
ET responds to increases in VPD. The theoretical ET response to VPD
over a range of reasonable environmental conditions and plant
characteristics varies from a strong decrease in ET in response to
increasing VPD (water conservative) to a strong increase in ET in
response to VPD (water intensive), highlighting the diversity of plant
water regulation strategies.

The ET response varies due to: 1) climate, with tropical and
temperate climates more likely to exhibit a positive ET response than
boreal and arctic climates; 2) photosynthesis strategy, with C3 plants
more likely to exhibit a positive ET response than C4 plants; and 3)
due to plant type, with crops more likely to exhibit a positive ET
response, and shrubs and gymniosperm trees more likely to exhibit a
negative ET response. These results, derived from previous literature
connecting plant parameters to plant and climate characteristics,
highlight the utility of our simplified framework for understanding
complex land atmosphere systems in terms of idealized scenarios.
