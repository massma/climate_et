We derived a new form of Penman Monteith using the concept of
semi-empirical optimal stomatal regulation \citep{Lin_2015,
  MEDLYN_2011} and near constant uWUE \citep{Zhou_2015} to remove the
implicit dependence of stomatal conductance on GPP and ET. With our
new form of Penman Monteith we developed a theory for when an
ecosystem will tend to reduce or increase ET with increasing VPD,
which we evaluated against a range of eddy-covariance data spanning
different climates and plant functional types. The goal was to capture
the leading order behavior of the system to gain some intrinsic
knowledge for its behavior. This intuition can be used to disentangle
land atmosphere feedbacks in more complicated scenarios, and will aid
interpretation of observations and sophisticated models.

Our theory suggests that for a majority of environmental conditions,
plants will tend to conserve water and reduce ET with increasing
VPD. Stomatal regulation and plant physiological response strongly
regulate ET, and this regulation varies by PFT. CROs are the least
water conservative, while DBF, EBF, SAV, GRA, MF, WSA, ENF and CSH are
progressively more water conservative (more likely to reduce ET in
response to increasing VPD). SAV and WSA exhibit positive ET response
to VPD not necessarily because of poor water conservation strategies
relative to other PFTs, but because of greater occurrence of high
atmospheric demand (VPD) relative to other PFTs. Observations of ET
response to VPD exhibit the same general behavior as the theory, with
ET response becoming more positive (atmospheric demand dominating) as
environmental VPD increases within a PFT, and more negative for PFTs
that are adapted to arid conditions and prioritize water conservation
over primary production.

Our paper builds important intuition for how plants will
respond to VPD perturbations. We show that given optimal stomatal
function and fixed environmental conditions, the ET-VPD dependence is
theoretically concave upward, with ET increasing with increasing VPD as
VPD increases past some critical value (Table \ref{vpd_crit}). However
future research should focus on fully understanding the form of
stomatal VPD dependence, as this result is sensitive to the exponent of
VPD dependence, which we currently believe is 1/2 \citep{MEDLYN_2011,
  Zhou_2014}. Our results are applicable to understanding the impact
of  expected increases in VPD
induced by global change.  Plant physiological responses to direct
CO$_2$ effects \citep[e.g.,][]{Swann_2016, Lemordant_2018} receives more attention
than physiological response to indirect effects like increased
VPD. Here, we provide broad PFT-focused results showing a likely
decrease in ET in response to positive VPD perturbations (atmospheric
drying), which is consistent with recent observational analysis
\citep[e.g.,][]{Rigden_2017}. Feedbacks between the land and the
atmosphere may alter the net response to a long-timescale global VPD perturbation,
but our focus on the one way plant response to a VPD perturbation in
the atmospheric boundary layer is an important first step to
disentangling such feedbacks, both in observations and model
simulations of the present and future. By removing Penman Monteith's
dependence on implicit relationships between GPP, VPD, and ET, we
allow for explicit, accurate future analysis of plant-VPD feedbacks
in the atmospheric boundary layer (Equation \ref{et}). Our approach
can be extended to examine varying plant response to more nuanced
consideration of plant type and climate. Any plant physiological
heterogeneity or feedback that can be conceptualized with shifts in $g_1$
\citep[e.g.][]{Lin_2015, Medlyn_2017} and/or uWUE
\citep[e.g.][]{Zhou_2014} are representable within our framework, which
opens the door for a hierarchy of more sophisticated climate- and
plant-specific analysis of ET sensitivity to environmental variables
(including VPD). We argue that such simplified conceptual frameworks
are critical tools for disentangling land-atmosphere feedbacks at
various scales, from diurnal to seasonal and beyond, and to
characterize ET response in a warmer, atmospherically drier, and
enriched CO$_2$ world.
