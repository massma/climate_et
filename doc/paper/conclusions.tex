We derived a new form of Penman Monteith using the concept of
semi-empirical optimal stomatal regulation \cite{Lin_2015,
MEDLYN_2011} and near constant uWUE \cite{Zhou_2015} to remove the
implicit dependence of stomatal conductance on GPP and ET. With our
new form of Penman Monteith we developed a theory for when an
ecosystem will tend to reduce or increase ET with increasing VPD,
which we framed using previous literature exploring the relationship
between plant parameters, plant types, and climate. The goal was to
understand the range of possible ET responses to VPD and develop some
intuition for how the ET response may vary with plant types and
climate. This intuition can be used to disentangle land atmosphere
feedbacks in more complicated scenarios, and will aid interpretation
of observations and more sophisticated models.

ET response to VPD can vary from strongly water conservative (ET
decreasing in response to increasing VPD), to strongly water intensive
(ET increasing in response to VPD), which is indicative of the
diversity of possible plant water conservation strategies. Higher uWUE
and g$_1$ values increase the likelihood of a positive ET response,
while decreasing temperature and increasing g$_a$ amplify the
magnitude of the response. Previous literature \cite{Zhou_2015,
Lin_2015} suggests that crops, through association with higher g$_1$
and uWUE, are more likely to exhibit a positive ET response to
VPD. Shrubs (lower uWUE), C4 plants (lower g$_1$), gymnosperm trees
(lower g$_1$), and plants in arctic and boreal climates (lower g$_1$)
are more likely to exhibit a negative ET response to VPD. However
interpretation of g$_1$-induced variability is partially muddied by
ambiguity in applying leaf-scale estimations of g$_1$ to the ecosystem
scale \cite{Medlyn_2017}.

Our paper builds intuition for how plants respond to VPD
perturbations. We show that given optimal stomatal function and fixed
environmental conditions, the ET-VPD dependence is theoretically
concave upward, with ET increasing with increasing VPD as VPD
increases past some critical value where $\frac{\partial \;
ET}{\partial \; VPD} = 0$. However future research should focus on
fully understanding the functional form of VPD dependence, as this
concave up result is sensitive to the exponent of VPD dependence,
which we currently believe is 1/2 for both uWUE and the stomatal
conductance model \cite{MEDLYN_2011, Zhou_2014}. Indeed, this
sensitivity to the exponent of VPD dependence is an important result
itself: land surface models, including those used in earth system
models for climate forecasts, employ different assumptions about the
exponent of VPD dependence in stomatal conductance
\cite<e.g.,>[]{Ball_1987, Leuning_1990, MEDLYN_2011}, and these
assumptions can fundamentally change the relationship between ET and
VPD from one that is concave upward (local minimum in ET) to one that
is concave downward (local maximum in ET).

Our results are also applicable to understanding the impact of
expected increases in VPD induced by global change.  Plant
physiological responses to direct CO$_2$ effects
\cite<e.g.,>[]{Swann_2016, Lemordant_2018} receives more attention
than physiological response to indirect effects like increased
VPD \cite{Novick_2016}. Here, we provide a framework for understanding ET response to VPD
using a simplified model of two plant parameters. Feedbacks between the
land and the atmosphere may alter the net response to a long-timescale
global VPD perturbation, but our focus on the one way plant response
to a VPD perturbation in the atmospheric boundary layer is an
important first step to disentangling such feedbacks, both in
observations and model simulations of the present and future. By
removing Penman Monteith's dependence on implicit relationships
between GPP, VPD, and ET, we allow for explicit, accurate future
analysis of plant-VPD feedbacks in the atmospheric boundary layer
(Eq. (\ref{et})). Our approach can be extended to examine varying
plant response to more nuanced consideration of plant type and
climate. Any plant physiological heterogeneity or feedback that can be
conceptualized with shifts in $g_1$ \cite<e.g.>[]{Lin_2015,
Medlyn_2017} and/or uWUE \cite<e.g.>[]{Zhou_2014} are representable
within our framework, which opens the door for a hierarchy of more
sophisticated climate- and plant-specific analyses of ET sensitivity
to environmental variables (including VPD). We argue that such
simplified conceptual frameworks are critical tools for disentangling
land-atmosphere feedbacks at various scales, from diurnal to seasonal
and beyond, and to characterize ET response in a warmer,
atmospherically drier, and enriched CO$_2$ world.
