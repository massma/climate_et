Vapor pressure deficit (VPD) is expected to rise over continents in
the future due to the combination of increased temperature and,
depending on region, decreased relative humidity
\citep{Byrne_2013}. Increases in VPD increase the atmospheric demand
for evapotranspirated water \citep{Penman_1948, Monteith_1965}, but
also stress plant stomata \citep{Leuning_1990,
MEDLYN_2011}. Understanding the net evaportanspiration (ET) response
to these two opposing effects of changes in VPD is crucial for
assessing the impact of environmental perturbations to VPD on the
water cycle.

The opposing effects of increased atmospheric demand and higher
stomatal stress lead to two possible perspectives for how ET responds
to shifts in VPD. The first, a hydrometeorological perspective, is
that higher VPD increases atmospheric demand for water from the land
surface, and this drives an increase in ET
\citep{Penman_1948}. However, plants' stomata have evolved to
optimally regulate the exchange of water and carbon, and tend to
partially close in response to increased atmospheric dryness
\citep{Farquhar_1978, Ball_1987, Leuning_1990, Katul_2009, MEDLYN_2011}. This
leads to a plant physiology perspective, in which an increase in VPD
may actually correspond to a decrease in ET because of stomatal
closure \citep[e.g.][]{Rigden_2017}.  In other words, the question
``When does VPD drive or reduce ET?'' can be related to whether plant
regulation or atmospheric demand dominates the ET response.

The ET response to changes in VPD alters water partitioning between
the soil and atmosphere. If ecosystem plant response reduces ET with
atmospheric drying then soil moisture will be better conserved. This
represents a sensible evolutionary strategy to cope with aridity: save
water for periods when atmospheric demand for water is relatively low,
and atmospheric carbon can be accessed with a relatively smaller cost
in water loss. If instead stomata were fully passive \citep [similar
to soil pores, e.g. ][]{Or_2013}, increased atmospheric aridity would
strongly reduce soil moisture \citep{Berg_2017}. This could further
increase aridity as low soil moisture levels increases the Bowen
ratio, leading to increased temperature and atmospheric drying
\citep[][]{Bouchet_1963, Morton_1965, Brutsaert_1999, Ozdogan_2006,
Salvucci_2013, Gentine_2016, Berg_2016, Zhou_2019}. Therefore, passive
regulation and a lack of soil moisture conservation does not seem to
be a sensible strategy for plants from an evolutionary standpoint.

** paragraph on soil moisture dependenec? **

We can use intuition about plant water conservation strategy to
hypothesize about ET response to changes in VPD. Plants and ecosystems
that evolved to conserve water, such as arid shrubs, should be more
likely to reduce ET with increasing VPD, and plants that have evolved
or have been engineered to prioritize carbon gain over water
conservation, such as crops, will be more likely to increase ET with
increasing VPD. Atmospheric conditions must matter as well. At the
ecosystem scale, there are limits to plant water conservation
strategies. As atmospheric demand for water (VPD) increases,
ecosystems may begin to reach their water conservation limits and
might not be able to entirely limit ET flux to the atmosphere. At this
stage any further increase in VPD will most likely drive a (limited)
increase in ET, because the increase in atmospheric demand for water
overwhelms the limited plant response to conserve water.

The objective of the present manuscript is to use reasonable
approximations established in prior research as a tool to develop a
framework for understanding plant response to atmospheric drying and
evaluating the VPD dependence of ET while keeping other variables
fixed. This framework will aid interpretation of observations, full
complexity models, and facilitate the disentanglement of complex
land-atmosphere feedbacks. In particular, we can understand the impact
of climate change, given the expected increase of VPD.  In the past,
similar simplified approaches were used to understand interactions
between stomatal conductance, evapotranspiration and the environment
\citep[e.g.,][]{Jarvis_1984, Jarvis_1986, Mcnaughton_1991}. However,
at the time researchers' understanding of the form of VPD's effect on
plant physiology was limited, so they could not systematically explore
the sensitivity of ET to VPD, including VPD's effect on stomatal
conductance and plant function.

Recent results have drastically improved our understanding of VPD's
impact on physiology, especially at the leaf
level. \citet{MEDLYN_2011} developed a model for leaf-scale stomatal
conductance ($g_s$), including VPD response, by combining an optimal
photosynthesis theory \citep{Cowan_1977, Katul_2009} with an empirical
approach, and extended use of this model to the ecosystem scale in
\citet{Medlyn_2017}. Additionally, \citet{Zhou_2014} demonstrated that
a quantity underlying water use efficiency $\left(uWUE = \frac{GPP\;
\sqrt{VPD}}{ET}\right)$ properly captures a constant relationship
between GPP, ET, and VPD over a diurnal cycle at the ecosystem
scale. uWUE is also relatively well conserved in the growing season
across space and time, within a plant functional type (PFT)
\citep{Zhou_2015}. ** here talk about difference between leaf level
and ecosystem scale, also connect uWUE to g1 ** While stomatal
conductance parameterizations and uWUE greatly simplify complex plant
physiological processes, they therefore still capture ecosystem
behavior for vegetated surfaces \citep{Medlyn_2017, Zhou_2014}, and
are novel tools to transparently develop intuition for the behavior of
complex land-atmosphere systems.

In this manuscript, we leverage uWUE and recent developments in
stomatal conductance parameterizations \citep{MEDLYN_2011} to derive
the theoretical one-way response of ET to VPD with other environmental
variables properly controlled for, i.e. we develop a framework for
evaluating the partial derivative of ET with respect to VPD. We
explicitly include VPD's full effect on stomatal conductance,
including its impact on photosynthesis. We explore the range of
possible ET responses to VPD, given parameters previously established
in peer reviewed literature. Additionally, we show the sensitivity of
the ET-VPD relationship to model and framework choice, highlighting
the importance of: 1) future research on stomatal conductance and
ecosystem scale modeling, and 2) thoughtful selection of
photosynthesis and stomatal conductance models in more sophisticated
land surface and earth system models. Finally, we discuss the caveats
of our approach and the sensitivity of results to assumptions about
stomatal conductance.
