Vapor pressure deficit (VPD) is expected to rise over continents in
the future due to the combination of increased temperature and,
depending on region, decreased relative humidity
\cite{Byrne_2013}. Increases in VPD increase the atmospheric demand
for evapotranspirated water \cite{Penman_1948, Monteith_1965}, but
also reduce stomatal conductance through stomatal closure
\cite{Rawson1977, Leuning_1990, Mott2007, Damour2010,
MEDLYN_2011}. Understanding the net evapotranspiration (ET) response
to these two opposing effects of changes in VPD is crucial for
assessing the impact of environmental VPD perturbations on the water
cycle.

The opposing effects of increased atmospheric demand and higher
stomatal closure lead to two possible perspectives for how ET responds
to shifts in VPD. The first, a hydrometeorological perspective, is
that higher VPD increases atmospheric demand for water from the land
surface, and this drives an increase in ET
\cite{Penman_1948}. However, plants' stomata have evolved to
optimally regulate the exchange of water and carbon, and tend to
partially close in response to increased atmospheric dryness
\cite{Farquhar_1978, Ball_1987, Leuning_1990, Katul_2009,
MEDLYN_2011}. This leads to a plant physiology perspective, in which
an increase in VPD may actually correspond to a decrease in ET because
of stomatal closure \cite<e.g.>[]{Rigden_2017}.  In other words, the
question ``When does VPD drive or reduce ET?'' can be related to
whether plant regulation or atmospheric demand dominates the ET
response.

The ET response to changes in VPD alters water partitioning between
the soil and atmosphere. If ecosystem plant response reduces ET with
atmospheric drying then soil moisture will be better conserved. This
represents a sensible evolutionary strategy to cope with aridity: save
water for periods when atmospheric demand for water is relatively low,
and atmospheric carbon can be accessed with a relatively smaller cost
in water loss. If instead stomata were fully passive \cite<similar
to soil pores, e.g. >[]{Or_2013}, increased atmospheric aridity would
strongly reduce soil moisture \cite{Berg_2017}. This could further
increase aridity as low soil moisture levels increases the Bowen
ratio, leading to increased temperature and atmospheric drying
\cite{Bouchet_1963, Morton_1965, Brutsaert_1999, Ozdogan_2006,
Salvucci_2013, Gentine_2016, Berg_2016, Zhou_2019}. Therefore, passive
regulation and a lack of soil moisture conservation does not seem to
be a sensible strategy for plants from an evolutionary
standpoint. This simplified logic explains generally why plants
evolved to respond to VPD, but also excludes many details and special
cases (e.g. plant to plant interaction and highly specialized
photosynthesis strategies like Crassulacean acid metabolism
photosynthesis).

We can use intuition about plant water conservation strategy to
hypothesize about ET response to changes in VPD. Plants and ecosystems
that evolved to conserve water, such as arid shrubs, should be more
likely to reduce ET with increasing VPD, and plants that have evolved
or have been engineered to prioritize carbon gain over water
conservation, such as crops, will be more likely to increase ET with
increasing VPD. Atmospheric conditions must matter as well. At the
ecosystem scale, there are limits to plant water conservation
strategies. As atmospheric demand for water (VPD) increases,
ecosystems may begin to reach their water conservation limits and
might not be able to entirely limit ET flux to the atmosphere. At this
stage any further increase in VPD will most likely drive a (limited)
increase in ET, because the increase in atmospheric demand for water
overwhelms the limited plant response to conserve water.

The objective of the present manuscript is to use reasonable
approximations established in prior research as a tool to develop a
framework for understanding plant responses to atmospheric drying and
the the VPD dependence of ET while keeping other variables fixed. This
framework will aid interpretation of observations, full complexity
models, and facilitate the disentanglement of complex land-atmosphere
feedbacks. In particular, our approach has applications for
understanding climate change impacts, given expected increases in VPD
with rising temperature.  In the past, similar simplified approaches
were used to understand interactions between stomatal conductance,
evapotranspiration and the environment \cite<e.g.,>[]{Jarvis_1984,
Jarvis_1986, Mcnaughton_1991}. However these researchers did not
explore explicitly the sensitivity of ET to VPD, including VPD's
effect on stomatal conductance and plant function.

Approaching the problem of ET response to VPD is aided by recent
results drastically improving our understanding of VPD's impact on
physiology, especially at the leaf level. \citeA{MEDLYN_2011}
developed a model for leaf-scale stomatal conductance ($g_s$),
including VPD response, by combining an optimal photosynthesis theory
\cite{Cowan_1977, Katul_2009, Katul_2010} with an empirical approach,
and extended use of this model to the ecosystem scale in
\citeA{Medlyn_2017}. Additionally, \citeA{Zhou_2014} demonstrated that
a quantity underlying water use efficiency $\left(uWUE = \frac{GPP\;
\sqrt{VPD}}{ET}\right)$ properly captures a constant relationship
between GPP, ET, and VPD over a diurnal cycle at the ecosystem
scale. uWUE is also relatively well conserved in the growing season
across space and time, within a plant functional type (PFT)
\cite{Zhou_2015}.  While stomatal conductance parameterizations and
uWUE greatly simplify complex plant physiological processes, they
 still capture ecosystem behavior for vegetated surfaces
\cite{Medlyn_2017, Zhou_2014}, and are useful tools to transparently
develop intuition for the behavior of complex land-atmosphere systems.

In this manuscript, we leverage uWUE and recent developments in
stomatal conductance parameterizations \cite{MEDLYN_2011} with a
Penman-Monteith framework \cite<hereafter PM,>[]{Penman_1948,
Monteith_1965} to derive the theoretical one-way response of ET to VPD
with other environmental variables properly controlled for, i.e. we
develop a framework for evaluating the partial derivative of ET with
respect to VPD. It is useful to disconnect the impact of VPD from
other variables as VPD is known to increase dramatically with future
climate and limit ET more than soil moisture \cite{Novick_2016}, but
disentangling VPD effects from soil moisture effects has been
difficult in previous research, given their co-variability
\cite{Lin_2018, Zhou_2019}.  We are
able to disentangle the effect of VPD on ET because we explicitly
include VPD's full effect on stomatal conductance, including its
impact on photosynthesis. The use of an energy balance framework (PM)
allows us to include in our analysis the effects of the energy cost of
evaporating water from a surface, which is an important factor in the
natural environment, compared to prescribed in situ environmental
conditions. This is relevant because previous research focusing on the
leaf scale \cite{Rawson1977, Turner1984, Oren1999, Damour2010,
Mott2013} does not consistently or analogously include the energetic
constraints on evapotranspiration under which ecosystems operate. The
leaf-scale results agree that stomatal conductance decreases in
response to VPD \cite{Oren1999, Damour2010}, which we expect to be
true for an ecosystem as well. However leaf scale results also
indicate that transpiration usually increases with increasing VPD in a
concave downward shape \cite<e.g.,>[]{Rawson1977, Turner1984,
Mott2013}, which may not be true for ET at the ecosystem scale once
energetic constraints on ET are included in the analysis.  Our
approach allows us to estimate the expected ET response to VPD at the
ecosystem scale, including the effects of surface energy balance
constraints, and assess how ecosystem ET response to VPD deviates from
previous leaf scale analyses.

This manuscript presents the range of possible ecosystem-scale ET
responses to VPD, given parameters previously established in peer
reviewed literature. Additionally, we explore the sensitivity of the
ET-VPD relationship to stomatal model and framework choice,
highlighting the importance of: 1) future research on stomatal
conductance and ecosystem scale modeling, and 2) thoughtful selection
of photosynthesis and stomatal conductance models in more
sophisticated land surface and earth system models.
