\documentclass[12pt]{article}

% fonts
\usepackage[scaled=0.92]{helvet}   % set Helvetica as the sans-serif font
\renewcommand{\rmdefault}{ptm}     % set Times as the default text font

% db/ not mandatory, but i recommend you use mtpro for math fonts.
% there is a free version called mtprolite.

% \usepackage[amssymbols,subscriptcorrection,slantedGreek,nofontinfo]{mtpro2}

\usepackage[T1]{fontenc}
\usepackage{amsmath}
\usepackage{amsfonts}

% page numbers
\usepackage{fancyhdr}
\fancypagestyle{newstyle}{
\fancyhf{} % clear all header and footer fields
\fancyfoot[R]{\vspace{0.1in} \small \thepage}
\renewcommand{\headrulewidth}{0pt}
\renewcommand{\footrulewidth}{0pt}}
\pagestyle{newstyle}

% geometry of the page
\usepackage[top=1in,
            bottom=1in,
            left=1in,
            right=1in]{geometry}

% paragraph spacing
\setlength{\parindent}{0pt}
\setlength{\parskip}{2ex plus 0.4ex minus 0.2ex}

% useful packages
\usepackage{natbib}
\usepackage{epsfig}
\usepackage{url}
\usepackage{bm}


\begin{document}

\large \textbf{Response to Reviewers: \textit{When does vapor pressure deficit drive or reduce
    evapotranspiration?}} \\
Adam Massmann, Pierre Gentine, and Changjie Lin \\
\today

\vspace{0.1in}

\normalsize
\section{Response to Reviewer 1}

Thank you very much for the thoughtful and thorough review. We
responded to all comments in the annotated PDF, but this is our first
time using that tool so if there are any issues please let us know. As
an aside, thanks for introducing us to the annotated PDF feature. We
were not aware of it, and it was pleasant to use.

\section{Response to Reviewer 2}

As with reviewer 1, thank you very much for the time spent on a
thoughtful review. Here are responses to your comments:

\begin{enumerate}

\item This paper actually discusses transpiration not
  evapotranspiration - while the role of soil evaporation is sometimes
  acknowledged (and then dismissed) - it would be much cleaner if
  references to evapotranspiration (ET) were replaced with
  transpiration (T). The role of soil evaporation and interception in
  a changing climate is a whole other (and difficult!) issue.

  \textit{Response}: Yes, we agree that the underlying theory was
  developed for transpiration, and that it would probably appear
  cleaner to just present the manuscript in terms of transpiration
  instead of evapotranspiration. However, we prefer to use ET, along
  with the language discussing the theory's likely limitation when
  evaporation is high relative to transpiration, for a few of
  reasons. The parameter ranges we use in Table 2 for uWUE and g1 and
  are derived from ecosystem ET measurements in Zhou et al. 2015 and
  Medlyn et al. 2017, rather than T measurements. Given the
  semi-empirical nature of the theory, and that the parameters were
  developed empirically from ET observations rather than T
  observations (and it would require the reader to dig into cited
  literature to figure this out), we believe it is more transparent to
  call the quantity ET rather than T, while including explicit
  language about the possible weaknesses of the theory when
  evaporation is large relative to transpiration. Additionally, Zhou
  et al. 2014 and Zhou et al. 2015 use "ET" rather than "T", so by
  using "ET" we are more consistent with that notation. However,
  Medlyn et al. 2017 use "E" for this quantity (their Eq. 5), but call
  it "canopy transpiration," which we actually did not prefer because
  we felt using "transpiration" obfuscated the fact that their
  estimate of $g$ was actually derived from ET measurements. So, while
  we agree that it could appear cleaner to just call the quantity T
  and remove our language discussing the theory's limitations and
  caveats with respect to evaporation, we also believe this could hide
  the fact the we use semi-empirical parameters derived for ET, and
  distance our work from the previous work we directly build upon (in
  particular Zhou et al. 2014 and 2015). While not relevant at all
  (but still helpful in the end), keeping ET rather than adopting T
  has a practical advantage that our title would not need to be
  changed, which would help match the final peer reviewed version of
  the manuscript to already publicly available pre-prints of the
  manuscript.

\item Lines 80-90. Plants also use evaporation to cool their leaves - is
  this significant? And does it suggest there should be some residual
  stomatal opening?

  \textit{Response}: Yes, some plants use transpiration to regulate
  temperature on their leaves in some conditions, and this could be a
  cause of stomatal opening/closing. We added language at lines 78-80
  acknowledging that the simplified logic in the opening excludes
  details, including how plants can use transpiration for thermal
  regulation: ``...excludes many details ... e.g., ... thermal
  regulation with transpiration ... Lin et al. 2017''

\item Line 210-211. It would be worth expanding on the RH - es
  relationship - what data?

\textit{Response}: Agreed, we added a figure to the supplemental material on
    the relationship between RH and es, and a reference to it in the text.

\item Table 4 legend. Needs a bit of clarification on the choice of ranges
  of T and ga.

  \textit{Response}: Agreed; added clarifying language to the caption
  for how T and g$_a$ ranges were estimated (FLUXNET data): ``Values
  for $T$ and g$_a$ are calculated from FLUXNET-2015 data (see
  supplemental material for description), rounding the 15th, 50th, and
  85th percentile to the nearest 5$^o$C and 0.005 m/s, respectively.''

\item Line 322-335. I found this paragraph a bit ambiguous. The second
  sentence suggests most ptfs are similar to the median type but the
  following sentences suggest this is not true. I think this is
  important - the reader needs to know whether the max and min values
  of parameters are common or outliers.

\textit{Response}: Thanks; we reworded this paragraph to make it more clear and
    explicit: "...However, crops, shrubs, and grass are exceptions; they
    exhibit an uWUE different than the median value in Table 2. Crops,
    which we expect to prioritize carbon uptake over water conservation,
    have a higher uWUE value closer to our maximum value. This higher uWUE
    results in a higher likelihood for an increasing ET response to VPD,
    which matches intuition given that we expect crops to keep stomata
    open for access to carbon, at the cost of increased water loss during
    high VPD. Shrubs, and to a lesser extent grass, have an uWUE closer to
    the minimum uWUE. For these plant types, we would then expect a
    decrease of ET in response to VPD."

\item Supplementary material. As I say above I commend the inclusion of
  this 'negative' result.  However I would like to know how soil
  water was derived - is this measured or modelled and is it at the
  surface or through the entire rooting zone?

\textit{Response}: Yes, more clarity about the source of the data is
    helpful. We added language to the supplement explicitly stating that
    SWC is taken from the FLUXNET-2015 data; specifically the shallowest
    observed layer at each site: "For SWC measurements, we use the
    shallowest observed layer available at each site."
\end{enumerate}


\end{document}

%%% Local Variables:
%%% mode: latex
%%% TeX-master: t
%%% End:
