
\section{Methods}
\label{methods}
The Penman-Monteith equation \citep [hereafter PM,][]{Penman_1948,
  Monteith_1965} estimates ET as a function of observable atmospheric
variables and surface conductances:
% \begin{linenomath*}
  \begin{equation}
    \label{orig_pen}
    ET = \frac{\Delta R_{net} + g_a \rho_a c_p VPD}{\Delta + \gamma(1 + \frac{g_a}{g_s})},
  \end{equation}
% \end{linenomath*}
where $\Delta$ is the change in saturation vapor pressure with
temperature, given by Clausius-Clapeyron ($\frac{d \; e_s}{d \; T}$),
$R_{net}$ is the net radiation minus ground heat flux, $g_a$ is
aerodynamic conductance, $\rho_a$ is air density, $c_p$ is specific
heat of air at constant pressure, $\gamma$ is the psychometric
constant, and $g_s$ is the stomatal conductance (Table
\ref{definitions}).

\begin{table}
  \caption{Definition of symbols and variables, with citation for how
    values are calculated, if applicable.}
  \label{definitions}
  \centering \footnotesize
  \begin{tabular}{l c c c}
    \hline
    Variable & Description & Units & Citation \\
    \hline
    $e_s$  & saturation vapor pressure & Pa  & - \\
    $T$  & temperature  & K & - \\
    $P$  & pressure & Pa  & - \\
    $\Delta$  & $\frac{\partial e_s}{\partial T}$ & Pa K$^{-1}$ & - \\
    $R_{net}$  & net radiation at land surface minus ground heat flux & W m$^{-2}$   & - \\
    $g_a$  & aerodynamic conductance & m s$^{-1}$  & \citet{Shuttleworth_2012} \\
    $\rho_a$  & air density & kg m$^{-3}$  & - \\
    $c_p$  & specific heat capacity of air at constant pressure & J K$^{-1}$ kg$^{-1}$ & - \\
    $VPD$  & vapor pressure deficit & Pa  & - \\
    $\gamma$  & psychometric constant & Pa K$^{-1}$   & - \\
    $g_{s-leaf}$  & leaf-scale stomatal conductance & m s$^{-1}$  & \citet{MEDLYN_2011} \\
    $g_{s}$  &  stomatal conductance & m s$^{-1}$
                                   & \citet{Medlyn_2017} \\
    $g_{1-leaf}$  & leaf-scale slope parameter & Pa$^{0.5}$
                                   & \citet{MEDLYN_2011} \\
    $g_{1}$  & ecosystem-scale slope parameter & Pa$^{0.5}$ & \citet{Medlyn_2017} \\
    $R$ & universal gas constant & J mol$^{-1}$ K$^{-1}$ & - \\
    $R_{air}$ & gas constant of air & J  K$^{-1}$ kg$^{-1}$ & - \\
    $\sigma$ & uncertainty parameter & -& - \\
    $c_a$ & CO$_2$ concentration & $\mu$ mol CO$_2$ mol$^{-1}$ air& - \\
    $\lambda$ & marginal water cost of leaf carbon & mol H$_2$O mol$^{-1}$ CO$_2$ & - \\
    $\Gamma$ & CO$_2$ compensation point & - & - \\
    $\Gamma^*$ & CO$_2$ compensation point without dark respiration & - & - \\
    \hline
  \end{tabular}
\end{table}


\citet{MEDLYN_2011} developed a model for stomatal conductance ($g_s$)
by combining an optimal photosynthesis theory \citep{Cowan_1977} with an empirical approach, which describes the
dependence of $g_s$ to VPD. This resulted in the following model for
leaf-scale stomatal conductance:

% \begin{linenomath*}
  \begin{equation}
    g_{s-leaf} = g_0 + 1.6 \left(1 +
      \frac{g_{1-leaf}}{\sqrt{VPD}}\right) \frac{A}{c_a},
    \label{leaf_medlyn}
  \end{equation}
% \end{linenomath*}
where $g_{1-leaf}$ is a leaf-scale ``slope'' parameter, A is the net
CO$_2$ assimilation rate, and $c_a$ is the atmospheric CO$_2$
concentration at the leaf surface. \cite{MEDLYN_2011} relate the slope
parameter ($g_{1-leaf}$) to physical parameters as:
% \begin{linenomath*}
  \label{slope}
  \begin{equation}
    g_{1-leaf} = \sqrt{\frac{3 \, \Gamma^* \, \lambda}{1.6}},\footnote{Note this expression has units of of (mmol mol$^{-1}$)$^{1/2}$, but this can be converted to Pa$^{1/2}$ using the ideal gas law.}
  \end{equation}
% \end{linenomath*}

where $\Gamma^*$ is the CO$_2$ compensation point for photosynthesis
(without dark respiration), and $\lambda$ is the marginal water cost
of leaf carbon
($\frac{\partial \; \text{transpiration}}{\partial \; A}$). So,
$g_{1-leaf}$ is a leaf-scale term reflecting the trade-off of water for
carbon uptake. The higher $g_{1-leaf}$, the more open the stomata and
the more they release water in exchange for carbon.


The Medlyn model for stomatal conductance has been shown to behave
very well across PFTs \citep[][]{Lin_2015}, and has been successfully
adopted to ecosystem scale analysis in \citet{Medlyn_2017}. In units
of m s$^{-1}$, the ecosystem scale stomatal conductance is:

% \begin{linenomath*}
  \begin{equation}
    g_s = \frac{R \,T}{P} \; 1.6 \left(1 + \frac{g_1}{\sqrt{VPD}}\right) \frac{GPP}{c_a},
    \label{medlyn}
  \end{equation}
% \end{linenomath*}

  where GPP is the ecosystem scale gross primary production, and $g_1$
  is an ecosystem scale analogue to $g_{1-leaf}$. We solve for $g_1$
  following \citet{Medlyn_2017} (Eq. (5)), and take the median $g_1$
  value to be representative of each PFT (Table \ref{pft}) instead of
  the mean to avoid extra weighting of rare outliers.

While Eq. (3) can be used in PM (Eq. (1)), it will make
analytical work with the function intractable because $GPP$ is
functionally related to ET itself. Additionally, a perturbation to VPD
should induce a physiological plant response that will alter GPP and
cause an indirect change in stomatal conductance, in addition to the
direct effect of VPD in Eq. (\ref{medlyn}). Therefore, in order to
derive the response of ET to VPD, we must account for the functional
relationship between GPP, ET, and VPD, and its effect on stomatal
conductance. We can use aforementioned semi-empirical results of
\citet{Zhou_2015} as a tool to approach this
problem. \citet{Zhou_2015}, showed that  underlying Water Use
Efficiency (uWUE):

% \begin{linenomath*}
  \begin{equation}
    uWUE = \frac{GPP \cdot \sqrt{VPD}}{ET}
    \label{uwue}
  \end{equation}
% \end{linenomath*}

is relatively constant across time and moisture conditions within a
plant functional type, and correctly captures a constant relationship
between GPP, ET and VPD over a diurnal cycle \citep{Zhou_2014}. The
theoretical derivation of the square root VPD dependence in $uWUE$
leverages the same assumptions used in \cite{MEDLYN_2011} to derive the
square-root VPD dependence of the stomatal conductance model (Eq.
(\ref{medlyn})).  We can use uWUE to remove the $GPP$ dependence from
$g_s$ in a way that makes PM analytically tractable:

% \begin{linenomath*}
  \begin{equation}
    g_s = \frac{R \, T}{P} 1.6 \left(1 + \frac{g_1}{\sqrt{VPD}}\right) \frac{uWUE \; ET}{c_a \; \sqrt{VPD}}.
    \label{new_g_s}
  \end{equation}
% \end{linenomath*}

Plugging Eq. (\ref{new_g_s}) into Eq. (\ref{orig_pen}) and
rearranging gives a new explicit expression for PM, in which
dependence on $GPP$ is removed:

% \begin{linenomath*}
  \begin{equation}
    ET = \frac{\Delta R_{net} + \frac{g_a\; P}{T} \left( \frac{ c_p VPD}{R_{air}} -  \frac{\gamma c_a \sqrt{VPD} }{ R \; 1.6\; \text{ uWUE } (1 + \frac{g_1}{\sqrt{VPD}})} \right) }{ \Delta + \gamma}
    \label{et}
  \end{equation}
% \end{linenomath*}

By accounting for photosynthesis changes in ecosystem conductance, with
Eq. (\ref{et}) we have derived for the first time, using recent
results \citep[][]{MEDLYN_2011, Zhou_2014, Zhou_2015, Medlyn_2017}, ET
explicitly as function of environmental variables and two
plant-specific constants, the slope parameter ($g_1$), and uWUE, both
reflecting water conservation strategy. The slope parameter is related
to the willingness of stomata to trade water for CO$_2$ and to keep
stomata open. uWUE is a semi-empirical ecosystem-scale constant
related to how WUE changes with VPD (specifically $VPD^{-1/2}$). It is
also roughly proportional to physical constants:

\[uWUE \appropto \sqrt{\frac{c_a - \Gamma}{1.6 \lambda}},\]

where $\Gamma$ is the CO$_2$ compensation point \citep[Eq. (5)
in][]{Zhou_2014}. So uWUE is related to atmospheric CO$_2$
concentration and compensation point, and is inversely proportional to
the marginal water cost of leaf carbon.


With Equation \ref{eq} we can take the partial derivative with respect
to VPD:

% \begin{linenomath*}
  \begin{equation}
    \frac{\partial \;  ET}{\partial \; VPD} = \frac{2\; g_a \;
      P}{T(\Delta + \gamma)}   \left(\frac{ c_p}{R_{air}} -
      \frac{\gamma c_a }{1.6 \; R\; \; \text{ uWUE }} \left(
        \frac{2 g_1 + \sqrt{VPD}}{2 (g_1 + \sqrt{VPD})^2}\right)
    \right),
    \label{d_et}
  \end{equation}
% \end{linenomath*}

  deriving an analytical framework for ecosystem response to
  atmospheric demand perturbations with environmental conditions held
  fixed. There are a few subtleties to taking the derivative in
  Eq. (\ref{d_et}): $\Delta$ ($\frac{d e_{s}}{d T}$) and $VPD$ are
  functionally related, so while taking the derivative we evaluate
  $\frac{\partial \; ET}{\partial \; VPD} = \frac{\partial \; ET}
  {\partial \; e_s} \frac{\partial \; e_s}{\partial \; VPD}
  \Big|_{\text{RH fixed}} + \frac{\partial \; ET}{\partial \; RH}
  \frac{\partial \; RH}{\partial \; VPD} \Big|_{\text{$e_s$
      fixed}}$. $RH$ and $e_s$ are assumed to be approximately
  independent, which is supported by data (not shown).

  This derivation relied either implicitly or explicitly on several
  assumptions. First, we assume that VPD at the leaf surface is the
  same as VPD at measurement height; physically this implies that
  leaves are perfectly coupled to the atmosphere. In reality, for some
  conditions and plant types the leaves can become decoupled from the
  boundary layer \citep{De_2017, Medlyn_2017}. Therefor, our
  derivation will be most applicable in times like the growing season,
  when relatively high insolation induces instability and convective
  boundary layers, and we would expect the surface to be generally
  well coupled. An additional assumption in the formulation of $uWUE$
  \citep{Zhou_2014, Zhou_2015} and \citet{Medlyn_2017}'s stomatal
  conductance model is that direct soil evaporation (E) contributions
  to ET remain small relative to transpiration (T). This should be
  more true during the growing season. The ratio of E to T may
  increase immediately after rainfall events due to high soil moisture
  and ponding, but VPD is generally low anyways during these
  times. However, some plant types allow for systematically larger
  contributions of E in ET, particularly those with sparse canopies
  and smaller relative amounts of transpiration. We therefore might
  expect that the theory will be most applicable to forest PFTs, which
  will be most strongly coupled to the boundary layer due to larger
  surface roughness, and will also generally have the highest ratios
  of transpiration to evaporation. Finally, we assume that $g_1$ and
  $uWUE$ are constant with respect to the conceptual VPD
  perturbation. Both quantities have been shown to be relatively
  constant with respect to changes in
  % TODO: make this a general discussion on partial derivatives and
  % what things are constant/how to interpret?
  VPD \cite{Franks_2017, Zhou_2014}. These parameters will however
  vary with plant species and characteristics \citep[e.g. wood density,
  ]{Lin_2015}, as well as environmental conditions including soil
  water content and temperature.
  % TODO: write exact parameter ranges we use, and introduce
  % scaling/sign term here?
  % \item

  To account for both this variability in plant parameters and the
  environment, we will systematically analyze how the ET reponse to
  VPD (Equation \ref{d_et}) varies. Equation \ref{d_et} includes a
  ``sign'' term that determines the sign of the response in addition
  to magnitude:

\begin{equation}
  \label{sign}
  \frac{c_p}{R_{air}} - \frac{\gamma c_a }{1.6 \; R\; \text{ uWUE }} \left( \frac{2 g_1 + \sqrt{VPD}}{2 (g_1 + \sqrt{VPD})^2}\right),
\end{equation}

and a ``scaling'' multiplying the ``sign'' term:

\begin{equation}
  \frac{g_a \; P}{T(\Delta + \gamma)}.
\end{equation}

In the ``sign term'' most of quantities are relatively constant,
except for VPD and the plant parameters g$_1$ and uWUE. In the scaling
term, most of the terms are relatively constant with the exception of
aerodynamic conductance ($g_a$) and temperature (especially its effect
through $\Delta$). To determine the range of probable ET reponses to
VPD we will systematically vary this parameters according to Table
\ref{param_varying}, while all other parametrs are held fixed (Table
\ref{param_fixed}). Physical variables ($g_a$, $T$) and plant
physiological paramters (g$_1$, uWUE) are varied according to
literature-based expectations for a range of growing season conditions
and plant types \citep{Zhou_2015, Medlyn_2017}. Using this previous
literature we can connect the effect of varying plant parameters to
specific plant types and characteristics.

All code and data used in this analysis, including those used to
generate the figures and tables, are publicly available at
\url{https://github.com/massma/climate\_et}.
%\hyperref[https://github.com/massma/climate\_et]{https://github.com/massma/climate_et}.

\begin{table}
  \caption{Variable quantities in the ET response to VPD. Each value
    is varied to determine the effect of a range of expected plant and
    environmental conditions on ET response to VPD. A citation is
    provided in cases where the quantites are directly derived from
    previous literature. Conceptual percentiles
    for interpreting each quantity are noted.}
  \label{param_varying}
  \centering
  \begin{tabular}{l c c c c c}
    \hline
    Symbol & units [units in citation] & min (~15th percentile) & med (~50th percentile) & max
                                                                                            (~15th
                                                                                            percentile)
    & citation  \\
    \hline
    \input{param_varying.tex}
    \hline
  \end{tabular}
\end{table}

\begin{table}
  \caption{Quantities that are fixed in the ET response to VPD
    (relative to those in Table \ref{param_varying}).}
  \label{param_varying}
  \centering
  \begin{tabular}{l c c}
    \hline
    Symbol & units & value \\
    \hline
    \input{param_fixed.tex}
    \hline
  \end{tabular}
\end{table}

\section{Results}
\label{results}

By varying four parameters (g$_a$, T, uWUE, g$_1$) at three different
values (Table \ref{param_varying}) we generate nine different values
for the scaling term, nine different curves (as a function of VPD) for
the sign term, and 81 different curves for the ET response to VPD
($\frac{\partial \; ET}{\partial \; VPD}$). Examining a subset of nine
of the curves for the ET response to VPD, defined by the minimum, median
and maximum of for each of the scaling and sign term, it is clear
that the range of responses includes those where ET almost always
decreases with increasing VPD (water conservative), ET almost always
increases with increasing VPD (water intensive), and intermediate
values where ET will increase or decrease with VPD depending on the
atmospheric demand for water (Figure \ref{full}).

\begin{figure}
  \centering \includegraphics{./fully_idealized.pdf}
  \caption{The functional form of the $\frac{\partial \; ET}{\partial
      \; VPD}$ for minimum, median and maximum values of both the sign
    term and the scaling term.}
  \label{full}
\end{figure}

Examining the sign term independent of the scaling term indicates the
dependence of water conservative versus water intensive response on
the main plant parameters, g$_1$ and uWUE (Figure \ref{sign}). Higher
g$_1$ and uWUE shift the curve towards increasing ET responses with
VPD (water intensive), and with increasing g$_1$ also leading to a
higher non-linear VPD dependence of the response.

\begin{figure}
  \centering \includegraphics{./fully_idealized)sign.pdf}
  \caption{The functional form of the $\frac{\partial \; ET}{\partial
      \; VPD}$ evaluated at the median value of the scaling term, for
    varying values of g$_1$ and uWUE as given in Table \ref{param_varying}.}
  \label{sign}
\end{figure}

Examining the scaling term independent of the sign term shows how
aerodynamic conductance (communication between the atmosphere and the
surface) and temperature (through $\Delta$ controls the efficiency of
energy conversion to latent heat), amplify or supress the plant
physiological response represented in the sign term. Both lower
temperatures and higher aerodynamic conductance lead to amplified ET
response to VPD, with the variabilty of aerodynamic conudcance
resulting in a slightly higher variability of ET response relative to
temperature, which is still an important factor.

\section{Discussion}
\label{discussion}


\small
% \begin{linenomath*}
  \begin{equation}
    VPD_{crit} = \frac{R_{air}}{4 c_p} \left( \frac{\gamma c_a}{1.6\; R \;  uWUE} + \sqrt{\frac{\gamma c_a}{1.6\; R \;  uWUE}\left( \frac{\gamma c_a}{1.6\; R \;  uWUE} + 8 g_1 \frac{c_p}{R_{air}}\right)} - 4 g_1 \frac{c_p}{R_{air}} \right),
    \label{vpd_min_et}
  \end{equation}
% \end{linenomath*}
  \normalsize

Differences in $VPD_{crit}$ are exclusively determined by uWUE and the
slope parameter ($g_1$) related to the plant functional type. A larger
uWUE means a smaller $VPD_{crit}$, and an ET response to increases VPD
that is more likely to be positive. At first glance this result is
somewhat counter-intuitive; we expect that plants with a higher water
use efficiency would be more water conservative. However, in reality
uWUE determines how $WUE$ changes with VPD:

\[WUE = \frac{GPP}{ET} = \frac{uWUE}{\sqrt{VPD}}.\]
\[\frac{\partial \; WUE}{\partial \; VPD} = -\frac{uWUE}{2 \;
    VPD^{3/2}}\]

So, plants with a higher uWUE will have a greater \textit{decrease} in
ecosystem-scale $WUE$ in response to increases in VPD. This decrease
in $WUE$ causes more water loss per unit carbon gain, and explains the
relationship between high uWUE and high likelihood of increases of ET
in response to increasing atmospheric drying (increases in VPD).

A tendency towards increasing ET response with increasing VPD can also
be caused by a high slope parameter ($g_1$), characteristic of plants
that at the leaf scale are more willing to trade water for access to
atmospheric CO$_2$. Plants that are less conservative will be thus be
more likely to increase ET with increasing VPD. Both the
aforementioned effects (large uWUE, $g_1$) can amplify each other, and
generally conspire to shift the sign term towards a positive value for
a given PFT.

This effect of uWUE and $g1$ on the sign term is most apparent by
comparing two extreme PFTs: water intensive crops (CRO) and water
conservative closed shrub (CSH). CRO has higher slope parameter and a
slightly higher uWUE ($g_1 = 140.7$ Pa$^{1/2}$; $2.85$ $\mu$-mol [C]
Pa$^{0.5}$ J$^{-1}$ [ET]) compared to CSH ($g1 = 75.1 \, Pa^{1/2}$,
$uWUE=2.82$ $\mu$-mol [C] Pa$^{0.5}$ J$^{-1}$ [ET]). These differences
in PFT parameters cause opposite ET responses to changes in VPD
between CRO and CSH. ET theoretically always decreases with increasing
VPD for the more water conservative CSH, while ET frequently increases
with increasing VPD for the more water intensive CRO (Fig.
\ref{idealized_sign}). CROs evolved or were bred to prioritize GPP and
yield and are thus not water conservative. They are very willing to
trade water for photosynthesis and productivity, despite changes in
VPD, while CSH are very unwilling to trade water for more
photosynthesis.

\begin{figure}
  \centering \includegraphics{./idealized_sign.pdf}
  \caption{The functional form of the sign term, with $\sigma$ held
    fixed at 1, and all terms except VPD set to PFT averages. For
    comparison, the observed range of VPD for each PFT is plotted
    below the x-axis. Stars denote 25th, 50th, and 75th percentiles,
    and the range of the line spans the 5th-95th percentiles of
    observed VPD. Vertical black lines denote the location of
    $VPD_{crit}$ for each PFT, with the exception of CSH and WSA, for which
    $VPD_{crit}$ is off-scale.}
  \label{idealized_sign}
\end{figure}


As expected, the slope parameter ($g_1$) is a primary determinant of
the VPD dependence for the sign term shown in Fig.
\ref{idealized_sign}. Plants that are more conservative (small $g_1$)
will tend to reduce ET with increasing VPD, and will be very effective
at reducing ET, especially at low VPD. However, at very high VPD,
gradients in vapor pressure at the leaf scale will become very strong
as stomata reach their limits of closure in response to VPD
(parameterized with $g_1$). As a result, ET response will begin to
asymptote towards a constant ecosystem-scale values as leaf-scale
response to VPD asymptotes towards zero.  Therefore, plants with a low
$g_1$ will have the largest VPD dependence of ET response because the
difference in ET response at low VPD (leaf stomatal response
dominates) and high VPD (VPD gradient dominates) is largest. This is
apparent in the strong VPD dependence of CSH, which has the lowest
slope parameter ($g_1=75.1$ Pa$^{1/2}$) (Fig. \ref{idealized_sign}).

To summarize our theoretical insights (Fig. \ref{idealized_sign} and
Table \ref{vpd_crit}), CROs are the least water conservative and have
the strongest overall tendency to increase ET with increasing VPD,
while CSH are the most water conservative and have the strongest
tendency to decrease ET with increasing VPD, as well as the strongest
VPD dependence of response. Fig. \ref{idealized_sign} clearly shows,
according to our theory, that for all PFTs except for crops there is
frequent occurrence of a negative (plant dominating) ET response to
increases in VPD. Therefore, plants are able in most atmospheric
conditions to reduce ET in response to increased VPD and thus to
reduce water loss. To better illustrate this, the ranges of observed
environmental VPDs at the FLUXNET sites are plotted parallel to the
x-axis. For CSH and WSA, VPD is always less than VPD$_{crit}$ (off
scale) so that the plant response dominates in typical environmental
conditions, emphasizing the water conservative strategy of those
plants. For CRO on the other hand, VPD is higher than VPD$_{crit}$ for
more than 50\% of observations, emphasizing that those plants operate
with an aggressive water usage strategy, are water intensive and were
actually engineered for photosynthesis rather than water saving. For
DBF, EBF, MF, GRA, and SAV more than half of the observed VPD are less
than VPD$_{crit}$, i.e. in conditions where plant response
dominates. SAV has a more water conservative response than the forest,
grass, and crop plan types, but still responds by increasing ET with
increasing VPD for about a quarter of observations, due to the high
aridity (VPD) of the SAV ecoclimate. It is also important to note that
for all PFTs, even when atmospheric demand dominates, ET response to
VPD is still far more negative than it would be for potential
evaporation $\partial PET/\partial VPD$, i.e. atmospheric demand only,
emphasizing that there is still a strong regulation of evaporative
flux by stomata and though the plant xylem. The sign term in the PET
case would just be a constant ($\frac{c_p}{R_{air}} \approx 3.5$),
which is far larger than any part of the curves for any PFT. Plants
are always regulating water exchange from the land surface, even when
they reach the limits of they ability to do so.

\subsection{Functional Form of the Scaling Term}
\label{scale_term}
While the above discussion of the sign of
$\frac{\partial \; ET}{\partial \; VPD}$ is important to answer our
question of when ET response increases or decreases with VPD,
understating the overall magnitude of the ET response is important to
soil-plant-atmosphere water budgeting. So we now more closely examine
the terms that affect how the sign term is scaled:

\begin{equation}
  \frac{g_a \; P}{T(\Delta + \gamma)}.
\end{equation}

$\frac{P}{T}$ is an air-density term, which varies little compared to
aerodynamic conductance and Clausius-Clapeyron ($\Delta$). The
psychometric constant ($\gamma$) is also relatively constant, so the
scaling term should be primarily a function of aerodynamic conductance
and temperature, through the Clausius-Clapeyron relationship
$\Delta$. This is as expected, given that the aerodynamic conductance
represents the efficiency of exchange between the surface and the
atmosphere. As aerodynamic conductance increases, any plant response
will be communicated more strongly to the atmosphere (and vice-versa).

$\Delta$'s presence in the scaling term also matches physical
intuition. $\Delta$ (and also the approximately constant $\gamma$)
control the efficiencies with which surface energy is converted to
latent and sensible heat \citep{Monteith_1965}. The functional from of
$\Delta$ will be the same across PFTs, but the temperature range may
vary slightly. In contrast, aerodynamic conductance will vary strongly
with PFT due to the importance of surface roughness for aerodynamic
conductance. So most of the differences in scaling between PFT should
be in the aerodynamic conductance term.

The control of the scaling term variability between PFTs by
aerodynamic conductance is confirmed by data (Fig.
\ref{scale_vary}). Differences between PFT are almost entirely due to
differences in aerodynamic conductance, rather than differences in
observed temperature ranges. The scaling term for the tree PFTs (DBF,
EBF, ENF, MF) is generally about double the scaling terms for other
PFTs which have lower surface roughness and generally smaller
aerodynamic conductance (GRA, CSH, CRO). The savannah (WSA, SAV) PFT's
scaling is somewhere between GRA, CSH, and CRO, and DBF, EBF, ENF, and
MF, due to higher variability and surface roughness.

\begin{figure}
  % \centering
  \centerline{\includegraphics[width=1.2\textwidth]{./idealized_scale.pdf}}
  \caption{Primary sources of variability for the scaling term, as a
    function of PFT. The 5th-95th percentile range of temperature is
    plotted at the 5th, 25th, 50th, 75th, and 95th percentiles of
    aerodynamic conductance, as observed for each PFT.}
  \label{scale_vary}
\end{figure}

Within each PFT, the scaling term variability is controlled both by
environmental temperature and aerodynamic conductance variability
(Fig. \ref{scale_vary}). While the observed variability of the
aerodynamic conductance contributes more to the scaling term
variability than temperature, the temperature contribution is
non-negligible. Specifically, the scaling term is generally larger at
low temperatures when latent heat is relatively inefficient at moving
energy away from the surface. This effect amplifies the role of
aerodynamic conductance variability at low temperatures.

To summarize, variability between PFTs is mostly controlled by
systematic differences in aerodynamic conductance, due to differences
in surface roughness between each PFT, and possibly to a lesser extent
wind conditions. In contrast, variability within PFT is also
controlled by temperature, through Clausius-Clapeyron. But,
aerodynamic conductance variability generally impacts the scaling term
more than temperature, even within PFTs.



\subsection{Functional form of ET dependence on VPD and its relation
  to the VPD exponent}
\label{functional_form}

The theory described in Sect. \ref{sign_term} indicates that for a
given $uWUE$ and $g_1$, the ET dependence on VPD should be concave
upward, which is confirmed by eddy covariance data across most PFTs. In other words, there should be some local minimum in ET at a
critical VPD$_{crit}$, assuming the scaling and plant terms  (e.g. aerodynamic
conductance, $\Delta$, $g_1$ and $uWUE$) are held fixed. This result warrants further
investigation, because to our knowledge no one has derived the
theoretical ecosystem-scale relationship between ET and VPD while
controlling for other environmental conditions. In particular, from
personal communication, there is an apparent lack of consensus over
whether the shape of the ET-VPD curve should be concave upward (our
result) or concave downward in the absence of dramatic water stress. Given that understanding the ET-VPD
relationship of the one-way plant response is fundamental to
hypothesizing about any feedbacks between the land surface and the
atmosphere, we analyze why our derived ET-VPD relationship is concave
upward, particularly with respect to the exponent of VPD dependence in
$uWUE$ and the Medlyn unified stomatal conductance model.

There is a theoretical basis for the square root VPD dependence in
both the stomatal conductance model and $uWUE$ based on the assumption
that stomata behave to maximize carbon gain while minimizing water
loss, which observations also generally support \citep{Lloyd_1991,
  MEDLYN_2011, Lin_2015, Zhou_2014, Zhou_2015, Medlyn_2017}. However,
some purely empirical results that fit the exponent of the VPD
dependence to data have shown that it may vary slightly from 1/2,
suggesting that stomata, as well as ecosystem-scale quantities based
on stomata theory, may not always function optimally \citep{Zhou_2015,
  Lin_2018}. Specifically with regards to $uWUE$, one would not expect
that this ecosystem scale WUE quantity will respond to VPD exactly
analogously to stomata. Direct soil evaporation's contributions to ET
should shift the exponent of the VPD dependence, especially at
conditions of low GPP when we would expect a systematically larger
portion of direct soil evaporation contributions to ET, because we
would also expect lower amounts of transpiration at low GPP. \citet{Zhou_2015}'s
results corroborate this: they found a mean empirically fit
exponential VPD dependence of 0.55, varying slightly from the
theoretically optimal value of of 0.5 for AmeriFlux
sites. Interpreting \citet{Lin_2018}'s results, which also show
variance in the empirical exponent of the VPD dependence of the
stomatal conductance model, is more difficult as \citet{Lin_2018} do
not handle GPP/A dependence of stomatal conductance in a directly
analogous manner to the optimal theory in \citet{MEDLYN_2011} and
\citet{Medlyn_2017}. Regardless, given that these recent results on
the relationship between VPD, GPP, and ET \citep{MEDLYN_2011,
  Zhou_2014, Zhou_2015, Medlyn_2017} form the backbone of our analysis
and are what allowed us to derive an explicit ET expression for the
first time (Eq. (\ref{et})), we will analyze if and how assumptions
about the exponent of the VPD dependence impacts the shape of the
ET-VPD dependence. This analysis is also important to understand whether the choice of stomatal conductance model alters the fundamental behavior of the ET-VPD relationships, as many commonly used models  utilize a VPD exponent other than the 1/2 suggested by optimal theory \citep[e.g. ][ which uses an exponent of 1]{Leuning_1990}.

By introducing $n$ and $m$ we can free our stomatal
conductance model from assumptions about VPD dependence:
% \begin{linenomath*}
  \begin{equation}
    g_s = \frac{R \, T}{P} 1.6 \left(1 + \frac{g*}{VPD^m}\right) \frac{*WUE \; ET}{c_a \; VPD^n},
    \label{m_n}
  \end{equation}
% \end{linenomath*}
where:
\[*WUE = \frac{GPP}{ET}VPD^n,\] and $g*$ is a generic slope parameter
of units $VPD^m$. To determine how the exponent $n$ and $m$ alter the
shape of the ET-VPD dependence we find the roots of the second
derivative of ET, using Eq. (\ref{m_n}) for stomatal conductance
($g_s$), with respect to VPD: \input{d2_solutions.tex} With this
result we have defined the family of curves separating concave up from
concave down ET solutions (Fig. \ref{concave}). These curves are
only functions of the exponent of the VPD dependence and a quantity we
call non-dimensional VPD ($VPD^m/g*$). Several important
relations reveal themselves from Eq. (\ref{curves}):

\begin{figure}
  \centering
  \centerline{\includegraphics[width=0.75\textwidth]{./concave.pdf}}
  \caption{ Solutions corresponding to inflection points between
    concave up and concave down ET-VPD curves (Eq. (\ref{curves}))
    for three specific scenarios. Solutions are defined in terms of a
    non-dimensional VPD ($VPD^m/g_*$), but to aide physical
    interpretation the horizontal axis is additionally provided in terms of dimensionalized VPD assuming $m=1/2$ and
    $g_*=110\; Pa^{1/2}$ (average of all PFT $g_1$). The vertical axis
    has a different interpretation depending on the solution
    curve. For the blue line ($m$ varying), it corresponds to $m$,
    for the orange line ($n$ varying) it corresponds to $n$, and for
    the green line it corresponds to the value of both $n$ and $m$
    ($n=m$). Regions of the parameter space that correspond to
    concave up and concave down results are labeled.}
  \label{concave}
\end{figure}

\begin{itemize}
  \item For optimal behavior (n, m = 1/2) the ET-VPD curve will be
    concave up regardless of the magnitude of the plant constants
    $g_1$ and $uWUE$. Therefore, the general concave up nature of our
    results, given an assumption of optimal behavior, is insensitive
    to plant type.
  \item For all physically possible exponents of VPD dependence ($n,
    m$), whether the solution is concave up or concave down does not
    depend on $uWUE$.
  \item In general, increasing the exponent of VPD dependence
    increases the likelihood of a concave down result. Additionally, as
    the exponent of VPD dependence increases from the optimum value of
    1/2, whether the curve is concave upward or concave downward
    becomes a function of the plant specific slope parameter $g_*$,
    through non-dimensional VPD ($VPD^m/g_*$). Because the exponent of
    the VPD dependencies is capable of altering the fundamental shape
    of ET-VPD dependence, future research investment in understanding
    the exact VPD dependence of stomatal conductance, and further
    reconciliation of empirical and theoretical stomatal and
    ecosystem behavior should be prioritized.
\end{itemize}
While it is possible that in the future some other form of
VPD dependence is derived, at present \cite{MEDLYN_2011} and
\cite{Zhou_2014} firmly established n=m=1/2 as the most likely
candidate given current theory and empirical data. Additionally, we
argue that a concave up result matches physical intuition more than a
concave down result. Plants must maintain nutrient and sugar transport
through the phloem and xylem. To accomplish this, stomata must remain
slightly open \citep{De_2013, Nikinmaa_2013, Ryan_2014}. Furthermore,
even if complete stomatal closure were possible, cuticular water loss
and [at the ecosystem-scale] direct soil evaporation are still sources
of ET which increase with VPD, independent of stomatal
closure. Therefore, in the limit as VPD becomes large and we assume
plants are exercising all strategies to reduce ET, any further increase
in VPD should result in an increase in ET through cuticular water loss
and/or direct soil evaporation. This inevitable transition from
conditions when stomata respond strongly to VPD to conditions when
stomata response is asymptoting towards full closure would cause a
concave up ET-VPD curve, which is matched by the theory. In short,
plant response becomes more limited as VPD increases, while
atmospheric demand monotonically increases with VPD, leading to the
 result that atmospheric demand dominates plant response when
atmospheric demand is high.

This analysis allows us to understand the theoretical shape of the ET
response to VPD with environmental conditions held
fixed. Accomplishing this with purely statistical
methods applied to flux observations would be very difficult, given
the relatively fast time scale of plant response and the
non-stationarity of [solar forced] environmental conditions over the
relatively coarse (half hourly) flux estimates (which is required to
obtain robust eddy-covariance statistics). Our results on the shape of
the ET-VPD curve with environmental conditions held
fixed can be built upon with future work examining how changes in VPD
and environmental conditions (e.g. soil water storage) feedback upon
one another. In the soil water storage example, over very long time
scales extremely high VPD perturbations coupled with no precipitation
could result in decreases in soil water storage such that water
becomes limiting. This could be represented by an extension of our
framework in which uWUE is allowed to decrease with decreasing SWC, as
observed in Sect. \ref{swc_section}. Here, we focus our results by
assuming constant PFT-wide conditions to build baseline intuition for
ET-VPD dependence. For most PFTs, the theory with plant function held
fixed matches the leading order behavior of the observations where
plant function varies (Sect. \ref{testing}).
