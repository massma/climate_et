
\section{Methods}
\label{methods}
The Penman-Monteith equation \citep [hereafter PM,][]{Penman_1948,
  Monteith_1965} estimates ET as a function of observable atmospheric
variables and surface conductances:
% \begin{linenomath*}
  \begin{equation}
    \label{orig_pen}
    ET = \frac{\Delta R_{net} + g_a \rho_a c_p VPD}{\Delta + \gamma(1 + \frac{g_a}{g_s})},
  \end{equation}
% \end{linenomath*}
where $\Delta$ is the change in saturation vapor pressure with
temperature, given by Clausius-Clapeyron ($\frac{d \; e_s}{d \; T}$),
$R_{net}$ is the net radiation minus ground heat flux, $g_a$ is
aerodynamic conductance, $\rho_a$ is air density, $c_p$ is specific
heat of air at constant pressure, $\gamma$ is the psychometric
constant, and $g_s$ is the stomatal conductance (Table
\ref{definitions}).

\begin{table}
  \caption{Definition of symbols and variables, with citation for how
    values are calculated, if applicable.}
  \label{definitions}
  \centering \footnotesize
  \begin{tabular}{l c c c}
    \hline
    Variable & Description & Units & Citation \\
    \hline
    $e_s$  & saturation vapor pressure & Pa  & - \\
    $T$  & temperature  & K & - \\
    $P$  & pressure & Pa  & - \\
    $\Delta$  & $\frac{\partial e_s}{\partial T}$ & Pa K$^{-1}$ & - \\
    $R_{net}$  & net radiation at land surface minus ground heat flux & W m$^{-2}$   & - \\
    $g_a$  & aerodynamic conductance & m s$^{-1}$  & \citet{Shuttleworth_2012} \\
    $\rho_a$  & air density & kg m$^{-3}$  & - \\
    $c_p$  & specific heat capacity of air at constant pressure & J K$^{-1}$ kg$^{-1}$ & - \\
    $VPD$  & vapor pressure deficit & Pa  & - \\
    $\gamma$  & psychometric constant & Pa K$^{-1}$   & - \\
    $g_{s-leaf}$  & leaf-scale stomatal conductance & m s$^{-1}$  & \citet{MEDLYN_2011} \\
    $g_{s}$  &  stomatal conductance & m s$^{-1}$
                                   & \citet{Medlyn_2017} \\
    $g_{1-leaf}$  & leaf-scale slope parameter & Pa$^{0.5}$
                                   & \citet{MEDLYN_2011} \\
    $g_{1}$  & ecosystem-scale slope parameter & Pa$^{0.5}$ & \citet{Medlyn_2017} \\
    $R$ & universal gas constant & J mol$^{-1}$ K$^{-1}$ & - \\
    $R_{air}$ & gas constant of air & J  K$^{-1}$ kg$^{-1}$ & - \\
    $\sigma$ & uncertainty parameter & -& - \\
    $c_a$ & CO$_2$ concentration & $\mu$ mol CO$_2$ mol$^{-1}$ air& - \\
    $\lambda$ & marginal water cost of leaf carbon & mol H$_2$O mol$^{-1}$ CO$_2$ & - \\
    $\Gamma$ & CO$_2$ compensation point & - & - \\
    $\Gamma^*$ & CO$_2$ compensation point without dark respiration & - & - \\
    \hline
  \end{tabular}
\end{table}


\citet{MEDLYN_2011} developed a model for stomatal conductance ($g_s$)
by combining an optimal photosynthesis theory \citep{Cowan_1977} with an empirical approach, which describes the
dependence of $g_s$ to VPD. This resulted in the following model for
leaf-scale stomatal conductance:

% \begin{linenomath*}
  \begin{equation}
    g_{s-leaf} = g_0 + 1.6 \left(1 +
      \frac{g_{1-leaf}}{\sqrt{VPD}}\right) \frac{A}{c_a},
    \label{leaf_medlyn}
  \end{equation}
% \end{linenomath*}
where $g_{1-leaf}$ is a leaf-scale ``slope'' parameter, A is the net
CO$_2$ assimilation rate, and $c_a$ is the atmospheric CO$_2$
concentration at the leaf surface. \cite{MEDLYN_2011} relate the slope
parameter ($g_{1-leaf}$) to physical parameters as:
% \begin{linenomath*}
  \label{slope}
  \begin{equation}
    g_{1-leaf} = \sqrt{\frac{3 \, \Gamma^* \, \lambda}{1.6}},\footnote{Note this expression has units of of (mmol mol$^{-1}$)$^{1/2}$, but this can be converted to Pa$^{1/2}$ using the ideal gas law.}
  \end{equation}
% \end{linenomath*}

where $\Gamma^*$ is the CO$_2$ compensation point for photosynthesis
(without dark respiration), and $\lambda$ is the marginal water cost
of leaf carbon
($\frac{\partial \; \text{transpiration}}{\partial \; A}$). So,
$g_{1-leaf}$ is a leaf-scale term reflecting the trade-off of water for
carbon uptake. The higher $g_{1-leaf}$, the more open the stomata and
the more they release water in exchange for carbon.


The Medlyn model for stomatal conductance has been shown to behave
very well across PFTs \citep[][]{Lin_2015}, and has been successfully
adopted to ecosystem scale analysis in \citet{Medlyn_2017}. In units
of m s$^{-1}$, the ecosystem scale stomatal conductance is:

% \begin{linenomath*}
  \begin{equation}
    g_s = \frac{R \,T}{P} \; 1.6 \left(1 + \frac{g_1}{\sqrt{VPD}}\right) \frac{GPP}{c_a},
    \label{medlyn}
  \end{equation}
% \end{linenomath*}

  where GPP is the ecosystem scale gross primary production, and $g_1$
  is an ecosystem scale analogue to $g_{1-leaf}$. We solve for $g_1$
  following \citet{Medlyn_2017} (Eq. (5)), and take the median $g_1$
  value to be representative of each PFT (Table \ref{pft}) instead of
  the mean to avoid extra weighting of rare outliers.

\begin{table}
  \caption{Plant functional types, their abbreviation, calculated
    Medlyn coefficient, and calculated uWUE. uWUE from
    \citet{Zhou_2015}, including the observed standard deviation, is
    shown for comparison. Note that uWUE from \citet{Zhou_2015} is
    calculated from a different set of sites, and that units are
    converted such that the quantities work with Equations 1-8 and the
    variables defined Table \ref{definitions}.}
  \small
  \label{pft}
  \centering
  \begin{tabular}{l c c @{\qquad} c c}
    \hline
    \multirow{2}[3]{*}{Abbreviation} & \multirow{2}[3]{*}{PFT} & \multirow{2}[3]{*}{$g_1$ (Pa$^{0.5}$)} & \multicolumn{2}{c}{uWUE ($\mu$-mol [C] Pa$^{0.5}$ J$^{-1}$ [ET])}  \\
    \cmidrule{4-5}

                                     & & & fitted & \citet{Zhou_2015} \\

    \hline
    \input{pft_params.tex}
    \hline
  \end{tabular}
\end{table}

While Eq. (3) can be used in PM (Eq. (1)), it will make
analytical work with the function intractable because $GPP$ is
functionally related to ET itself. Additionally, a perturbation to VPD
should induce a physiological plant response that will alter GPP and
cause an indirect change in stomatal conductance, in addition to the
direct effect of VPD in Eq. (\ref{medlyn}). Therefore, in order to
derive the response of ET to VPD, we must account for the functional
relationship between GPP, ET, and VPD, and its effect on stomatal
conductance. We can use aforementioned semi-empirical results of
\citet{Zhou_2015} as a tool to approach this
problem. \citet{Zhou_2015}, showed that  underlying Water Use
Efficiency (uWUE):

% \begin{linenomath*}
  \begin{equation}
    uWUE = \frac{GPP \cdot \sqrt{VPD}}{ET}
    \label{uwue}
  \end{equation}
% \end{linenomath*}

is relatively constant across time and moisture conditions within a
plant functional type, and correctly captures a constant relationship
between GPP, ET and VPD over a diurnal cycle \citep{Zhou_2014}. The
theoretical derivation of the square root VPD dependence in $uWUE$
leverages the same assumptions used in \cite{MEDLYN_2011} to derive the
square-root VPD dependence of the stomatal conductance model (Eq.
(\ref{medlyn})).  We can use uWUE to remove the $GPP$ dependence from
$g_s$ in a way that makes PM analytically tractable:

% \begin{linenomath*}
  \begin{equation}
    g_s = \frac{R \, T}{P} 1.6 \left(1 + \frac{g_1}{\sqrt{VPD}}\right) \frac{uWUE \; ET}{c_a \; \sqrt{VPD}}.
    \label{new_g_s}
  \end{equation}
% \end{linenomath*}

Plugging Eq. (\ref{new_g_s}) into Eq. (\ref{orig_pen}) and
rearranging gives a new explicit expression for PM, in which
dependence on $GPP$ is removed:

% \begin{linenomath*}
  \begin{equation}
    ET = \frac{\Delta R_{net} + \frac{g_a\; P}{T} \left( \frac{ c_p VPD}{R_{air}} -  \frac{\gamma c_a \sqrt{VPD} }{ R \; 1.6\; \text{ uWUE } (1 + \frac{g_1}{\sqrt{VPD}})} \right) }{ \Delta + \gamma}
    \label{et}
  \end{equation}
% \end{linenomath*}

By accounting for photosynthesis changes in ecosystem conductance, with
Eq. (\ref{et}) we have derived for the first time, using recent
results \citep[][]{MEDLYN_2011, Zhou_2014, Zhou_2015, Medlyn_2017}, ET
explicitly as function of environmental variables and two
plant-specific constants, the slope parameter ($g_1$), and uWUE, both
reflecting water conservation strategy. The slope parameter is related
to the willingness of stomata to trade water for CO$_2$ and to keep
stomata open. uWUE is a semi-empirical ecosystem-scale constant
related to how WUE changes with VPD (specifically $VPD^{-1/2}$). It is
also roughly proportional to physical constants:

\[uWUE \appropto \sqrt{\frac{c_a - \Gamma}{1.6 \lambda}},\]

where $\Gamma$ is the CO$_2$ compensation point \citep[Eq. (5)
in][]{Zhou_2014}. So uWUE is related to atmospheric CO$_2$
concentration and compensation point, and is inversely proportional to
the marginal water cost of leaf carbon.


With Equation \ref{eq} we can take the partial derivative with respect
to VPD:

% \begin{linenomath*}
  \begin{equation}
    \frac{\partial \;  ET}{\partial \; VPD} = \frac{2\; g_a \;
      P}{T(\Delta + \gamma)}   \left(\frac{ c_p}{R_{air}} -
      \frac{\gamma c_a }{1.6 \; R\; \; \text{ uWUE }} \left(
        \frac{2 g_1 + \sqrt{VPD}}{2 (g_1 + \sqrt{VPD})^2}\right)
    \right),
    \label{d_et}
  \end{equation}
% \end{linenomath*}

  deriving an analytical framework for ecosystem response to
  atmospheric demand perturbations with environmental conditions held
  fixed. There are a few subtleties to taking the derivative in
  Eq. (\ref{d_et}): $\Delta$ ($\frac{d e_{s}}{d T}$) and $VPD$ are
  functionally related, so while taking the derivative we evaluate
  $\frac{\partial \; ET}{\partial \; VPD} = \frac{\partial \; ET}
  {\partial \; e_s} \frac{\partial \; e_s}{\partial \; VPD}
  \Big|_{\text{RH fixed}} + \frac{\partial \; ET}{\partial \; RH}
  \frac{\partial \; RH}{\partial \; VPD} \Big|_{\text{$e_s$
      fixed}}$. $RH$ and $e_s$ are assumed to be approximately
  independent, which is supported by data (not shown).

  This derivation relied either implicitly or explicitly on several
  assumptions. First, we assume that VPD at the leaf surface is the
  same as VPD at measurement height; physically this implies that
  leaves are perfectly coupled to the atmosphere. In reality, for some
  conditions and plant types the leaves can become decoupled from the
  boundary layer \citep{De_2017, Medlyn_2017}. Therefor, our
  derivation will be most applicable in times like the growing season,
  when relatively high insolation induces instability and convective
  boundary layers, and we would expect the surface to be generally
  well coupled. An additional assumption in the formulation of $uWUE$
  \citep{Zhou_2014, Zhou_2015} and \citet{Medlyn_2017}'s stomatal
  conductance model is that direct soil evaporation (E) contributions
  to ET remain small relative to transpiration (T). This should be
  more true during the growing season. The ratio of E to T may
  increase immediately after rainfall events due to high soil moisture
  and ponding, but VPD is generally low anyways during these
  times. However, some plant types allow for systematically larger
  contributions of E in ET, particularly those with sparse canopies
  and smaller relative amounts of transpiration. We therefore might
  expect that the theory will be most applicable to forest PFTs, which
  will be most strongly coupled to the boundary layer due to larger
  surface roughness, and will also generally have the highest ratios
  of transpiration to evaporation. Finally, we assume that $g_1$ and
  $uWUE$ are constant with respect to the conceptual VPD
  perturbation. Both quantities have been shown to be relatively
  constant with respect to changes in
  % TODO: make this a general discussion on partial derivatives and
  % what things are constant/how to interpret?
  VPD \cite{Franks_2017, Zhou_2014}. These parameters will however
  vary with plant species and characteristics \citep[e.g. wood density,
  ]{Lin_2015}, as well as environmental conditions including soil
  water content and temperature.
  % TODO: write exact parameter ranges we use, and introduce
  % scaling/sign term here?
  % \item

To account for both this variability in plant parameters and the
environmental, we will systematically analyze how the ET reponse to
VPD (Equation \ref{d_et}) varies. Equation \ref{d_et} includes a
``sign'' term  that determines the sign of the response in addition to
magnitude:

\begin{equation}
  \label{sign}
  \frac{c_p}{R_{air}} - \frac{\gamma c_a }{1.6 \; R\; \text{ uWUE }} \left( \frac{2 g_1 + \sqrt{VPD}}{2 (g_1 + \sqrt{VPD})^2}\right),
\end{equation}

and a ``scaling'' multiplying the ``sign'' term:

\begin{equation}
  \frac{g_a \; P}{T(\Delta + \gamma)}.
\end{equation}

In the ``sign term'' most of quantities are relatively constant,
except for VPD and the plant parameters g$_1$ and uWUE. In the scaling
term, most of the terms are relatively constant with the exception of
aerodynamic conductance ($g_a$) and temperature (especially its effect
on $\Delta$). To determine the range of probable ET reponses to VPD we
will systematically vary this parameters according to Table
\ref{param_values}. Environmental parameters ($g_a$, $T$) are varied
according to ranges we would expect for growing season conditions
across a range of plant types (canopy height is a determinant $g_a$),
and plant parameters (g$_1$, uWUE) are varied according to ranges
established in previous literature \citep{Zhou_2015,
  Medlyn_2017}. Using this previous literature we can connect the
effect of varying plant parameters to specific plant types and
characteristics. Lastly, all code and data used in this analysis,
including those used to generate the figures and tables, are publicly
available at \url{https://github.com/massma/climate\_et}.
%\hyperref[https://github.com/massma/climate\_et]{https://github.com/massma/climate_et}.

\section{Results and Discussion}
\label{results}

By construction, the variability in the $\sigma$ term (Eq.
(\ref{sigma})) contains all model and observational uncertainties. For
an observation that perfectly matches our model and constant uWUE
assumption, $\sigma$ will be one. Therefore, for our assumptions and
framework to be reasonable $\sigma$ should be close to 1. An
additional concern is that $\sigma$ may in fact be correlated with
$VPD$, in which case the dependence would need to be accounted for
when taking the derivative. Fortunately, there is a very weak
dependence of $\sigma$ on VPD in their joint distribution, and
$\sigma$ is indeed close to unity i.e. $O(1)$ (Fig.
\ref{joint_vpd_sigma}). Given this weak dependence and the
distribution of $\sigma$ we have confidence in our model framework and
the data quality.

\begin{figure}[h]
  \centering\includegraphics[width=0.75\textwidth]{./joint_vpd_sigma.pdf}
  \caption{The joint distribution of $VPD$ and $\sigma$, with outliers
    removed (defined as lowest and highest 5\% of $\sigma$). $\sigma$
    exhibits a weak dependence on $VPD$, and $\sigma$ is $O(1)$ for
    the bulk of the observations.}
  \label{joint_vpd_sigma}
\end{figure}

Before calculating the sensitivity of ET to VPD, we will consider the
functional form of Eq. (\ref{d_et}). There are two main terms: a
``scaling'' term, which modifies the magnitude but not the sign of the
ET response to VPD ($\frac{\partial \; ET}{\partial \; VPD}$):


and a ``sign'' term, which determines whether ET increases or
decreases with VPD (i.e. atmospheric demand driven or physiologically
controlled):

All variables are positive, so the relative magnitude between the
first term and the second term in the sign term (Eq. (\ref{sign}))
will determine whether ET increases or decreases with increasing
VPD. If the second term is larger then plant control dominates and ET
decreases with increasing VPD. However, if the first term is larger,
then atmospheric demand dominates and ET increases with increasing
VPD.

\subsection{Functional Form of the Sign Term}
\label{sign_term}
First, we explore the variables within the sign term to gain better
intuition on the driver of either the increase or reduction of ET with
VPD. CO$_2$ concentration ($c_a$) and the psychometric constant
($\gamma$) are relatively constant over the dataset considered here so
that the variability is dominated by $\sigma$ and $VPD$. uWUE could
vary with soil moisture but has been shown to be relatively constant
\citep{Zhou_2015}. This then means that the sign term only depends on
VPD for a given PFT and is approximately just a function of $VPD$. We
can further determine a critical threshold separating an increase from
a decrease in ET, i.e. the threshold $VPD_{crit}$ such that the
derivative vanishes $\frac{\partial \; ET}{\partial \; VPD} = 0$:
\small
% \begin{linenomath*}
  \begin{equation}
    VPD_{crit} = \frac{R_{air}}{4 c_p} \left( \frac{\gamma c_a}{1.6\; R \;  uWUE} + \sqrt{\frac{\gamma c_a}{1.6\; R \;  uWUE}\left( \frac{\gamma c_a}{1.6\; R \;  uWUE} + 8 g_1 \frac{c_p}{R_{air}}\right)} - 4 g_1 \frac{c_p}{R_{air}} \right),
    \label{vpd_min_et}
  \end{equation}
% \end{linenomath*}
  \normalsize

  noting that $VPD_{crit}$ mostly depends on the PFT parameters uWUE
  and $g_1$, and only varies weakly with climate as most other
  parameters related to the environment are nearly constant. The
  calculated value of $VPD_{crit}$ for each PFT is shown in Table
  \ref{vpd_crit}. For any values of $VPD$ less than $VPD_{crit}$, ET
  will decrease with increasing VPD
  ($\frac{\partial \; ET}{\partial \; VPD} < 0$), and for values of
  $VPD$ greater than $VPD_{crit}$, ET will increase with increasing
  VPD ($\frac{\partial \; ET}{\partial \; VPD} > 0$). In other words,
  ecosystems regulate and mitigate evaporative losses up to the VPD
  limit, $VPD_{crit}$, above which atmospheric demand is just too high
  to be entirely compensated by stomatal and ecosystem regulation. We
  note however that even though ET increases again above the critical
  threshold, $VPD_{crit}$, ET is still much lower than potential
  evaporation as stomata are still strongly regulating vapor fluxes to
  the atmosphere. However, even in the absence of soil pore
  evaporation ET cannot go completely to zero at high VPD, because
  stomata are still slightly open to perform some photosynthesis
  \citep{Ball_1987, Leuning_1990, MEDLYN_2011}. In addition, upward
  xylem transport is necessary to maintain phloem transport, as well
  as nutrient transport and thus carbon allocation \citep{De_2013,
    Nikinmaa_2013, Ryan_2014}.

\begin{table}
  \caption{Values of $VPD_{crit}$, where
    $\frac{\partial \; ET}{\partial \; VPD} = 0$, evaluated at PFT
    average values for $R_{air}$, $\gamma$, and $c_a$. PFT-specific
    constants ($g_1$, uWUE) are provided in Table \ref{pft}. For
    values of $VPD$ less than $VPD_{crit}$,
    $\frac{\partial \; ET}{\partial \; VPD}$ will be negative, and for
    values of $VPD$ greater than $VPD_{crit}$,
    $\frac{\partial \; ET}{\partial \; VPD}$ will be positive.}
  \centering
  \begin{tabular}{l c c c c c}
    \hline
    PFT & $R_{air}$ & $c_a$ (ppm) & $\gamma$  & \textbf{$VPD_{crit}$ (Pa)} \\
    \hline
    \input{vpd_crit.tex}
    \hline
  \end{tabular}
  \label{vpd_crit}
\end{table}


Differences in $VPD_{crit}$ are exclusively determined by uWUE and the
slope parameter ($g_1$) related to the plant functional type. A larger
uWUE means a smaller $VPD_{crit}$, and an ET response to increases VPD
that is more likely to be positive. At first glance this result is
somewhat counter-intuitive; we expect that plants with a higher water
use efficiency would be more water conservative. However, in reality
uWUE determines how $WUE$ changes with VPD:

\[WUE = \frac{GPP}{ET} = \frac{uWUE}{\sqrt{VPD}}.\]
\[\frac{\partial \; WUE}{\partial \; VPD} = -\frac{uWUE}{2 \;
    VPD^{3/2}}\]

So, plants with a higher uWUE will have a greater \textit{decrease} in
ecosystem-scale $WUE$ in response to increases in VPD. This decrease
in $WUE$ causes more water loss per unit carbon gain, and explains the
relationship between high uWUE and high likelihood of increases of ET
in response to increasing atmospheric drying (increases in VPD).

A tendency towards increasing ET response with increasing VPD can also
be caused by a high slope parameter ($g_1$), characteristic of plants
that at the leaf scale are more willing to trade water for access to
atmospheric CO$_2$. Plants that are less conservative will be thus be
more likely to increase ET with increasing VPD. Both the
aforementioned effects (large uWUE, $g_1$) can amplify each other, and
generally conspire to shift the sign term towards a positive value for
a given PFT.

This effect of uWUE and $g1$ on the sign term is most apparent by
comparing two extreme PFTs: water intensive crops (CRO) and water
conservative closed shrub (CSH). CRO has higher slope parameter and a
slightly higher uWUE ($g_1 = 140.7$ Pa$^{1/2}$; $2.85$ $\mu$-mol [C]
Pa$^{0.5}$ J$^{-1}$ [ET]) compared to CSH ($g1 = 75.1 \, Pa^{1/2}$,
$uWUE=2.82$ $\mu$-mol [C] Pa$^{0.5}$ J$^{-1}$ [ET]). These differences
in PFT parameters cause opposite ET responses to changes in VPD
between CRO and CSH. ET theoretically always decreases with increasing
VPD for the more water conservative CSH, while ET frequently increases
with increasing VPD for the more water intensive CRO (Fig.
\ref{idealized_sign}). CROs evolved or were bred to prioritize GPP and
yield and are thus not water conservative. They are very willing to
trade water for photosynthesis and productivity, despite changes in
VPD, while CSH are very unwilling to trade water for more
photosynthesis.

\begin{figure}
  \centering \includegraphics{./idealized_sign.pdf}
  \caption{The functional form of the sign term, with $\sigma$ held
    fixed at 1, and all terms except VPD set to PFT averages. For
    comparison, the observed range of VPD for each PFT is plotted
    below the x-axis. Stars denote 25th, 50th, and 75th percentiles,
    and the range of the line spans the 5th-95th percentiles of
    observed VPD. Vertical black lines denote the location of
    $VPD_{crit}$ for each PFT, with the exception of CSH and WSA, for which
    $VPD_{crit}$ is off-scale.}
  \label{idealized_sign}
\end{figure}


As expected, the slope parameter ($g_1$) is a primary determinant of
the VPD dependence for the sign term shown in Fig.
\ref{idealized_sign}. Plants that are more conservative (small $g_1$)
will tend to reduce ET with increasing VPD, and will be very effective
at reducing ET, especially at low VPD. However, at very high VPD,
gradients in vapor pressure at the leaf scale will become very strong
as stomata reach their limits of closure in response to VPD
(parameterized with $g_1$). As a result, ET response will begin to
asymptote towards a constant ecosystem-scale values as leaf-scale
response to VPD asymptotes towards zero.  Therefore, plants with a low
$g_1$ will have the largest VPD dependence of ET response because the
difference in ET response at low VPD (leaf stomatal response
dominates) and high VPD (VPD gradient dominates) is largest. This is
apparent in the strong VPD dependence of CSH, which has the lowest
slope parameter ($g_1=75.1$ Pa$^{1/2}$) (Fig. \ref{idealized_sign}).

To summarize our theoretical insights (Fig. \ref{idealized_sign} and
Table \ref{vpd_crit}), CROs are the least water conservative and have
the strongest overall tendency to increase ET with increasing VPD,
while CSH are the most water conservative and have the strongest
tendency to decrease ET with increasing VPD, as well as the strongest
VPD dependence of response. Fig. \ref{idealized_sign} clearly shows,
according to our theory, that for all PFTs except for crops there is
frequent occurrence of a negative (plant dominating) ET response to
increases in VPD. Therefore, plants are able in most atmospheric
conditions to reduce ET in response to increased VPD and thus to
reduce water loss. To better illustrate this, the ranges of observed
environmental VPDs at the FLUXNET sites are plotted parallel to the
x-axis. For CSH and WSA, VPD is always less than VPD$_{crit}$ (off
scale) so that the plant response dominates in typical environmental
conditions, emphasizing the water conservative strategy of those
plants. For CRO on the other hand, VPD is higher than VPD$_{crit}$ for
more than 50\% of observations, emphasizing that those plants operate
with an aggressive water usage strategy, are water intensive and were
actually engineered for photosynthesis rather than water saving. For
DBF, EBF, MF, GRA, and SAV more than half of the observed VPD are less
than VPD$_{crit}$, i.e. in conditions where plant response
dominates. SAV has a more water conservative response than the forest,
grass, and crop plan types, but still responds by increasing ET with
increasing VPD for about a quarter of observations, due to the high
aridity (VPD) of the SAV ecoclimate. It is also important to note that
for all PFTs, even when atmospheric demand dominates, ET response to
VPD is still far more negative than it would be for potential
evaporation $\partial PET/\partial VPD$, i.e. atmospheric demand only,
emphasizing that there is still a strong regulation of evaporative
flux by stomata and though the plant xylem. The sign term in the PET
case would just be a constant ($\frac{c_p}{R_{air}} \approx 3.5$),
which is far larger than any part of the curves for any PFT. Plants
are always regulating water exchange from the land surface, even when
they reach the limits of they ability to do so.

\subsection{Functional Form of the Scaling Term}
\label{scale_term}
While the above discussion of the sign of
$\frac{\partial \; ET}{\partial \; VPD}$ is important to answer our
question of when ET response increases or decreases with VPD,
understating the overall magnitude of the ET response is important to
soil-plant-atmosphere water budgeting. So we now more closely examine
the terms that affect how the sign term is scaled:

\begin{equation}
  \frac{g_a \; P}{T(\Delta + \gamma)}.
\end{equation}

$\frac{P}{T}$ is an air-density term, which varies little compared to
aerodynamic conductance and Clausius-Clapeyron ($\Delta$). The
psychometric constant ($\gamma$) is also relatively constant, so the
scaling term should be primarily a function of aerodynamic conductance
and temperature, through the Clausius-Clapeyron relationship
$\Delta$. This is as expected, given that the aerodynamic conductance
represents the efficiency of exchange between the surface and the
atmosphere. As aerodynamic conductance increases, any plant response
will be communicated more strongly to the atmosphere (and vice-versa).

$\Delta$'s presence in the scaling term also matches physical
intuition. $\Delta$ (and also the approximately constant $\gamma$)
control the efficiencies with which surface energy is converted to
latent and sensible heat \citep{Monteith_1965}. The functional from of
$\Delta$ will be the same across PFTs, but the temperature range may
vary slightly. In contrast, aerodynamic conductance will vary strongly
with PFT due to the importance of surface roughness for aerodynamic
conductance. So most of the differences in scaling between PFT should
be in the aerodynamic conductance term.

The control of the scaling term variability between PFTs by
aerodynamic conductance is confirmed by data (Fig.
\ref{scale_vary}). Differences between PFT are almost entirely due to
differences in aerodynamic conductance, rather than differences in
observed temperature ranges. The scaling term for the tree PFTs (DBF,
EBF, ENF, MF) is generally about double the scaling terms for other
PFTs which have lower surface roughness and generally smaller
aerodynamic conductance (GRA, CSH, CRO). The savannah (WSA, SAV) PFT's
scaling is somewhere between GRA, CSH, and CRO, and DBF, EBF, ENF, and
MF, due to higher variability and surface roughness.

\begin{figure}
  % \centering
  \centerline{\includegraphics[width=1.2\textwidth]{./idealized_scale.pdf}}
  \caption{Primary sources of variability for the scaling term, as a
    function of PFT. The 5th-95th percentile range of temperature is
    plotted at the 5th, 25th, 50th, 75th, and 95th percentiles of
    aerodynamic conductance, as observed for each PFT.}
  \label{scale_vary}
\end{figure}

Within each PFT, the scaling term variability is controlled both by
environmental temperature and aerodynamic conductance variability
(Fig. \ref{scale_vary}). While the observed variability of the
aerodynamic conductance contributes more to the scaling term
variability than temperature, the temperature contribution is
non-negligible. Specifically, the scaling term is generally larger at
low temperatures when latent heat is relatively inefficient at moving
energy away from the surface. This effect amplifies the role of
aerodynamic conductance variability at low temperatures.

To summarize, variability between PFTs is mostly controlled by
systematic differences in aerodynamic conductance, due to differences
in surface roughness between each PFT, and possibly to a lesser extent
wind conditions. In contrast, variability within PFT is also
controlled by temperature, through Clausius-Clapeyron. But,
aerodynamic conductance variability generally impacts the scaling term
more than temperature, even within PFTs.

\subsection{Bulk statistics of ET response to VPD}
\label{stats_sec}

In this section we consider direct observations of ET response with
eddy-covariance data, while including uncertainty with the $\sigma$
term (Sect. \ref{methods}). These observational results of ET
response (Table \ref{stats}) largely confirm our theoretical analysis, presented in the previous sections. For all PFTs, mean ET response to increasing VPD
is negative. However, ET response evaluated at the average of all
variables (e.g. $\sigma$, $T$, $c_a$, $VPD$) is positive for CRO, and
negative for all other PFTs. This difference in mean ET response as
compared to the ET response at mean environmental conditions is due to
the non-linear nature of the response, in which negative responses are
generally larger magnitude than positive responses (Fig.
\ref{idealized_sign}). Therefore, both the mean ET response as well as
the ET response at mean environmental conditions matches our
expectations from the theory (Sect. \ref{sign_term}), with the
exception that CRO observations are shifted more towards a negative ET
response than we expect.

\begin{table}
  \caption{Statistics of $\frac{\partial \; ET}{\partial \; VPD}$ as a
    function of PFT.}
  \centering
  \begin{tabular}{l c c c}
    \hline
    PFT & $\overline{\frac{\partial \; ET}{\partial \; VPD}}$ & $\frac{\partial \; ET}{\partial \; VPD}\left(\overline{env}\right)$ & fraction $\frac{\partial \; ET}{\partial \; VPD} < 0.$ \\
    \hline
    \input{stats.tex}
    \hline
  \end{tabular}
  \label{stats}
\end{table}

Regarding the frequency of negative and positive ET response, all PFTs
exhibit a decreasing ET response with increasing VPD (physiologically
controlled, water conservative response) for the majority of
observations. The more water conservative PFTs generally exhibit
higher frequency of negative ET response, especially when one factors
in the distribution of environmental VPD (e.g. SAV and WSA grow in
more arid climates, Fig. \ref{idealized_sign}). In general the bulk
statistics match our theoretical expectations well, with the caveat
that inclusion of uncertainty shifts crops towards a slightly more
negative response to VPD, and shifts many of the other PFTs, which still
exhibit a high frequency of negative ET response to VPD, towards more frequent
occurrence of positive ET response than the theory and Fig.
\ref{idealized_sign} would suggest. The bulk statistics motivate a
more thorough examination of the structure of uncertainty and more
sophisticated validation of our theory's performance against
observations.

\subsection{Validation of theory at eddy-covariance sites}
\label{testing}
We now compare more sophisticated distributions of the observed
response to our simplified theory (Sect. \ref{sign_term}). The
observed distribution of the sign term, as compared to what the theory
would predict, is provided in Fig. \ref{test_sign}. Our goal was to
capture the leading order behavior of the ET dependence on VPD. Given
the assumptions we made, and the uncertainties of flux tower
observations themselves, we expect a relatively large amount of noise
when reproducing the derivatives of ET. However, the data largely
reproduces our theoretical analysis.

\begin{figure}
  \centering
  \centerline{\includegraphics[width=1.2\textwidth]{./test_sign.pdf}}
  \caption{Comparison of the sign term with model uncertainty included
    (box plots) to the sign term as calculated with simplifying
    assumptions (blue line, as in Fig. \ref{idealized_sign}). Each
    box plot corresponds to 5\% of the data, and the 5-95\% range of
    VPD is plotted.}
  \label{test_sign}
\end{figure}

This is particularly true for DBF and MF; the theory matches the
leading order behavior of the function when uncertainty is included,
and the observations match the theory with the addition of noise. The
VPD dependence, given by the slope parameter ($g_1$), follows the
median values of each bin. Perhaps most importantly, the x-intercept,
and thus VPD$_{crit}$, matches nearly exactly between the theory and
the observations. Therefore the sign of the ET response to increases
in VPD should be well matched, subject to the unavoidable constraints
of noise, much of which comes from the observations themselves. The
uncertainty is non negligible; there are many observations in each bin
for which the the sign of observation is opposite the response
predicted by the theory, but to leading order our theory matches the
observations well.

While CSH has a much different functional form of the sign term than
DBF and MF, CSH observations also match our theory to leading order,
albeit with a bit more variability as a function of VPD. Again, the
VPD dependence mostly determined from the slope parameter ($g_1$)
closely matches the medians in the observation bins. The
VPD-independent, strongly negative response is also captured. For CSH,
there is rare occurrence of observed positive ET response with VPD
($\approx 6$\%), even with uncertainty, so the sign of the
observations almost always matches the sign of the theory, which
states that ET response should always be negative.

Biases between the theory and observations are similar for ENF, WSA,
CRO and SAV. At low VPD the theory and the observations match
well. However, at high VPD (upper 10-20\% of observations) the theory
is biased slightly towards positive response as compared to the
observations (Fig. \ref{test_sign}). For ENF this slight bias could
be explained by a negative bias in $g_1$. However, for WSA, CRO, and
SAV the observations in the highest VPD bins exhibit a downturn
towards more negative response, which cannot be captured by the
functional form of the sign term (Fig.
\ref{idealized_sign}). Therefore, to explain the observations one of
the variables in the sign term must change at extreme VPD (upper
10-20\%). The most likely candidate is $uWUE$, which we have assumed
constant to meet our goal of developing intuition for leading order
behavior, but might be expected to decrease at extremely low SWC. This extremely low SWC should be correlated with high VPD in very dry environments
through sensible heating increases \citep{Gentine_2016}. The theory
for EBF exhibits similar limitations as for ENF, WSA, and SAV, except
a larger portion of observations are biased negative as compared to
the theory ($\approx$ 35\%, Fig. \ref{test_sign}). However, in
general, and specifically for non-extreme VPD (VPD  $< \approx$70-90th
percentiles), the theory matches the observations for the tree and
savannah plant types well.

The theory for GRA suffers from similar, but much more severe,
limitations as for CRO, WSA, and SAV. GRA observations are
characterized by a consistent trend back towards negative ET response
at higher VPD, which the functional form of our theory is incapable of
accounting for. As compared to CRO, WSA and SAV, the divergence
between the theory and the observations is far greater for GRA,
biasing 40-50\% of observations at higher VPD. In addition for the
potential for soil moisture to alter $uWUE$, there are other sources
of plant heterogeneity specific to GRA (and some extent CRO) that may
alter $uWUE$ (or $g_1$) or invalidate other assumptions made in the
methods section (Sect. \ref{methods}). We do not account for
variability in plant height and surface roughness, or differences in
C3 vs C4 photosynthesis and water strategies, which we might expect to
vary substantially across sites, years, and season for GRA. These
deficiencies could largely explain the inability of our theory to
exactly match the observations in croplands. For example, a
superposition of sites with C3 plants (at low environmental VPD) and
C4 plants (at high environmental VPD) would explain the observed shift
back towards negative ET response at high VPD when all sites are
binned together, as in Fig. \ref{test_sign}. We hypothesize that the
theory would validate against observations much better if these
sources of variability were accounted for, at a cost of increased
complexity and analytical opacity.


While the above discussion shows that our theory has some limitations
when applied to some PFTs, especially for grasslands, it does well for
DBF, MF and CSH PFTs, and captures the response at non-extreme VPD
(VPD $< \approx$ 70th-90th percentiles) for
ENF, EBF, CRO, WSA, and SAV. In general, the leading order behavior
observed in the data is captured by the theory. Departures between our
theory and observations, specifically at high VPD, could be explained
and conceptualized with shifts in $g_1$ and/or $uWUE$ due to
site-specific plant type variability (e.g. more arid-adapted ecosystems
at more arid sites), or temporal variability for some environmental
conditions (e.g. decreases in $uWUE$ at extremely low soil
moisture). To focus on general behavior and develop intuition for
PFT-scale response, we have ignored these sources of variability in
the present analysis. However, any site-specific or temporal plant
functional variability that can be conceptualized with shifts to $g_1$
and $uWUE$ can be analyzed with our framework. This opens a door to future analyses in which plant behavior in anomalous conditions
can be explained and analyzed using Eq. (\ref{et}).

\subsection{Observed ET response to VPD}

Most of the results presented so far focus on the sign term, so now we
turn to observations of ET response with the scaling term included
(Fig. \ref{data_scatter}). Until now, in the interest of developing
leading order intuition for ET response, and to be conservative in our
acknowledgment of model and observational error, we've considered
$\sigma$ variability to be a measure of uncertainty. An alternative
viewpoint is that $\sigma \cdot uWUE$ represents spatial and temporal
variability of uWUE, which may be expected within bounds \citep[see
Table \ref{pft}, also ][]{Zhou_2015}. This is a less conservative
view; some of the $\sigma \cdot uWUE$ variability will be due to model
and observational error, so by viewing $\sigma \cdot uWUE$ as
\textit{real} variability we run the risk of mistaking noise for
signal. However, the advantage of this viewpoint is that, from
\citet{Zhou_2014}, we have very high confidence that uWUE fit at the
hourly timescale (as we do with $\sigma \cdot uWUE$) correctly
captures the relationship between ET, GPP, and VPD, and that our form
of PM introduces minimal error with its use of uWUE.

\begin{figure}
  \centering
  \centerline{\includegraphics[width=1.2\textwidth]{./data_scatter.png}}
  \caption{Scatter plots of observed
    $\frac{\partial \; ET}{\partial \; VPD}$, including $\sigma$
    variability, as a function of PFT, temperature and VPD. Please
    note differences in the colorbar scale.}
  \label{data_scatter}
\end{figure}

Therefore, with the caveat that some of the signal presented in Fig.
\ref{data_scatter} may in fact be noise, we can interpret the observed
distribution of ET response to VPD. In general, the observed response
matches the intuitive theory. ET response to VPD shifts towards
positive values as VPD increases (atmospheric demand dominating). CRO
exhibit the highest occurrence of positive ET response, and the
observations confirm that CRO are the most water intensive. CSH are
the most water conservative, with a strong negative response. DBF,
EBF, ENF, MF, SAV and WSA are also water conservative, but show some
occurrence of positive ET response to VPD, particularly at higher
observed VPD, as the theory predicts. GRA, while generally water
conservative, does not match the theory well, with increasing frequency of
\textit{negative} response at high VPD. This is as expected, given
previous discussion on how we might expect more inter-site and inter-year variability for GRA (Sect. \ref{testing}).

Fig. \ref{data_scatter} also includes the impact of the scaling
term. For a given VPD, the magnitude of ET response does not vary
strongly with temperature, confirming that any impact of the scaling
term on the magnitude of the response is primarily due to changes in
aerodynamic conductance. Intuitively this is reasonable; aerodynamic
conductance will control how dominant balances at the land surface are
communicated to the boundary layer. While $\Delta$ controls the
efficiency of energy conversion to latent heat, it appears this is a
second order term, relative to $g_a$, for scaling ET response.

As with Fig. \ref{test_sign}, Fig. \ref{data_scatter} matches our
expectation based on simplified theory. The sign term is most strongly
scaled by $g_a$, and in general the occurrence of positive ET response
increases as VPD increases. The willingness with which a given plant type
evolved to use water dictates the occurrence of positive versus
negative ET response. Water conservative ecosystems are highly effective
at mitigating the effects of atmospheric demand, and can store water
for later use by reducing ET in response to increasing VPD.

\subsection{Limitations of theory: very dry soil moisture conditions}
\label{swc_section}
In formulating our theory with Penman Monteith, we implicitly did not
account for very dry soil moisture conditions. For the majority of
environmental conditions observed at the eddy covariance sites used
here, soil conditions were not extremely dry so that we could assume a
constant uWUE and $g_1$. We posit that ecosystems will generally
optimize to host plants living in conditions which they evolved
for. However, in extreme conditions and drought scenarios soil water
content (SWC) could become the limiting factor for ET response to VPD,
which our theory does not account for. In addition, low soil moisture
conditions themselves increase VPD through land-atmosphere feedback
\citep[][]{Bouchet_1963, Morton_1965, Brutsaert_1999, Ozdogan_2006,
  Salvucci_2013, Gentine_2016, Berg_2016}.

Within our framework, any systematic bias due to the
failure to account for SWC's effects in very dry conditions should
manifest itself in a functional relationship between $\sigma$, our
uncertainty measure, and SWC, which is observed at the FLUXNET
sites. Examining the relationship between $\sigma$ and SWC will test
to what extent our theory breaks down in very dry soil moisture
conditions.

If we again view $\sigma \cdot uWUE$ as, in addition to a measure of
uncertainty, a short time-scale observation of uWUE, we would expect
$\sigma \cdot uWUE$ to decrease at low SWC. If $\sigma \cdot uWUE$ is
a very strong function of SWC, then our theory should be conditioned
more strongly on well watered soil conditions. If $\sigma \cdot uWUE$
is weakly a function of SWC, then our theory is more universal and
independent of soil moisture conditions.

Indeed, for all PFTs, there is some slight dependence of $\sigma \cdot
uWUE$ on SWC, especially at low SWC (Fig. \ref{swc_boxplot}). The
portion of observations for which our theory is biased by
SWC-limitations varies by PFT, due to the nonlinear threshold at which
soil moisture availability limits plant function. For CRO, MF and DBF,
soil water content only matters for about the lowest 5\% of observations
(each box is 5\% of observations in Fig. \ref{swc_boxplot}). And, CRO,
relative to DBF, has a very weak dependence on soil moisture, which is
reflective of the high likelihood that CRO sites are optimally irrigated
and not water stressed, suggesting that observed departures from the
theory for CRO are due to other factors (e.g. different photosynthesis
pathways; C3/C4) than soil moisture. For ENF, EBF, SAV, and WSA,
systematic SWC-induced biases in $\sigma \cdot uWUE$ emerge in about the
lowest 20-30\% of SWC conditions, although large variability in the
SWC-$(\sigma \cdot uWUE)$ relationship hampers interpretation for
WSA. CSH presents a special case: the limited number of sites (2)
preclude comment on the exact PFT-wide relationship.

In contrast to ENF, CRO, and DBF, for GRA the relationship between
$\sigma \cdot uWUE$ and SWC is more linear and affects a greater
portion of observations; about 60\% of observations. Clearly, for GRA
soil water frequently impacts plant function and alters ET response,
and our theory is limited for the majority of environmental
conditions. It is therefore not surprising that our theory tested
poorly against the data for GRA, relative to to the other PFTs
(Sect. \ref{testing}). For all other PFTs occurrence of soil
moisture impacts was rare enough to not manifest itself in bulk
statistics and figures.

\begin{figure}
  \centering
  \centerline{\includegraphics[width=1.2\textwidth]{./swc_boxplot.pdf}}
  \caption{$\sigma \cdot uWUE$ as a function of PFT and SWC. Each box
    plot represents 5\% of all observations. Note that the highest
    10\% of SWC observations are excluded to better resolve variability in the much more narrow bins at lower SWC (e.g. SWC has long tails at high values).}
  \label{swc_boxplot}
\end{figure}

The observed dependence of $\sigma \cdot uWUE$ on SWC for GRA would
explain the deficiencies of our theory compared to the observations in
Fig. \ref{test_sign}, specifically the trend back towards
negative ET response at high VPD. Aforementioned feedbacks between
the land surface and the atmosphere, which are not accounted for due
to our focus on the one-way response of the land surface to
atmospheric conditions, would cause high VPD to be correlated with low
SWC \citep[][]{Gentine_2016, Berg_2016}. So, at high VPD observations
of low SWC are more likely, and this low SWC causes a lower uWUE
(Fig. \ref{swc_boxplot}). The lower uWUE at low SWC/high VPD then
leads to the observed downturn towards decreasing ET with increasing
VPD, and the deviation between our theory (based on a constant uWUE
assumption) and observations. It is also perhaps not coincidental that
the portion of $\sigma \cdot uWUE$ affected by low soil moisture
observations is similar to the portion of observations that do not
match our theory at high VPD. Indeed, this may suggest that for all PFTs
except for GRA, coupling and feedbacks between SWC, VPD and plant
function are relatively rare. Future research will explore these relatively rare feedbacks in extreme conditions, which due to analytical intractability will require more opaque numerical analysis of many more complex processes, including boundary layer growth and state and their relationship to surface layer coupling and free-tropospheric lapse rates and humidity.

To summarize, our theory is limited by its inability to account for
soil water impacts on land surface response, and feedbacks between SWC
and VPD. Fortunately, for most PFTs SWC's effect on ET is relatively
rare ($<$30\% of observations) and does not manifest itself in the
majority of observations and bulk statistics. However for GRA, SWC decreases water use efficiency for the majority of
the observations. Soil moisture effects explain the deficiencies of
our theory in Sect. \ref{testing}, particularly for GRA. By
conceptualizing SWC effects as a change in $uWUE$ (and/or $g_1$), it
will be possible for future analysis to explore the importance of soil
moisture on plant response to VPD, and feedbacks between plant
function, SWC, and VPD.

\subsection{Functional form of ET dependence on VPD and its relation
  to the VPD exponent}
\label{functional_form}

The theory described in Sect. \ref{sign_term} indicates that for a
given $uWUE$ and $g_1$, the ET dependence on VPD should be concave
upward, which is confirmed by eddy covariance data across most PFTs. In other words, there should be some local minimum in ET at a
critical VPD$_{crit}$, assuming the scaling and plant terms  (e.g. aerodynamic
conductance, $\Delta$, $g_1$ and $uWUE$) are held fixed. This result warrants further
investigation, because to our knowledge no one has derived the
theoretical ecosystem-scale relationship between ET and VPD while
controlling for other environmental conditions. In particular, from
personal communication, there is an apparent lack of consensus over
whether the shape of the ET-VPD curve should be concave upward (our
result) or concave downward in the absence of dramatic water stress. Given that understanding the ET-VPD
relationship of the one-way plant response is fundamental to
hypothesizing about any feedbacks between the land surface and the
atmosphere, we analyze why our derived ET-VPD relationship is concave
upward, particularly with respect to the exponent of VPD dependence in
$uWUE$ and the Medlyn unified stomatal conductance model.

There is a theoretical basis for the square root VPD dependence in
both the stomatal conductance model and $uWUE$ based on the assumption
that stomata behave to maximize carbon gain while minimizing water
loss, which observations also generally support \citep{Lloyd_1991,
  MEDLYN_2011, Lin_2015, Zhou_2014, Zhou_2015, Medlyn_2017}. However,
some purely empirical results that fit the exponent of the VPD
dependence to data have shown that it may vary slightly from 1/2,
suggesting that stomata, as well as ecosystem-scale quantities based
on stomata theory, may not always function optimally \citep{Zhou_2015,
  Lin_2018}. Specifically with regards to $uWUE$, one would not expect
that this ecosystem scale WUE quantity will respond to VPD exactly
analogously to stomata. Direct soil evaporation's contributions to ET
should shift the exponent of the VPD dependence, especially at
conditions of low GPP when we would expect a systematically larger
portion of direct soil evaporation contributions to ET, because we
would also expect lower amounts of transpiration at low GPP. \citet{Zhou_2015}'s
results corroborate this: they found a mean empirically fit
exponential VPD dependence of 0.55, varying slightly from the
theoretically optimal value of of 0.5 for AmeriFlux
sites. Interpreting \citet{Lin_2018}'s results, which also show
variance in the empirical exponent of the VPD dependence of the
stomatal conductance model, is more difficult as \citet{Lin_2018} do
not handle GPP/A dependence of stomatal conductance in a directly
analogous manner to the optimal theory in \citet{MEDLYN_2011} and
\citet{Medlyn_2017}. Regardless, given that these recent results on
the relationship between VPD, GPP, and ET \citep{MEDLYN_2011,
  Zhou_2014, Zhou_2015, Medlyn_2017} form the backbone of our analysis
and are what allowed us to derive an explicit ET expression for the
first time (Eq. (\ref{et})), we will analyze if and how assumptions
about the exponent of the VPD dependence impacts the shape of the
ET-VPD dependence. This analysis is also important to understand whether the choice of stomatal conductance model alters the fundamental behavior of the ET-VPD relationships, as many commonly used models  utilize a VPD exponent other than the 1/2 suggested by optimal theory \citep[e.g. ][ which uses an exponent of 1]{Leuning_1990}.

By introducing $n$ and $m$ we can free our stomatal
conductance model from assumptions about VPD dependence:
% \begin{linenomath*}
  \begin{equation}
    g_s = \frac{R \, T}{P} 1.6 \left(1 + \frac{g*}{VPD^m}\right) \frac{*WUE \; ET}{c_a \; VPD^n},
    \label{m_n}
  \end{equation}
% \end{linenomath*}
where:
\[*WUE = \frac{GPP}{ET}VPD^n,\] and $g*$ is a generic slope parameter
of units $VPD^m$. To determine how the exponent $n$ and $m$ alter the
shape of the ET-VPD dependence we find the roots of the second
derivative of ET, using Eq. (\ref{m_n}) for stomatal conductance
($g_s$), with respect to VPD: \input{d2_solutions.tex} With this
result we have defined the family of curves separating concave up from
concave down ET solutions (Fig. \ref{concave}). These curves are
only functions of the exponent of the VPD dependence and a quantity we
call non-dimensional VPD ($VPD^m/g*$). Several important
relations reveal themselves from Eq. (\ref{curves}):

\begin{figure}
  \centering
  \centerline{\includegraphics[width=0.75\textwidth]{./concave.pdf}}
  \caption{ Solutions corresponding to inflection points between
    concave up and concave down ET-VPD curves (Eq. (\ref{curves}))
    for three specific scenarios. Solutions are defined in terms of a
    non-dimensional VPD ($VPD^m/g_*$), but to aide physical
    interpretation the horizontal axis is additionally provided in terms of dimensionalized VPD assuming $m=1/2$ and
    $g_*=110\; Pa^{1/2}$ (average of all PFT $g_1$). The vertical axis
    has a different interpretation depending on the solution
    curve. For the blue line ($m$ varying), it corresponds to $m$,
    for the orange line ($n$ varying) it corresponds to $n$, and for
    the green line it corresponds to the value of both $n$ and $m$
    ($n=m$). Regions of the parameter space that correspond to
    concave up and concave down results are labeled.}
  \label{concave}
\end{figure}

\begin{itemize}
  \item For optimal behavior (n, m = 1/2) the ET-VPD curve will be
    concave up regardless of the magnitude of the plant constants
    $g_1$ and $uWUE$. Therefore, the general concave up nature of our
    results, given an assumption of optimal behavior, is insensitive
    to plant type.
  \item For all physically possible exponents of VPD dependence ($n,
    m$), whether the solution is concave up or concave down does not
    depend on $uWUE$.
  \item In general, increasing the exponent of VPD dependence
    increases the likelihood of a concave down result. Additionally, as
    the exponent of VPD dependence increases from the optimum value of
    1/2, whether the curve is concave upward or concave downward
    becomes a function of the plant specific slope parameter $g_*$,
    through non-dimensional VPD ($VPD^m/g_*$). Because the exponent of
    the VPD dependencies is capable of altering the fundamental shape
    of ET-VPD dependence, future research investment in understanding
    the exact VPD dependence of stomatal conductance, and further
    reconciliation of empirical and theoretical stomatal and
    ecosystem behavior should be prioritized.
\end{itemize}
While it is possible that in the future some other form of
VPD dependence is derived, at present \cite{MEDLYN_2011} and
\cite{Zhou_2014} firmly established n=m=1/2 as the most likely
candidate given current theory and empirical data. Additionally, we
argue that a concave up result matches physical intuition more than a
concave down result. Plants must maintain nutrient and sugar transport
through the phloem and xylem. To accomplish this, stomata must remain
slightly open \citep{De_2013, Nikinmaa_2013, Ryan_2014}. Furthermore,
even if complete stomatal closure were possible, cuticular water loss
and [at the ecosystem-scale] direct soil evaporation are still sources
of ET which increase with VPD, independent of stomatal
closure. Therefore, in the limit as VPD becomes large and we assume
plants are exercising all strategies to reduce ET, any further increase
in VPD should result in an increase in ET through cuticular water loss
and/or direct soil evaporation. This inevitable transition from
conditions when stomata respond strongly to VPD to conditions when
stomata response is asymptoting towards full closure would cause a
concave up ET-VPD curve, which is matched by the theory. In short,
plant response becomes more limited as VPD increases, while
atmospheric demand monotonically increases with VPD, leading to the
 result that atmospheric demand dominates plant response when
atmospheric demand is high.

This analysis allows us to understand the theoretical shape of the ET
response to VPD with environmental conditions held
fixed. Accomplishing this with purely statistical
methods applied to flux observations would be very difficult, given
the relatively fast time scale of plant response and the
non-stationarity of [solar forced] environmental conditions over the
relatively coarse (half hourly) flux estimates (which is required to
obtain robust eddy-covariance statistics). Our results on the shape of
the ET-VPD curve with environmental conditions held
fixed can be built upon with future work examining how changes in VPD
and environmental conditions (e.g. soil water storage) feedback upon
one another. In the soil water storage example, over very long time
scales extremely high VPD perturbations coupled with no precipitation
could result in decreases in soil water storage such that water
becomes limiting. This could be represented by an extension of our
framework in which uWUE is allowed to decrease with decreasing SWC, as
observed in Sect. \ref{swc_section}. Here, we focus our results by
assuming constant PFT-wide conditions to build baseline intuition for
ET-VPD dependence. For most PFTs, the theory with plant function held
fixed matches the leading order behavior of the observations where
plant function varies (Sect. \ref{testing}).
