\documentclass[a4paper,11pt]{article}
\usepackage{amsmath,amsfonts,amsthm,dsfont,amssymb}
\usepackage[blocks]{authblk}
\newenvironment{keywords}{\noindent\textbf{Keywords:}}{}
\newenvironment{classification}{\noindent\textbf{AMS subject classifications.}}{}
\date{}
\newcommand{\email}[1]{\texttt{\small #1}}

\begin{document}
% % % % %--------------------------------------------------------------------
% % % % %          Title of the Paper and Acknowledgement
% % % % %--------------------------------------------------------------------
  \title{When does vapor pressure deficit drive or reduce evapotranspiration?\footnote{Oral presentation preferred}}
% % % % %--------------------------------------------------------------------
% % % % %         Authors,, Affiliations and email ids
% % % % %--------------------------------------------------------------------

\author[1]{\underline{Adam Massmann}}
\author[1]{Pierre Gentine}
\author[2]{Changjie Lin}

\affil[1]{Department of Earth and Environmental Engineering, Columbia University, New York, NY 10027

  \textit{Contact:} \email{akm2203@columbia.edu}}
\affil[2]{Department of Hydraulic Engineering, Tsinghua University, Beijing, CN}
% % % % %--------------------------------------------------------------------


	\maketitle
	
\begin{abstract}
  Depending on plant response (e.g. stomatal closure), ecosystem-scale evapotranspiration can either increase or decrease with changes in vapor pressure deficit. This ecosystem response drives evapotranspiration and atmospheric moisture feedbacks. We use data from 75 FluxNet sites within a Penman-Monteith framework to examine when ecosystem evapotranspiration is suppressed or enhanced by increases in vapor pressure deficit. Evapotranspiration response is quantified as a function of soil moisture, atmospheric conditions, and plant functional type. The framework is designed so model and observational uncertainties are represented in the results. This in-situ observation-based analysis aids understanding for how ecosystems will respond and/or contribute to future shifts in atmospheric water demand.


\end{abstract}

% \begin{keywords}
% ginverse, drazin inverse, diffrential equation
% \end{keywords}

% \begin{classification}
% 13C10; 15A09; 15A24; 15B57
% \end{classification}
%  \begin{thebibliography}{100}
 
%  \bibitem{einstein} 
%  Albert Einstein. 
%  {\em Zur Elektrodynamik bewegter K{\"o}rper}. 
%  Journal of Mathematics, 322(10):89--921, 1905.


%   \bibitem{impj}  The Japan Reader {\em Imperial Japan 1800-1945} 1973:
%   Random House, N.Y.

%   \bibitem{norman} E. H. Norman {\em Japan's emergence as a modern
%   state} 1940: International Secretariat, Institute of Pacific
%   Relations.

%   \bibitem{fo} Bob Tadashi Wakabayashi {\em Anti-Foreignism and Western
%   Learning in Early-Modern Japan} 1986: Harvard University Press.

%   \end{thebibliography}

\end{document}

