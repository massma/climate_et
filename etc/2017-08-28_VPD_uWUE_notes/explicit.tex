\usepackage{amsmath}
\usepackage{amssymb}
\usepackage{bm}
\usepackage{graphicx}
\usepackage{epstopdf}
\usepackage{setspace}
\DeclareGraphicsRule{.tif}{png}{.png}{`convert #1 `basename #1 .tif`.png}
\usepackage{color}
\pagestyle{plain}

% am additions
\usepackage{makecell}
\usepackage{multirow}
\usepackage{booktabs}
\usepackage{authblk}
\usepackage{textcomp}
\usepackage{hyperref}
\RequirePackage{natbib}

\bibliographystyle{agufull08}

\def\acknowledgments{\vskip12pt\noindent{\bf
Acknowledgments\vrule depth 6pt
width0pt\relax}\\*\noindent\ignorespaces}

\newcommand{\appropto}{\mathrel{\vcenter{
      \offinterlineskip\halign{\hfil$##$\cr
        \propto\cr\noalign{\kern2pt}\sim\cr\noalign{\kern-2pt}}}}}


\textheight 9 true in
\textwidth 6.5 true in
\hoffset -0.5 true in
\voffset -.75 true in
 
\mathsurround=2pt  \parskip=2pt
\def\crv{\cr\noalign{\vskip7pt}} 
\def\a{\alpha } \def\b{\beta } \def\d{\delta } \def\D{\Delta } \def\e{\epsilon }
\def\g{\gamma } \def\G{\Gamma} \def\k{\kappa} \def\l{\lambda } \def\L{\Lambda }
\def\th{\theta } \def\Th{\Theta} \def\r{\rho} \def\o{\omega} \def\O{\Omega}
\def\ve{\varepsilon} 

\def\sA{{\cal A}} \def\sB{{\cal B}} \def\sC{{\cal C}} \def\sI{{\cal I}}
\def\sR{{\cal R}} \def\sF{{\cal F}} \def\sG{{\cal G}} \def\sM{{\cal M}}
\def\sT{{\cal T}} \def\sH{{\cal H}} \def\sD{{\cal D}} \def\sW{{\cal W}}
\def\sL{{\cal L}} \def\sP{{\cal P}} \def\s{\sigma } \def\S{\Sigma}
\def\sU{{\cal U}} \def\sV{{\cal V}} \def\sY{{\cal Y}}

\def\gm{\gamma -1}
\def\summ{\sum_{j=1}^4}

\def\bb{{\bm b}} \def\yb{{\bm y}}
\def\ub{{\bm u}}  \def\xb{{\bm x}} \def\vb{{\bm v}} \def\wb{{\bm w}}
\def\omegab{{\bm \omega}} \def\rb{{\bm r}} \def\ib{{\bm i}} \def\jb{{\bm j}}
\def\lb{{\bm l}} \def\kb{{\bm k}} \def\Ab{{\bm A}} \def\fb{{\bm f}} \def\Ub{{\bm U}}
\def\Fb{{\bm F}} \def\nb{{\bm n}} \def\Db{{\bm D}} \def\eb{{\bm e}}
\def\gb{{\bm g}}  \def\Gb{{\bm G}} \def\hb{{\bm h}} \def\Yb{{\bm Y}} \def\Rb{{\bm R}} 
\def\Tb{{\bm T}}

\def\As1{{\bf {\cal A}}_1}\def\DO{{\cal D}_0} \def\UO{{\cal U}_0}
\def\ie{{\it{i.e.}}}

\def\ubbar{{\bf {\bar{u}}}} \def\sbar{{\bar{\sigma }}} \def\ubar{{\bar{u}}}  
\def\abar{{\bar{a}}} \def\vbar{{\bar{v}}}  \def\rbar{{\bar{\rho}}}
\def\pbar{{\bar{p}}} \def\ebar{{\bar{e}}} \def\Tbar{{\bar{T}}}
\def\bbar{{\bar{\beta}}} \def\Mbar{{\bar{M}}}  \def \sMbar{{\bar{\cal M}}}
\def\Ebar{{\bar{E}}} \def\sMbar{{\bar{\cal M}}}
\def\sPbar{{\bar{\cal P}}} \def\xbar{{\bar{x}}}

\newcommand{\pdv}[2]{\frac{\partial#1}{\partial#2}}
\newcommand{\dv}[2]{\frac{d#1}{d#2}}
\newcommand{\ord}[2]{#1^{(#2)}}
\newcommand{\vct}[1]{\vec{#1}}

 \newcommand{\bc}{\begin{center}}
 \newcommand{\ec}{\end{center}}
 
 \newcommand{\bq}{\begin{equation}}
 \newcommand{\eq}{\end{equation}}
 
 \newcommand{\beqs}{\begin{eqnarray}}
 \newcommand{\eeqs}{\end{eqnarray}}
 
 \newcommand{\beqa}{\begin{eqnarray*}}
 \newcommand{\eeqa}{\end{eqnarray*}}
 
 \newcommand{\ol}{\overline}
 \newcommand{\ul}{\underline}
 
 \newcommand{\dint}{{\int \!\! \int \!\!}}
 \newcommand{\tint}{{\int \!\! \int \!\! \int \!\!}}
 
 \newcommand{\bfig}{\begin{figure}}
 \newcommand{\efig}{\end{figure}}
 
 \newcommand{\cen}{\centering}
 \newcommand{\n}{\noindent}
 
 \newcommand{\btab}{\begin{table}}
 \newcommand{\etab}{\end{table}}
 
 \newcommand{\btbl}{\begin{tabular}}
 \newcommand{\etbl}{\end{tabular}}
 
 \newcommand{\bdes}{\begin{description}}
 \newcommand{\edes}{\end{description}}
 
 \newcommand{\benum}{\begin{enumerate}}
 \newcommand{\eenum}{\end{enumerate}}
 
 \newcommand{\bite}{\begin{itemize}}
 \newcommand{\eite}{\end{itemize}}
 
 \newcommand{\cle}{\clearpage}
 \newcommand{\npg}{\newpage}
 
 \newcommand{\bss}{\begin{singlespace}}
 \newcommand{\ess}{\end{singlespace}}
 
 \newcommand{\bhalf}{\begin{onehalfspace}}
 \newcommand{\ehalf}{\end{onehalfspace}}
 
 \newcommand{\bds}{\begin{doublespace}}
 \newcommand{\eds}{\end{doublespace}}
 
 \newcommand{\eps}{\mbox{$\epsilon$}} 
 \newcommand{\stilde}{\mbox{$\tilde s$}} 
 \newcommand{\shat}{\mbox{$\hat s$}} 

 \newcommand{\blue}{\color{blue}}
 \newcommand{\red}{\color{red}}
 \newcommand{\magenta}{\color{magenta}}
 \newcommand{\green}{\color{green}}
 \newcommand{\nc}{\normalcolor}




%\usepackage{amsmath}
\usepackage{graphics, graphicx}
\graphicspath{ {./} } %directory for figures 
%\pagestyle{empty}

\begin{document}
\begin{center}
\large{Using uWUE to analyticaly evaluate VPD effect on WUE, GPP, and ET}\end{center}

\bigskip

This document is to analytically caluclate ET and $\frac{\partial ET} {\partial VPD}$.

Penman montieth is given by:

\begin{equation}
  ET = \frac{\Delta R + g_a \rho_a c_p D_{s}}{\Delta + \gamma(1 + \frac{g_a}{g_s})}
\end{equation}

where from Medlyn we have:
\begin{equation}
  g_s = \text{LAI } \frac{R T}{P} 1.6 \left(1 + \frac{g_1}{\sqrt{D_{s}}}\right) \frac{A}{c_s}
\end{equation}

Here, $g_{1M}$ has units of kPa$^{0.5}$, $A$ is the net CO$_2$ assimilation rate in $\mu$ mol m$^{-2}$ s$^{-1}$, $c_s$ is the CO$_2$ concentration in ppm (supposed to be at leaf surface). The problem is the $A$ term that we need to get rid of. Also $g_w$ as above has units of mol (air?) m$^{-2}$ s$^{-1}$. 

we can use:

\begin{equation}
  \text{uWUE} = \frac{\text{GPP} \cdot \sqrt{\text{VPD}}}{ET}
\end{equation}

As long as we put uWUE into $\mu mol$, we can use the uWUE provided by Zhou et. al. To do this just multiply be a factor $ \frac{1 mol C}{12.011 g C} \frac{1.e6 \mu mol}{1 mol}$. Then make sure that units of VPD match. Note the table in Franks et al. 2017 and Zhou et al. have VPD in kPa and hPa, respectively.

Then we can show that if we multiply $g_s$ by $\frac{ uWUE \; ET}{A \sqrt{D_s} L_v}$ (which is equivalent to 1 with unit conversion above), we can rearrange Penman Moneith to get:

\begin{equation}
  ET = \frac{\Delta R + \frac{g_a\; P}{T} \left( \frac{ c_p D_{s}}{R_{air}} - \frac{\gamma c_s \sqrt{D_s} L_v }{\text{LAI } R* \; 1.6 \text{ uWUE } (1 + \frac{g_1}{\sqrt{D_s}})} \right) }{ \Delta + \gamma}
\end{equation}

If we want to solve for a ``pseudo-LAI'' (in practice this will also be a function of noise and model error in addition to leaf area) as a function of all other knowns, we have:
\begin{equation}
LAI  = - \frac{g_a \gamma c_s \sqrt{D_s} L_v P }{ \left(\text{ ET } ( \Delta + \gamma) - \Delta R - g_a \rho_a c_p D_{s}\right) 1.6 \; R\; T\; \text{ uWUE } (1 + \frac{g_1}{\sqrt{D_s}})}
\end{equation}

Now also taking the derivative of ET w.r.t. $D_s$ we get:

\begin{equation}
\frac{\partial \;  ET}{\partial \; D_s} = \frac{g_a \; P}{T(\Delta + \gamma)}   \left(\frac{ c_p}{R_{air}} - \frac{\gamma c_s L_v  }{\text{LAI }1.6 \; R\; \text{ uWUE }} \left( \frac{2 g_1 + \sqrt{D_s}}{2 (g_1 + \sqrt{D_s})^2}\right) \right)
\end{equation}

Now that we have ET, we can also use uWUE to get GPP:

\begin{equation}
  \frac{ET}{L_v} = \frac{\text{GPP} \cdot \sqrt{D_s}}{  \text{uWUE}}
\end{equation}

Plugging this into Equation (4) gives:

\begin{equation}
  \text{GPP} = \frac{\text{uWUE}\; \Delta R + g_a \left( \text{uWUE}\; \rho_a c_p D_{s} - \frac{\gamma c_s \sqrt{D_s} L_v P }{\text{LAI } R \; T \; 1.6  (1 + \frac{g_1}{\sqrt{D_s}})} \right) }{L_v \; \sqrt{D_s} \left(\Delta + \gamma \right)}
\end{equation}

Taking the derivative w.r.t $D_s$ gives:

\begin{equation}
  \frac{\partial GPP}{\partial D_s} = \frac{1}{2 L_v D_s^{3/2} \left(\Delta + \gamma \right)} \left( -\text{uWUE}\; \Delta R + g_a \text{uWUE}\; \rho_a c_p  D_s - \frac{g_a g_1  \gamma c_s L_v P }{\text{LAI } R \; T \; 1.6 ( 1 + \frac{g_1}{\sqrt{D_s}})^2 } \right)
\end{equation}

We can also solve for WUE, again using uWUE (just make sure units are correct: mol C/mol H$_2$O):

\begin{equation}
\frac{GPP}{ET} = \frac{\text{uWUE}}{\sqrt{D_s}}
\end{equation}

And the derivative:

\begin{equation}
  \frac{\partial \text{WUE}}{\partial D_s} = -\frac{\text{uWUE}}{2 D_s^{3/2}}
\end{equation}


\end{document}